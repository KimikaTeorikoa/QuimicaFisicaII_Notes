\chapter{Apéndice: Conceptos Matemáticos y de la Física Clásica
Importantes en Mecánica Cuántica}
\section{La Relación de Euler}
A menudo encontraremos en nuestras funciones de onda, términos
exponenciales imaginarios, como $\mathrm{e}^{\mathrm{i}\theta}$.
La relación de Euler (1707-1783), a la que Feynmann llamó "la
fórmula más sobresaliente de las matemáticas" permite expresar
este tipo de exponenciales a partir de funciones trigonométricas:
\begin{equation}
    \mathrm{e}^{\mathrm{i}\theta} = \cos{\theta} + \mathrm{i}\sin{\theta}
\end{equation}
Asímismo, para cualquier número imaginario $z=x + \mathrm{i}y$,
podemos escribir
\begin{equation}
    z=x + \mathrm{i}y = r(\cos{\theta} + \mathrm{i}\sin{\theta})=
    \mathrm{e}^{\mathrm{i}\theta}
\end{equation}
donde $\theta$ es el ángulo formado por la contribución 
real, $\mathrm{Re}(z)$, y la contribución imaginaria, 
$\mathrm{Im}(z)$, de $z$.



