\chapter{Apéndice: Conceptos Matemáticos y de la Física Clásica
Importantes en Mecánica Cuántica}

\section{Números complejos}
En nuestro estudio de la Mecánica Cuántica, muy a menudo nos
encontramos con números complejos, que incluyen el número imaginario
$\mathrm{i}=\sqrt{-1}$. En general, podemos escribir un número complejo
como la suma de su parte real y su parte imaginaria
\begin{equation}
    z= x+y\mathrm{i} 
    \label{eq:complex}
\end{equation}
donde
\begin{align}
    x=&\mathrm{Re}(z) \\
    y=&\mathrm{Im}(z) 
\end{align}
Las operaciones con números complejos son sencillas en el caso
de la suma o la resta, dado que los términos real e imaginario 
operan por separado. En el caso de la multiplicación, tenemos
que considerar cada binomio independientemente. Por ejemplo,
\begin{equation}
\begin{split}
    (2-\mathrm{i})(-3+2\mathrm{i}) = &
    -6 + 3\mathrm{i}+ 4\mathrm{i}- 2\mathrm{i}^2=\\
    &-4 + 7 \mathrm{i}
\end{split}
\end{equation}
Nótese que aquí estamos haciendo uso de $\mathrm{i}^2=-1$. Para
dividir números complejos, es conveniente definir el complejo
conjugado. Para el caso del número complejo $z$, definido en la
Ecuación \ref{eq:complex}, el complejo conjugado $z^\star$ es
\begin{equation}
    z^\star= x-y\mathrm{i} 
\end{equation}
donde hemos cambiado el signo de la parte imaginaria. Un número 
complejo multiplicado por su conjugado es siempre un número real
\begin{equation}
    zz^\star= (x+y\mathrm{i})(x-y\mathrm{i})= x^2+y^2
\end{equation}
Finalmente, para calcular el cociente de dos números imaginarios, 
por ejemplo
\begin{equation}
    z=\frac{2+\mathrm{i}}{1+2\mathrm{i}}
\end{equation}
podemos multiplicar y dividir por el complejo conjugado del
denominador
\begin{equation}
    z=\frac{2+\mathrm{i}}{1+2\mathrm{i}}\bigg(\frac{1-2\mathrm{i}}{1-2\mathrm{i}}\bigg)=\frac{4-3\mathrm{i}}{5}=\frac{4}{5}-\frac{3}{5}\mathrm{i}
\end{equation}

A menudo representamos los números complejos en un sistema de coordenadas
bidimensional, donde en el eje de abscisas representamos la parte 
real ($\mathrm{Re}(z$)) y en el eje de ordenadas representamos la 
parte imaginaria ($\mathrm{Im}(z$)). Si dibujamos el número complejo 
en este sistema de coordenadas, se ubicará en el punto $z=(x,y)$.
El vector que une el origen con este punto es el vector $\mathbf{r}$ 
cuyo módulo es $r=(x^2 + y^2)^{1/2}$. Este vector forma
un ángulo $\theta$.

\section{La Relación de Euler}
Un número complejo se puede expresar siempre en relación al módulo
$r$ y el ángulo $\theta$ usando la fórmula de Euler,  
a la que Feynmann llamó "la fórmula más sobresaliente de las
matemáticas"
\begin{equation}
    \mathrm{e}^{\mathrm{i}\theta} = \cos{\theta} + \mathrm{i}\sin{\theta}
\end{equation}
Para cualquier número imaginario podemos escribir
\begin{equation}
    z = x + y\mathrm{i} = r(\cos{\theta} + \mathrm{i}\sin{\theta})=
    r\mathrm{e}^{\mathrm{i}\theta}
\end{equation}
Si recordamos nuestra definición del producto entre un número 
complejo y su conjugado, vemos naturalmente cómo se cancela la
parte imaginaria
\begin{equation}
    zz^\star=r\mathrm{e}^{\mathrm{i}\theta}(r\mathrm{e}^{-\mathrm{i}\theta})= r^2
\end{equation}
A menudo encontraremos en nuestras funciones de onda, términos
exponenciales imaginarios, como $\mathrm{e}^{\mathrm{i}\theta}$.





