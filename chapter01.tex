\chapter{Antecedentes de la Mecánica Cuántica}
En el s. XIX, las leyes de la Física Clásica que describían 
el comportamiento de partículas y ondas parecía tener la madurez
suficiente como para que, en palabras atribuidas a Lord Kelvin,
``sólo restase la toma de medidas cada vez más 
precisas''\sidenote{``There is nothing new to be discovered in
physics now. All that remains is more and more precise measurement.''
En realidad, la cita es apócrifa.}. Sin embargo, 
es entonces cuando se producen una serie de cambios muy 
importantes  en la Física debido, en primer término, a
observaciones experimentales que la descripción 
clásica  no era capaz de explicar.

\section{Radiación del Cuerpo Negro}
La primera de estas observaciones está relacionada con un
fenómeno que podemos observar cotidianamente. Cuando calentamos un metal 
a muy elevadas temperaturas vemos que empieza a cambiar de 
color, adquiriendo tonalidades primero rojizas y luego azuladas.
Esto se debe a que  el cuerpo calentado emite una radiación
electromagnética. Y el cambio de coloración se debe a que 
en esta radiación, no todas las frecuencias 
están igualmente presentes. Si estudiamos cómo cambia la 
distribución de longitudes de onda, $\lambda$,  a medida que va 
aumentando la temperatura, veremos que el máximo de la distribución
se desplaza hacia longitudes de onda más bajas. Esta observación la
captura la \textbf{ley de desplazamiento de Wien},
\begin{equation}
\lambda_\mathrm{max}T=\mathrm{constante}
\label{eq:Wien}
\end{equation}
que fue el primer esfuerzo en sistematizar el comportamiento
de la radiación del denominado ``cuerpo negro''.
Definimos el \textbf{cuerpo negro} como un objeto 
idealizado que es capaz de absorber y emitir 
todas las longitudes de onda uniformemente\sidenote{
Kirchoff es quien define el cuerpo negro en 1859.}. 
Una buena
representación del cuerpo negro sería un dispositivo experimental 
con una cavidad cuyo interior es reflectante, que se prepara en
equilibrio a una temperatura constante $T$, y en el que hay apenas 
un pequeño orificio, que nos permite medir la radiación. 

Muchos físicos teóricos intentaron explicar esta observación
en el s. XIX. \textbf{Lord Rayleigh} trató este problema de manera
clásica, asumiendo que el campo electromagnético se podía describir
como una colección de osciladores con todas las frecuencias o
longitudes de onda posibles. Si había una radiación de una longitud
de onda determinada, esta radiación se correspondería con un oscilador
excitado. A partir de una descripción que asumía que la temperatura
distribuiría su energía de acuerdo con el \textbf{principio de
equipartición de la energía}, llegó con la ayuda de \textbf{James
Jeans} a la siguiente relación:
\begin{equation}
\rho=\frac{8\pi k_BT}{\lambda^4},
\label{eq:rayleigh-jeans}
\end{equation}
donde $\rho$ es la densidad de estados para una longitud de onda
$\lambda$, y $k_B$ es la constante de Boltzmann 
($k_B=R/N_A=1.381\times 10^{−23}$ J K$^{−1}$).
Esta ecuación, conocida como la \textbf{ecuación de Rayleigh-Jeans},
expresa la contribución a la energía de la radiación para cada una de las 
longitudes de onda a la densidad de estados. 
Como vemos la dependencia con  $\lambda$ es inversa con la
cuarta potencia.  A valores elevados de $\lambda$, la predicción
es acertada, pero a longitudes de onda bajas (correspondiente a
radiación de muy alta energía), la densidad aumentaría 
monotónicamente. Esto  conduciría a la denominada ``catástrofe 
ultravioleta'', porque de acuerdo con el aumento monotónico de
la densidad de estados, la cantidad de radiación de alta energía 
nunca dejaría de aumentar. 
Este resultado revelaba inequívocamente un fallo en la descripción
clásica del sistema.

En 1900, \textbf{Max Planck} resolvió este problema asumiendo
que la energía de cada oscilador estaba limitada a una serie 
de valores discretos, es decir, estaba cuantizada.
\sidenote{Planck recibió el Premio Nobel en 1918 
``como reconocimiento 
por sus servicios para el avance de la Física por
el descubrimiento de los cuantos de energía''.
Max Planck tardó en
aceptar las implicaciones de su descubrimiento, hasta decir ya en 
1911 que ``La hipótesis de los cuantos nunca se
desvanecerá del mundo" y ``No creo que esté yendo demasiado lejos si expreso la opinión de que con esta hipótesis se sientan las 
bases para la construcción de una teoría que algún día está destinada a impregnar los rápidos y delicados eventos del mundo molecular con una nueva luz''. }
En la aproximación de Raleigh y Jeans se asumía que las energías 
de los osciladores podía asumir cualquier valor, y eran por tanto
continuas. Planck, en cambio, introdujo la revolucionaria idea
de que las energías de los osciladores podían transmitirse en
forma de paquetes discretos, cuyas energías debían
ser proporcionales a múltiplos enteros de las frecuencias
$\nu$, es decir, 
\begin{equation}
E=nh\nu,
\end{equation}
donde $n=1,2,3...$. La expresión resultante
para la densidad de estados de acuerdo con esta descripción
es la denominada \textbf{distribución de Planck},
\begin{equation}
    \rho=\frac{8\pi h c}{\lambda^5(\mathrm{e}^{hc/\lambda k_BT}-1)}\
\label{eq:planck}
\end{equation}
Para longitudes de onda bajas, $\rho$ se hace cero debido al 
término exponencial del denominador, que tiende a $\infty$ más 
rápidamente que el decaimiento del término $\lambda^5$, 
evitando así la catástrofe ultravioleta.
En la teoría de Planck, la constante $h$ era indeterminada,
pero ajustando a los datos experimentales se pudo recuperar 
su valor de $h=6.626\times 10^{-34}$J$\cdot$s. 
\sidenote{A lo largo de este curso, haremos uso de esta 
\textbf{constante de Planck} y de su forma reducida,
$\hbar=h/2\pi=1.055\times 10^{-34}$  m$^2$kg/s.} 
De acuerdo con
esta nueva interpretación ``cuántica'', los osciladores se 
excitan sólo cuando hay suficiente energía disponible para 
alcanzar la energía $h\nu$.

\section{El Efecto Fotoeléctrico}
La segunda evidencia experimental que iba más allá de lo que la 
Física clásica era capaz de predecir es el denominado efecto
fotoeléctrico. 
En este caso, el experimento, realizado por el científico
alemán \textbf{Heinritz Hertz} entre 1886 y 1887 para
verificar la teoría de Maxwell, consistía en 
exponer a un material determinado a una radiación ultravioleta.
La predicción clásica era que al ir aumentando la intensidad
de la radiación, los electrones de la superficie del material
oscilarían más violentamente y acabarían por desprenderse
con una energía cinética dependiente de la intensidad.
Asimismo, la expectativa era que los electrones se liberasen
para cualquier valor de la frecuencia incidente.
Las observaciones experimentales contradijeron estas predicciones.
Fueron las siguientes:
\begin{enumerate}
    \item Hasta alcanzar una determinada frecuencia de radiación
    umbral, $\nu_0$, no se observaba nada. Solo a
    partir de una frecuencia umbral, $\nu_0$, característica
    de cada material, empezaban a liberarse electrones.
    \item La energía cinética de los electrones dependía de la frecuencia de la radiación incidente, y era independiente de
    la intensidad de esta radiación.
\end{enumerate}
En particular, este segundo punto era contrario a las predicciones de
Maxwell. Incluso con muy bajas intensidades de radiación
se producía la ionización del material con tal de que se cumpliese
la condici\'on $\nu>\nu_0$.
El efecto fotoeléctrico indica que la radiación
tiene comportamiento de partícula, y que sus partículas 
constituyentes, los fotones, son capaces de colisionar
con otras partículas como los electrones de un metal. 

¿De qué manera se resolvió este problema? Ya en el s. XX, 
\textbf{Albert Einstein} propuso una solución a esta paradoja
usando la hipótesis de Planck pero extendiéndola de manera
importante.
\sidenote{Einstein resolvió este problema en 1905, su 
``annus mirabilis'', cuando también describió el movimiento 
Browniano y la Teoría de la Relatividad Especial. Fue
el efecto fotoeléctrico por el que recibió
del Premio Nobel en 1921.}
Mientras que Planck creía que una vez se producía la emisión
de luz, ésta se comportaba como una onda clásica, Einstein
propuso que la misma radiación existía en pequeños
paquetes de energía $E=h\nu$, conocidos como 
\textbf{cuantos de energía} o \textbf{fotones}.
\sidenote{Quien los llamó así en 1926 fue G. N. Lewis, 
más conocido por su 
definición de ácidos y bases.}
Para liberar un electrón, Einstein teorizó, es
necesario un cuanto de energía con frecuencia
superior a un umbral 
$\nu_0$ característico de cada material. 
Usando un simple argumento de conservación de energía,
mostró que la energía excedente se liberaría en forma de energía 
cinética. Por tanto
podemos escribir la siguiente expresión
\begin{equation}
\frac{1}{2}m_ev^2=h\nu - \Phi \label{eq:work}
\end{equation}
En la lado izquierdo de la ecuación nos encontramos con la 
expresión familiar de la energía cinética. En el lado
derecho, $h\nu$ es la energía de la radiación incidente, y 
$\Phi$ es la llamada \textbf{``función trabajo''} o 
\textbf{``work function''}, es decir, el umbral necesario para 
producir la emisión del electrón. En el caso de que
$h\nu<\Phi$ no se produce liberación de electrones, pero si
por el contrario $h\nu>\Phi$ entonces sí que se produce
ionización y con una cantidad de energía que podemos calcular.

La ecuación \ref{eq:work} pudo utilizarse para determinar 
la constante de Planck, $j$. En su momento, este fue un resultado
sorprendente, dada la dificultad para aceptar la misteriosa
teoría cuántica. A partir de dos experimentos diferentes
era posible determinar con alta precisión una misma constante.

\section{Hipótesis de de Broglie}
La explicación del efecto fotoeléctrico estableció el carácter de
partícula de las ondas, en forma de fotones. 
Experimentos similares sirvieron para 
establecer el aun más revolucionario comportamiento como onda 
de la materia. El experimento más importante lo llevaron a cabo 
\textbf{Davisson} y \textbf{Germer}, 
que observaron la difracción de electrones en un cristal. 
La difracción es la interferencia causada por un objeto 
interpuesto en el camino de una onda, que puede ser constructiva
o destructiva. Al hacer este experimento se observó que también 
los electrones, comportándose como ondas, difractaban.
Otras partículas también son capaces de difractar, como han 
revelado después experimentos en partículas $\alpha$ o en
el hidrógeno molecular, confirmando que las partículas tienen
características de ondas. 

¿Pero cómo racionalizamos esta dualidad de ondas y cuerpos?
Una primera indicación nos la da la relación del físico Francés
\textbf{Louis de Broglie}, que propuso que una partícula que viajase
con un momento lineal $p$ tendría una longitud de onda asociada
\begin{equation}
\lambda=\frac{h}{p}\label{eq:debroglie}
\end{equation}
Esta expresión captura la \textbf{dualidad onda-partícula},
que se opone de manera frontal a la descripción de la Física 
Clásica.
Los cuerpos de gran tamaño tienen momentos lineales enormes, debido
a su gran masa. Por eso sus longitudes de onda son indetectables
y las propiedades oscilatorias de los cuerpos macroscópicos no 
pueden ser observados.	

\section{Principio de Incertidumbre de Heisenberg}
Otra de las grandes rupturas que establece la Mecánica Cuántica 
con la Mecánica Clásica es nuestra capacidad de medir una 
determinada propiedad con precisión arbitraria. Dentro de la 
formulación de la Mecánica Cuántica es muy importante el Principio
de Incertidumbre, enunciado por \textbf{Werner Heisenberg} en 1927
y que le valió un Premio Nobel en 1932 por "la creación de la Mecánica
Cuántica". 
Establece que es imposible determinar simultáneamente el momento 
y la posición de una partícula. 

Aunque veremos este principio en 
mayor profundidad más adelante en el curso, la precisión de una
medida viene determinada por
\begin{equation}
    \Delta x\Delta q\geq 1/2\hbar\label{eq:heiss}
\end{equation}
donde $\Delta p=(\langle p^2\rangle-\langle p \rangle^2)^{1/2}$ y
$\Delta q=(\langle q^2\rangle-\langle q\rangle^2)^{1/2}$. En realidad
el principio de incertidumbre no sólo afecta al momento y a la posición, sino a 
cualquier par de \textbf{variables complementarias}. Del mismo modo
que en la Ecuación \ref{eq:heiss} para posición y momento, para 
la duración de un proceso cuántico $t$ y su energía $E$ podemos
escribir
\begin{equation}
    \Delta E\Delta t\geq 1/2\hbar.
\end{equation}

\section{Espectros atómicos}
La última evidencia experimental que comentaremos procede de los 
espectros atómicos. Si  obtenemos un espectro de emisión
de un elemento como el hidrógeno, con lo que nos encontramos
es que el espectro es discontinuo. Por tanto, la emisión de un
átomo se produce sólo a unos determinados valores de la longitud
de onda. De paso podemos decir que en el caso de una molécula 
también el espectro de absorción, que corresponde a frecuencias
que permiten a una molécula cambiar de estado, es discontinuo.

Esto hizo que tuvieran que reformularse los modelos atómicos que 
se empleaban, y que parecían incompatibles con esta discontinuidad.
El primer modelo que tuvo en consideración esta propiedad fue
el de \textbf{Niels Bohr}, que partía de dos suposiciones: 
\begin{enumerate}
    \item La primera es que un electrón orbita alrededor del
    núcleo en un conjunto discreto de estados estacionarios, 
    correspondientes a órbitas circulares que satisfacen que la
    fuerza centrífuga y la fuerza de atracción entre el núcleo y el
    electrón son iguales
    \begin{equation}
        \frac{e^2}{4\pi \epsilon_0 r^2}=m_e\frac{v^2}{r}
    \end{equation}
    \item La segunda suposición era la que introducía la cuantización
    y asumía que las órbitas permitidas debían satisfacer la relación
    $n\hbar=m_evr$.
\end{enumerate}
Reorganizando estas ecuaciones podemos calcular el radio de las órbitas,
\begin{equation}
    r_n=\bigg(\frac{4\pi \epsilon_0\hbar^2}{m_ee^2}\bigg)n^2=a_0n^2    
\end{equation}
y la correspondiente energía de los diferentes niveles
\begin{equation}
    \begin{split}
%    A & = \frac{\pi r^2}{2} \\
% & = \frac{1}{2} \pi r^2
    E_n= & \frac{1}{2}m_ev^2 - \frac{1}{4\pi \epsilon_0}\frac{e^2}{r}=\\
    & -\frac{m_e}{2\hbar^2}\bigg(\frac{e^2}{4\pi\epsilon_0}\bigg)\bigg(\frac{1}{n^2}\bigg)
    \end{split}
\end{equation}
Para el hidrógeno, esta descripción es satisfactoria, y fue capaz de
explicar transiciones entre niveles de acuerdo con la siguiente 
expresión
\begin{equation}
    \tilde{\nu}=R_H\bigg(\frac{1}{n^2_1} -\frac{1}{n^2_2}\bigg) 
    \label{eq:Rydberg}
\end{equation}
donde $\tilde{\nu}=1/\lambda$ es el número de onda, $R_H$=109677 cm$^{-1}$ 
es la constante de Rydberg del átomo de hidrógeno. La Ecuación 
\ref{eq:Rydberg} explica las líneas espectrales de Lyman, Balmer, 
y Paschen, donde $n_1=1,2, 3$ y $n_2=2,3,4$, respectivamente.
