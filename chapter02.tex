\chapter{Postulados de la Mecánica Cuántica}
La Mecánica Cuántica está basada en una serie de
postulados. Los postulados son, según el diccionario
de la Real Academia de la Lengua, proposiciones cuya
verdad se admite sin pruebas y que sirven de base para
ulteriores razonamientos. A continuación introduciremos
los postulados de la Mecánica Cuántica, que usaremos 
como fundamento para describir más adelante sistemas
sencillos.

\section*{Estados y funciones de onda}
\begin{theorem}
Todas las propiedades observables de un sistema físico 
están contenidas en su función de onda, $\psi(q,t)$,
dependiente de las coordenadas ($q$) de las
partículas que componen el sistema y del tiempo ($t$). 
\end{theorem}

En Mecánica Cuántica, por tanto, el concepto de 
trayectoria es remplazado por la \textbf{función de
onda}, que contiene toda la información dinámica 
sobre el sistema.

\section{Interpretación de la función de onda}
Para desarrollar una intuición sobre el significado
de la función de onda es útil recurrir a la 
denominada \textbf{interpretación de 
Born}, que también enunciamos en la forma de 
postulado.
\begin{theorem} 
La probabilidad de que una partícula se encuentre en 
un volumen $\tau$ en el punto $r$ es proporcional
a $|\psi(r)|^2d\tau=\psi^\star \psi d\tau$
\end{theorem}
El cuadrado de la función de onda es por tanto una
\textbf{densidad de probabilidad}. Y si tomamos la
integral de esta densidad en una pequeña region 
$d\tau$ del espacio, obtendremos la probabilidad de
encontrar el sistema en esa región $d\tau$. En la 
medida en que la suma de todas las probabilidades
debe ser igual a la unidad, la función de onda debe 
cumplir una condición de normalización,
\begin{equation}
    N^2\int_{-\infty}^\infty\psi^\star\psi dx=1
    \label{eq:norm}
\end{equation}

En ocasiones, para el caso tridimensional, puede 
ser más conveniente escribir esta condición de 
normalización en coordenadas esféricas, es decir,
en función del radio $r$, la colatitud $\theta$
y el ángulo azimutal $\phi$. La interconversión
entre coordenadas cartesianas y esféricas es 
trivial, 
\begin{equation}
  \begin{array}{@{}ll@{}}
    x = & r\sin\theta\cos\phi \\
    y = & r\sin\theta\sin\phi \\
    z = & \cos \theta
  \end{array}
\end{equation}
de tal manera que la integral normalizada que 
expresábamos en una sóla dimensión en la 
Ecuación~\ref{eq:norm} se puede reescribir usando
$d\tau=r^2\sin\theta drd\theta d\phi$ y modificando
los límites de integración de la siguiente manera
\begin{equation}
    \int_\infty^\infty\psi^\star\psi d\tau=
    \int_0^\infty\int_0^\pi\int_0^{2\pi}\psi^\star\psi r^2\sin\theta drd\theta d\phi
\end{equation}

La interpretación de Born y la ecuación de Schrödinger
que, como veremos más adelante, es una ecuación
diferencial de segundo orden, introducen una serie de
requisitos para las funciones de onda aceptables:
\begin{itemize}
    \item Debe ser unívoca, es decir, no debe 
    estar definida con más de un valor en cada punto.
    \item Debe tener un cuadrado integrable.
    \item Debe ser continua.
    \item Debe ser derivable.
\end{itemize}


\section{Observables y operadores}
\begin{theorem}
Los observables se representan mediante operadores 
Hermíticos.
\end{theorem}
Para entender este postulado debemos definir primero
los operadores. Los operadores, $\hat{\Omega}$, no son
sino funciones que convierten una función de onda 
en otra:
\begin{equation}
    \psi(q,t)\xrightarrow{\hat{\Omega}}\psi'(q,t)
\end{equation}
Un ejemplo de operador, pueden ser el operador
diferencial con respecto de $x$, $\hat{D}_x=\frac{d}{dx}$.
Al aplicarlo sobre la función \textit{seno}, obtenemos
la función \textit{coseno}.
\begin{equation}
    \hat{D}_x\sin x=\frac{d}{dx}\sin x =\cos x
\end{equation}
Un \textbf{operador lineal} es un tipo especial de operador
que satisface las siguientes relaciones
\begin{equation}
      \begin{array}{rl}
        \hat{\Omega}(\psi+\phi) &=\hat{\Omega}\psi + \hat{\Omega}\phi\\
        \hat{\Omega}c\psi &=c\hat{\Omega}\psi
    \end{array}
\end{equation}

Un tipo de operador particular en el que estamos 
interesados en Mecánica Cuántica son los que se pueden
expresar en la forma de una \textbf{ecuación de valores
propios}:
\begin{equation}
    \hat{\Omega}\psi(q,t)=\omega\psi(q,t)
\end{equation}
donde $\omega$ es el valor propio (\textit{eigenvalue})
correspondiente al operador $\hat{\Omega}$. Cuando
encontramos esta propiedad, entonces decimos que $\psi$ 
es función propia ($\textit{eigenfunction}$) de este
operador. Este tipo de expresión es la que nos vamos a
encontrar en el caso del operador Hamiltoniano, 
$\hat{H}$, para la energía total pero también para 
otros observables que mostramos en la 
Tabla~\ref{tab:operators}.

\begin{table}[h]
  \begin{center}
    %\footnotesize%
    \small
    \begin{tabular}{lcr}
      \toprule
      Nombre & Expresión \\
      \midrule
      Posición & $\hat{x}=x$ \\
      Momento lineal & $\hat{p}_x=-i\hbar \frac{\partial}{\partial x}$ \\
      Energía cinética & $\hat{K}=\frac{\hat{p}_x^2}{2m}=
      -\frac{\hbar^2}{2m}\frac{\partial^2}{\partial x^2}$\\
      Energía potencial & $\hat{V}=V(x,y,z)$ \\
      Hamiltoniano & $\hat{H}=\hat{K}+\hat{V}$ \\
      \bottomrule
    \end{tabular}
  \end{center}
  \caption{Operadores habituales en Mecánica Cuántica.}
  \label{tab:operators}
\end{table}

Finalmente, un \textbf{operador hermítico} es el que
satisface la siguiente relación
\begin{equation}
    \int\psi_i^\star\hat{\Omega}\psi_jdx=
    \bigg\{\int\psi_j^\star\hat{\Omega}\psi_idx\bigg\}^\star
\end{equation}
Estos operadores tienen dos propiedades especiales:
\begin{itemize}
    \item Sus valores propios son reales.
    \item Sus funciones propias son ortogonales, es decir:
    $\int_{-\infty}^\infty\psi_1^\star\psi_2d\tau=0$
\end{itemize}
El conjunto de funciones propias de un operador 
    hermítico forman un \textbf{conjunto completo}.

\section{El resultado de una medida}
\begin{theorem}
Cuando un sistema está descrito por una función de onda
$\psi$, el valor del observable $\Omega$ en una serie 
de medidas es igual al valor promedio (“expectation 
value”) del operador.
\end{theorem}
Cuando nos encontramos con que una función de onda no es
función propia de un operador $\hat{\Omega}$ no podemos
utilizar el método de los operadores para obtener el valor
promedio de $\Omega$. Sin embargo podemos expresar la
función de onda como una combinación lineal o
\textbf{superposición} de un conjunto de funciones
propias
\begin{equation}
\psi =  c_1\psi_1 + c_2\psi_2 + ... = \sum_kc_k\psi_k    
\end{equation}
En esta expresión los valores de $c_k$ son coeficientes
reales o imaginarios y las funciones de onda $\psi_k$
corresponden con diferentes estados del sistema. Esto
significa que cualquier función de onda puede 
expresarse como una combinación lineal de estos estados.

Cuando se realiza una medida, se obtendrá el valor 
propio $\omega_k$ correspondiente a una de las funciones
propias $\psi_k$ del operador. La probabilidad de
encontrar cada una de las soluciones es proporcional
al cuadrado del módulo  del coeficiente $|c_k|^2$ dentro
de la combinación lineal. El promedio tras un gran número
de observaciones lo da el valor esperado, $\langle
\Omega\rangle$.



\section{La ecuación de Schrödinger}
\begin{theorem}
La función de onda evoluciona en el tiempo de acuerdo 
con la ecuación de Schrödinger dependiente del tiempo:
\begin{equation}
    \hat{H}\Psi = i\hbar\frac{\partial \Psi}{\partial t}
\end{equation}
\end{theorem}
Como conocemos el operador hamiltoniano, $\hat{H}$ (ver
Tabla \ref{tab:operators}), podemos escribir si trabajamos
en una sola dimensión
\begin{equation}
    \hat{H}\Psi = 
      -\frac{\hbar^2}{2m}\frac{\partial^2}{\partial x^2}\Psi + V(x)\Psi = i\hbar\frac{\partial \Psi}{\partial t}
\end{equation}
En el caso de los estados estacionarios 
se puede separar la parte de la función de onda
que depende de las coordenadas de la parte que 
depende del tiempo,
\begin{equation}
    \Psi(q,t)=\psi(q)\tau(t)
\end{equation}
Al aplicar el operador hamiltoniano 
sobre este tipo de función de onda
aquel afecta únicamente a la parte
que depende de las coordenadas. Por
tanto,
\begin{equation}
    \hat{H}\Psi(q,t)=
    \tau(t)\hat{H}\psi(q)=
    i\hbar\frac{\partial\psi(q)\tau(t)}{\partial t}=
    i\hbar\psi(q)\frac{\partial \tau(t)}{\partial t}
\end{equation}
Reorganizando los términos en esta ecuación encontramos
lo siguiente
\begin{equation}
    \frac{1}{\psi(q)}\hat{H}\psi(q) = \frac{i\hbar}{\tau(t)}\frac{\partial \tau(t)}{\partial t}= E
    \label{eq:schro-2}
\end{equation}
A partir de los dos términos a la derecha de la Ecuación
\ref{eq:schro-2} podemos encontrar la solución para la
Ecuación de Schrödinger,
\begin{equation}
    \tau(t)=A\exp(-iEt/\hbar)
\end{equation}
Por tanto la función de onda para un estado estacionario
es
\begin{equation}
    \Psi(q,t)=\psi(q)\exp(-iEt/\hbar)
\end{equation}
Si obtenemos $\Psi(q,t)^2$ veremos que las funciones para un
estado estacionario tienen una energía definida 
y una densidad de probabilidad constante en el tiempo.
En un estado estacionario podemos medir todos y cada uno de los 
operadores que conmuten con el hamiltoniano. Estos son el 
equivalente de las constantes de movimiento de la mecánica clásica.

\section{El principio de incertidumbre}
Ahora que entendemos más a fondo el funcionamiento de los
operadores podemos adentrarnos de nuevo en el
\textbf{Principio de Incertidumbre} enunciado por 
\textbf{Werner Heisenberg}. Como comentamos en el Tema 1
el principio establece en su formulación más básica que 
no se puede determinar con precisión  arbitraria el momento
y la posición de la partícula. 

Supongamos que tenemos una partícula perfectamente 
localizada en un punto del espacio. Utilizando el
principio de superposición, podemos obtener esta 
descripción combinando un gran número de ondas 
armónicas, es decir, generando un paquete de
ondas (\textit{wave packet}). Esto resultará en una
función de onda perfectamente localizada y por tanto
con un error muy pequeño en la posición ($\Delta x=0$).
Pero a su vez las distintas contribuciones a la función 
de onda tendrán muy diversos valores del momento lineal, 
lo que resulta en una gran incertidumbre en el valor 
del momento lineal ($\Delta p=\infty$). Si tenemos gran
certeza en uno de estos valores, entonces
tendremos gran incertidumbre en el otro valor.

El principio de incertidumbre, formulado típicamente para
$p$ y $q$, es en realidad más general. Si para cada 
operador $\hat{\Omega}$ podemos definir un error $\Delta\Omega$:
\begin{equation}
    (\Delta\Omega^2)=\sigma^2_\Omega=\langle (\hat{\Omega} - \langle \Omega\rangle)^2 \rangle=
    \langle\hat{\Omega}^2\rangle - \langle\Omega\rangle^2
\end{equation}
el principio de incertidumbre relaciona los errores de dos
\textbf{variables complementarias} de tal modo que
\begin{equation}
\Delta\Omega_1\Delta\Omega_2\geq
1/2|\langle[\Omega_1,\Omega_2]\rangle|
\end{equation}
donde  usamos el conmutador  $[\hat{\Omega}_1,\hat{\Omega}_2] =
\hat{\Omega}_1\hat{\Omega}_2 -\hat{\Omega}_2\hat{\Omega}_1$.
Son complementarias aquellas variables que no conmutan y para
las que se cumple 
$\hat{\Omega}_1(\hat{\Omega}_2\psi)\neq\hat{\Omega}_2(\hat{\Omega}_1\psi)$.
En el caso particular de la posición y el momento obtenemos que
\begin{equation}
    [\hat{x},\hat{p}_x] = i\hbar
\end{equation}
y por tanto son operadores
complementarios. Por el contrario, cuando
dos operadores hermíticos conmutan decimos 
que son \textbf{compatibles }, y comparten un conjunto
completo de funciones propias, $\psi_n$.

\section{Principio de exclusión o de Pauli}
Por último introduciremos una contribución decisiva
al desarrollo de la Mecánica Cuántica. El principio de
exclusión establece que las partículas cuánticas poseen una 
propiedad fundamental que toma un valor entero o semientero 
característico de cada partícula. Esta propiedad, 
llamada \textit{spin}, $s$, fue inicialmente introducida
por \textbf{Wolfgang Pauli} como lo que él denominaba una
dualidad no
describible clásicamente (``classically non-describable
two-valuedness'').

El spin permite clasificar las partículas en dos tipos.
Las partículas de espín semientero, como los electrones,
protones o neutrones, 
se denominan fermiones, y las de espín entero, como los
fotones, se llaman bosones. La función de onda de un 
colectivo de partículas idénticas debe ser simétrica 
(si se trata de bosones) o antisimétrica (fermiones) 
frente al intercambio de dos cualesquiera de las partículas.
\begin{equation}
    \Psi(q_1, ..., q_i, ..., q_j, ..., q_N)=\pm 
    \Psi(q_1, ..., q_j, ..., q_i, ..., q_N)
\end{equation}

El comportamiento de los sistemas de fermiones es muy 
distinto del de los sistemas de bosones. Esto se aprecia
al examinar sistemas de partículas independientes, en los
que cada partícula tiene su propia función de onda,
$\psi_i(qi)$, y la función colectiva es producto de las
individuales. La antisimetría exige que no haya dos 
fermiones simultáneamente en el mismo estado de una 
partícula, mientras que, potencialmente, todos los bosones
podrán coexistir en un estado idéntico.
Son fermiones las partículas que constituyen la materia
ordinaria: electrones, protones, neutrones, etc. Son 
bosones las partículas que actúan como intermediarias 
en las interacciones.




