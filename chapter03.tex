\chapter{Estudios Mecanocuánticos de Sistemas Modelo}
Después de haber introducido los antecedentes históricos y los
postulados de la Mecánica Cuántica, vamos a proceder a 
estudiar una serie de sistemas modelo: la partícula libre, la
partícula en una caja de potencial y el oscilador armónico.
Estos sistemas, que están sometidos a potenciales conservativos,
$V(x,y,z)$, tienen soluciones analíticas para la ecuación 
de Schrödinger independiente del tiempo y por ello serán de gran
utilidad en el estudio de átomos y moléculas. Un comportamiento
que veremos al estudiar los sistemas modelo es el que expresa
el \textbf{principio de correspondencia de Böhr}:
las predicciones de la Mecánica Cuántica deben converger con 
las de la Mecánica Clásica siempre que de una forma continua vamos 
desde el campo microscópico al macroscópico o bien en el límite   
de números cuánticos elevados.

\section{Partícula libre}
El sistema más sencillo que vamos a estudiar es la partícula libre,
que no está sujeta a ninguna restricción dentro de su movimiento
en una sola dimensión. En su descripción clásica la partícula libre
experimenta un potencial constante que podemos asumir que es igual a
$V(x)=0$. Desde un punto de vista clásico, podemos decir, por tanto,
que la energía total es igual a la energía cinética,
\begin{equation}
    E=K=1/2mv^2=p_x^2/2m
\end{equation}
Si despejamos el momento lineal, obtenemos la expresión para el
momento lineal, $p_x=\pm(2mE)^{1/2}$. De acuerdo con este resultado
tanto la energía como el momento pueden tomar cualquier valor.

En el caso de la descripción mecanocuántica partimos de 
la ecuación de Schödinger independiente del tiempo
\begin{equation}
    \hat{H}\psi=E\psi,
\end{equation}
donde en este caso $\hat{H}=-\frac{\hbar^2}{2m}\frac{d^2}{dx^2}$ al
no estar sometida la partícula a ningún potencial. La solución
general para esta ecuación es
\begin{equation}
    \psi = A\mathrm{e}^{ikx}+ B\mathrm{e}^{-ikx}\label{eq:freeparticle}
\end{equation}
En la Ecuación \ref{eq:freeparticle}, encontramos un término
exponencial positivo ($\mathrm{e}^{ikx}$) y otro negativo 
($\mathrm{e}^{-ikx}$). Para cada uno de ellos la densidad 
$|\psi|^2$ es independiente de la posición $x$.
A través de la \textbf{relación de Euler}, 
$\mathrm{e}^{\pm ikx}=\cos(kx)\pm\mathrm{i}\sin(kx)$, podemos reescribir
la Ecuación \ref{eq:freeparticle} como
\begin{equation}
    \psi=C\cos(kx) + D\sin(kx) \label{eq:freepangular}
\end{equation}
En ambos casos, los coeficientes
$A$, $B$, $C$ y $D$ deben determinarse considerando las condiciones
de contorno.

Si aplicamos el operador hamiltoniano sobre la ecuación 
\ref{eq:freeparticle} y dado que la energía total es igual
a la energía cinética, obtenemos que 
\begin{equation}
E=k^2\hbar^2/2m
\end{equation}
o bien $k=\sqrt{2mE}/\hbar$. La constante $k$ puede adoptar
cualquier valor y por tanto la energía no está cuantizada,
igual que sucedía en la descripción clásica. 

Asímismo podemos aplicar el operador momento lineal, 
$\hat{p}_x$ sobre la función de onda, que no es propia
de este operador. Sin embargo, si lo aplicamos 
independientemente sobre cada uno de los términos
($A\mathrm{e}^{\mathrm{i}kx}$ y $B\mathrm{e}^{-\mathrm{i}kx}$)
obtenemos
\begin{equation}
    p_x = \pm k\hbar
\end{equation}
Las soluciones correspondientes positiva y negativa 
se corresponden con la partícula desplazándose
hacia la izquierda o hacia la derecha en el eje $x$, 
respectivamente. Esto aclara el significado de los 
términos $A$ y $B$, que determinan cómo se ha preparado 
el sistema. En el caso de que la partícula se prepare 
viajando hacia valores positivos de $x$, entonces $B=0$; 
mientras que si viaja hacia valores negativos, $A=0$. 
En el caso en que
$D=0$, la situación se complica dado que queda la función
$\psi=C\cos(kx)$, que no es función propia del operador
momento lineal. Sin embargo podemos expresar la función como
$\psi=1/2C(\mathrm{e}^{ikx}+\mathrm{e}^{-ikx})$,
es decir, como una superposición de estados. Al medir el
momento la mitad de las veces obtendríamos que la partícula
viaja en sentido positivo y la otra mitad en sentido
negativo, y que el \textit{expectation value} sería cero.

También es interesante conectar este resultado 
con la relación de De Broglie (Ecuación
\ref{eq:debroglie}). Típicamente una ecuación de
onda para una longitud de onda $\lambda$ se
escribe como $\cos(2\pi x/\lambda)$ o como
$\sin(2\pi x/\lambda)$. Por tanto la longitud de
onda en la Ecuación \ref{eq:freepangular} es
$\lambda=2\pi/k$. Esto confirma la relación de 
De Broglie que relaciona momento y longitud de
onda $p=2\pi/\lambda\times\hbar=h/\lambda$.

\section{Partícula en una caja}
\subsection{En una sola dimensión}
El segundo sistema con el que nos enfrentamos es el de una partícula
sometida a un potencial infinito ($V(x)=\infty$) en $x=0$ y $x=L$, y 
constante (o cero) en la región del espacio comprendida entre estos
dos valores. Si en primer lugar intentamos conocer la función de onda
para $x\leq 0$ y $x\geq L$ podemos escribir la ecuación de Schrödinger
usando
\begin{subequations}
    \begin{align}
         \hat{H}\psi &=E\psi  \\
         -\frac{\hbar}{2m}\frac{\partial^2\psi}{\partial x^2} + \infty\psi &=E\psi
    \end{align}    
\end{subequations}
Si agrupamos términos llegamos a la siguiente expresión,
\begin{equation}
\frac{1}{\infty}\frac{\partial^2\psi}{\partial x^2} = \psi 
\end{equation}
Por tanto la función de onda que satisface esta ecuación es
$\psi=0$ y la densidad de probabilidad tanto a la izquierda 
de $x=0$ como a la derecha de $x=L$ es cero.

Más interesante es el tipo de función de onda que podemos
escribir para la región donde $V(x)=0$, que de nuevo es de
la forma
\begin{equation}
    \psi=A\mathrm{e}^{ikx} + B\mathrm{e}^{-ikx}
\end{equation}
Como hemos visto con anterioridad esta ecuación puede 
escribirse como 
\begin{equation}
    %\psi=C\cos(kx) + D\mathrm{i}\sin(kx)
    \psi=C\cos(kx) + D\sin(kx)\label{eq:box1d_trig}
\end{equation}
En este caso es necesario aplicar las condiciones de 
contorno que se derivan del potencial en los límites de
la caja (para $x=0$ o $x=L$). En estos límites sabemos que 
la función de onda es cero, $\psi(0)=0$ y $\psi(L)=0$, y por
tanto debemos restringir los valores de la Ecuación 
\ref{eq:box1d_trig} para garantizar que la función sea 
continua. Para que $\psi(0)=0$, el término correspondiente
al coseno debe ser cero. Por tanto, $C=0$, lo que nos deja
únicamente con  $\psi=D\sin(kx)$. Para que esta función
satisfaga la otra condición de contorno, 
$\psi(L)=D\sin(kL)=0$, es necesario que se cumpla $kL=n\pi$,
donde $n$ debe ser un número entero ($n=1,2,3,...)$. Es
muy importante resaltar que la \textbf{cuantización} 
(en este caso a través del número cuántico $n$) se deriva de
aplicar condiciones de contorno a nuestro sistema.

Otra condición que imponemos a nuestra función de onda es la
condición de normalización 
$\int_0^L\psi^\star \psi dx = 1$, 
que en este caso particular es
\begin{equation}
        D^2\int_0^L\sin^2 (n\pi x/L) dx = D^2L/2= 1
\end{equation}
Por tanto la constante que normaliza la ecuación es
$D=(2/L)^{1/2}$. Sustituyendo, la función de onda
resultante es
\begin{equation}
    \psi_n=\bigg(\frac{2}{L}\bigg)^{1/2}\sin(n\pi x/L)
    \label{eq:box1d}
\end{equation}
y su densidad de probabilidad se puede escribir como
\begin{equation}
    \psi^2_n(x)=\frac{2}{L}\sin^2\frac{n\pi x}{L}
\end{equation}
En el límite clásico en el que $n$ es muy elevado, nos 
encontramos con que hay tantos periodos de oscilación 
entre 0 y $L$ que es igualmente probable encontrar la 
partícula en todas las posiciones permitidas. Este 
resultado está de acuerdo con el principio de 
correspondencia.

A partir de la función de onda de la Ecuación \ref{eq:box1d}
podemos obtener la energía de la partícula en una caja. Para
ello sólo tenemos que aplicar el operador hamiltoniano, 
$\hat{H}$, en la región $0< x<L$. La aparición de $n$ en
la función de onda resulta en la cuantización de los 
niveles energéticos. Así,
\begin{equation}
    \begin{array}{lr}
        E_n =\frac{n^2h^2}{8mL^2},& \mathrm{ donde~}n=1,2,3...
    \end{array}
\end{equation}
La energía residual para $n=1$ es $E=h^2/8mL^2$. 
Nótese que la variación de la energía entre niveles
contiguos, $\Delta E=(2n+1)h/8mL^2$ aumenta al aumentar el
número cuántico $n$ y que cuando la masa del sistema es muy
elevada, llegamos a un continuo de energía, que se
corresponde con el límite clásico.

También podemos calcular el momento lineal. Sin embargo la 
función de onda (Ecuación ~\ref{eq:box1d}) no es función
propia del operador momento lineal. Para resolver este
problema, expresamos la función de onda como superposición
de funciones exponenciales $\mathrm{e}^{\pm in\pi x/L}$, 
que sí son funciones propias del operador $\hat{p}_x$
\begin{equation}
    \psi_n=\bigg(\frac{2}{L}\bigg)^{1/2}\sin(n\pi x/L)=
    \frac{1}{2i}\bigg(\frac{2}{L}\bigg)^{1/2}(\mathrm{e}^{\mathrm{i}n\pi x/L} - \mathrm{e}^{-\mathrm{i}n\pi x/L})
\end{equation}
Los dos valores propios del operador, $p=k\hbar$ y 
$p=-k\hbar$, donde $k=n\pi/L$, corresponden a la versión mecanocuántica de 
la partícula que en una caja de potencial se chocaría
alternativamente con una pared y con la otra.

\subsection{En tres dimensiones}
En el caso de que tengamos una partícula confinada
en más de una sola dimensión, el operador 
hamiltoniano tendrá que tener en cuenta los 
grados de libertad adicionales
\begin{equation}
    -\frac{\hbar}{2m}\bigg(\frac{\partial^2}{\partial x^2} + \frac{\partial^2}{\partial y^2}\bigg)\Psi= E\Psi
\end{equation}
Podemos separar la función de onda en sus diferentes
componentes, de modo que $\Psi=\mathrm{X}(x)\mathrm{Y}(y)$. Así, la 
energía tendrá componentes $E_x$ y $E_y$ y la
energía total será la suma de las energías.
Para cada componente podemos escribir una
solución igual que en el caso de la caja
en una sola dimensión
\begin{subequations}
    \begin{align}
        \mathrm{X}_{n_1}(x)=\bigg(\frac{2}{L_1}\bigg)^{1/2}\sin(n_1\pi x/L_1) \\
        \mathrm{Y}_{n_2}(y)=\bigg(\frac{2}{L_2}\bigg)^{1/2}\sin(n_2\pi y/L_2)
    \end{align}
\end{subequations}
y por tanto la función de onda global se podrá
escribir como
\begin{equation}
        \Psi_{n_1, n_2}(x,y)=\bigg(\frac{2}{(L_1L_2)^{1/2}}\bigg)\sin(n_1\pi x/L_1)\sin(n_2\pi y/L_2)
\end{equation}
que tiene una energía total igual a
\begin{equation}
    E_{n_1,n_2} =\bigg(\frac{n_1^2}{L_1^2} + \frac{n_2^2}{L_2^2}\bigg)\frac{h^2}{8m}
\end{equation}

En el caso de que las dimensiones de la caja
sean iguales, $L_1=L_2$, las expresión resultante
para la función de onda es
\begin{equation}
    \Psi_{n_1, n_2}(x,y)=\bigg(\frac{2}{L}\bigg)\sin(n_1\pi x/L)\sin(n_2\pi y/L)
\end{equation}
mientras que para la energía obtenemos
\begin{equation}
    E_{n_1,n_2}=\bigg(\frac{n_1^2+n_2^2}{L^2}\bigg)\frac{h^2}{8m}
\end{equation}
Esto resulta en un gran número de estados degenerados, 
es decir, de estados que tienen la misma energía,
dado que $E_{n_1,n_2}=E_{n_2,n_1}$.

\subsection{Efecto túnel}
En el caso de que las barreras de potencial tengan un 
valor finito la situación es diferente a la que nos
encontramos en la partícula en una caja. En estos casos,
se puede producir penetración del sistema en la barrera 
de potencial. Esto es así incluso en el caso de que la
energía de que dispone la partícula sea inferior a la
barrera de potencial, $E<V$. De esta manera se viola la
descripción clásica del sistema en que sería imposible
que la partícula cruzase esa barrera si se ha preparado
con una energía menor que la necesaria.

Supongamos que el sistema que queremos describir es una 
partícula que se acerca a una barrera de potencial por la 
izquierda de la misma. En la región a la izquierda de la 
barrera (región I) podemos usar la función de partición de 
la partícula libre, de tal modo que
\begin{equation}
    \psi_I=A\mathrm{e}^{\mathrm{i}kx} +
    B\mathrm{e}^{-\mathrm{i}kx}
    \label{eq:tunnelfree}
\end{equation}
donde $k\hbar=(2mE)^{1/2}$. En la zona de la barrera
(región II), por el contrario, la ecuación de Schrödinger 
ha de  incorporar el potencial $V$. En este caso la solución
general es
\begin{equation}
    \psi_{II}=C\mathrm{e}^{\kappa x} + D\mathrm{e}^{-\kappa x}
\label{eq:tunnel}
\end{equation}
donde $\kappa\hbar=\{2m(V-E)\}^{1/2}$. Como estamos
considerando el límite en el que $E<V$, entonces nos encontramos que ambos términos de la Ecuación 
\ref{eq:tunnel} son reales, $\kappa\hbar\in \mathbb{R}$ y
que la función de onda para la región II no es oscilatoria,
y corresponde por el contrario a un decaimiento exponencial
que ocurre dentro de la barrera.
Finalmente, al lado derecho de la barrera, $x>L$, podemos escribir de nuevo la ecuación 
correspondiente a la partícula libre:
\begin{equation}
    \psi_{III}=A'\mathrm{e}^{\mathrm{i}kx} + B'\mathrm{e}^{-\mathrm{i}kx}
\end{equation}
donde de nuevo $k\hbar=(2mE)^{1/2}$.

Como vimos en la presentación de las funciones de onda, 
una condición que deben cumplir para que sean aceptables 
es que sean continuas. Esto tiene importancia tanto en $x=0$
como en $x=L$. A partir de las expresiones de las Ecuaciones
\ref{eq:tunnelfree} y \ref{eq:tunnel} podemos escribir:
\begin{subequations}
    \begin{align}
         A + B =&C + D \\
         C\mathrm{e}^{\kappa L} + D\mathrm{e}^{-\kappa L}=&
         A'\mathrm{e}^{\mathrm{i}kL} + B'\mathrm{e}^{-\mathrm{i}kL}
    \end{align}
\end{subequations}
Por otra parte las funciones de onda deben cumplir la
condición de derivabilidad, tanto a la izquierda como a
la derecha de la barrera. Por tanto,
\begin{subequations}
    \begin{align}
        A\mathrm{i}k-B\mathrm{i}k =& C\kappa + D\kappa \\
        C\kappa\mathrm{e}^{\kappa L} - D\kappa\mathrm{e}^{-\kappa L} =& A'\mathrm{i}k\mathrm{e}^{\mathrm{i}kL}-B'\mathrm{i}k\mathrm{e}^{\mathrm{i}kL}
    \end{align}
\end{subequations}

De nuevo podemos intentar interpretar estos coeficientes 
asumiendo que el sistema se prepara de una determinada 
manera. Partamos de la situación en que una partícula se
acerca hacia la barrera en le región I. Este desplazamiento
está descrito por una función
de onda correspondiente al término $\mathrm{e}^{\mathrm{i}kx}$ 
y por tanto tiene una probabilidad $|A|^2$. Esta partícula
puede ser reflejada por la barrera y desplazarse hacia la
izquierda (término $\mathrm{e}^{-\mathrm{i}kx}$) con una 
probabilidad $|B|^2$. Por tanto la probabilidad de que se
refleje se podrá escribir como $R=|B^2|/|A^2|$. Asimismo,
la barrera puede ser atravesada por una componente no 
oscilatoria, cruzar la región II y llegar hasta el otro lado 
de la barrera con probabilidad $|A'|^2$. El término $|B'|^2=0$
corresponde a la partícula que se aproxima a la barrera por la 
derecha en la región III.

A partir de la probabilidad de llegada de la onda incidente
y de la onda transmitida podemos calcular
el \textbf{coeficiente de transmisión}, 
    $T=|A'|^2/|A|^2$
que en el caso de tener 
barreras elevadas y anchas, se puede simplificar como
\begin{equation}
    T\simeq 16\varepsilon(1-\varepsilon)\mathrm{e}^{-2\kappa L},
\end{equation}
donde $\varepsilon=E/V$. A partir de esta expresión podemos
entender claramente que el efecto túnel será más importante 
para barreras estrechas, donde $L$ es pequeño. Asímismo,
como depende exponencialmente y con signo negativo con 
$\kappa\hbar=\{2m(V-E)\}^{1/2}$, este efecto también es
más importante para partículas pequeñas que para partículas
grandes.

\section{Oscilador armónico}
El último sistema modelo que vamos a tratar en este Tema es
el oscilador armónico. Desde un punto de vista clásico, 
definimos el movimiento armónico cuando una partícula,
de acuerdo con la \textbf{Ley de Hooke}, experimenta una fuerza $F$ proporcional a su desplazamiento $x$ con 
respecto a la posición de equilibrio
\begin{equation}
    F=-kx
\end{equation}
donde $k$ es la constante de fuerza. La energía potencial 
a la que corresponde esta fuerza es
\begin{equation}
    V= 1/2kx^2
\end{equation}
En la descripción cuántica del oscilador armónico, podemos 
incorporar esta misma expresión para la energía potencial
en la ecuación de Schrödinger independiente del tiempo
\begin{equation}
    -\frac{\hbar^2}{2m}\frac{\partial^2\psi}{\partial x^2}
    + \frac{1}{2}kx^2\psi=E\psi \label{eq:schrodinger_ho}
\end{equation}
Las soluciones permitidas para esta ecuación deben cumplir
la condición de que $\psi=0$ en el límite de $x=\pm\infty$.
Los niveles energéticos permitidos corresponden a 
\begin{equation}
    E_v=(v+1/2)\hbar\omega
\end{equation}
donde estamos introduciendo el número cuántico $v$ y 
definiendo la frecuencia $\omega=(k/m)^{1/2}$. Al contrario
de lo que sucede para la partícula en la caja, en este 
caso $v$ sí puede ser cero, con lo que la energía del punto
cero en el oscilador armónico es $E_0=1/2\hbar\omega$. Al
contrario de lo que sucedía en la partícula en una caja, 
en este caso la separación entre niveles energéticos 
contiguos es constante, $E_{v+1}-E_{v}=\hbar\omega$.

La solución para la Ecuación \ref{eq:schrodinger_ho} 
tiene la forma
\begin{equation}
    \psi_v(y)=N_vH_v(y)\mathrm{e}^{-y^2/2}
\end{equation}
donde hemos hecho un cambio de variable, $y=x/\alpha$, 
con $\alpha=\big(\hbar/mk\big)^{1/4}$; $N_v$ es
la constante de normalización, $H_v$ es un polinomio de
Hermite y $\mathrm{e}^{-y^2/2}$ es una Gaussiana. 
Los polinomios de Hermite  son un tipo particular de
polinomio ortogonal y tienen la forma funcional que
mostramos en la Tabla~\ref{tb:hermite}. Así, podemos
escribir las funciones de onda para los números
cuánticos $v=0$ y $v=1$ 
\begin{subequations}
    \begin{align}
        \psi_0(x)&=N_0\mathrm{e}^{-y^2/2}=N_0\mathrm{e}^{-x^2/2\alpha^2} \\
        \psi_1(x)&=N_12\times y\mathrm{e}^{-y^2/2}
    \end{align}
\end{subequations}
Por tanto, para $v=0$ la forma funcional es sencillamente
una gaussiana centrada en $y=0$, mientras que para $v=1$ 
la función $\psi_1$ es negativa para $y<0$ y positiva para
$y>0$. Esta alternancia  entre funciones simétricas y
antisimétricas respecto a $y$ continua al ir aumentando el
número cuántico. El número de nodos en la densidad viene
determinado por $v$.

\begin{table}[t!]
    \centering
    \begin{tabular}{c|c}
     $v$ & $H_v(y)$ \\
     \hline
    0 & 1\\ 
    1 & $2y$\\ 
    2 & $4y^2-2$\\ 
    3 & $8y^3-12y$\\ 
    4 & $16y^4 - 48y^2 + 12$\\
    ... & 
    \end{tabular}
    \caption{Polinomios de Hermite $H_v(y)$ para diferentes 
    valores del número cuántico $y$.}
    \label{tb:hermite}
\end{table}

Un aspecto a destacar en las funciones de onda es que la
Gaussiana tiende a cero rápidamente a desplazamientos altos.
El término  $y^2=x^2(mk/\hbar)^{1/2}$ de la Gaussiana hace
que el exponencial decaiga más rápidamente cuanto mayores
sean la masa y más intensa sea la constante de fuerza del
potencial. Al ir aumentando el número cuántico $v$, la
densidad se vaya esparciendo cada vez más hacia los
laterales. %De la misma manera que en el caso del 
%pozo de potencial con un valor finito de $V$, en el caso del
%oscilador armónico se puede producir efecto túnel. 
%De nuevo, esta posibilidad es  contraria al resultado de
%la mecánica clásica, donde la partícula se vería limitada a
%la región correspondiente a $|x|\leq(2E/k)^{1/2}$. 
La convergencia con la descripción clásica se produce para
valores elevados de $v$, donde la densidad es máxima en las
regiones en que la energía cinética es cero.