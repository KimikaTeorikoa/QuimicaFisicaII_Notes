\chapter{Átomos Hidrogenoides}
Una vez hemos descrito una serie de sistemas modelo, podemos pasar a aplicar
sus soluciones en el estudio de sistemas relevantes para la Química.
En primer lugar, estudiaremos átomos hidrogenoides, es decir, átomos con tan
sólo un electrón (H, He$^+$, Li$^{2+}$, O$^{7+}$...). Son sistemas importantes
en la medida en que para ellos podemos obtener una solución exacta para la 
ecuación de Schrödinger.

En este Tema, a menudo nos resultará de utilidad expresar las 
magnitudes en unidades atómicas, es decir, normalizando la carga, 
la masa, la longitud, el momento angular o la energía, por los valores 
que mostramos en la Tabla \ref{tb:au}.

\begin{table}[b!]
    \centering
    \small{}
    \begin{tabular}{|l|l|c|l|}
    \hline
         Propiedad & Unidad & Equivalencia SI& Nombre  \\
    \hline
    \hline
         Carga elemental &$e$ & 1.602e$^{-19}$ C & \\ 
         Masa & $m_e$ & 9.109e$^{-31}$ kg & \\ 
         Momento angular & $\hbar$ & 1.055e$^{-34}$ J$\cdot$s & \\ 
         Constante eléctrica & $4\pi \varepsilon_0$ & 1.113e$^{-10}$ C^2$\cdot$J$^{-1}\cdot$m$^{-1}$ &\\
         Longitud & $a_0=4\pi \varepsilon_0\hbar^2/m_ee^2$ & 5.291e$^{-10}$ m & bohr \\ 
         Energía & $E_h=e^2/4\pi \varepsilon_0a_0$  & 4.359e$^{-18}$ J & hartree \\ 
    \hline
    \end{tabular}
    \caption{Conversión entre unidades atómicas y del SI.}
    \label{tb:au}
\end{table}

\section{Hamiltoniano para el átomo hidrogenoide}
En este caso podremos reducir el problema a la suma de dos problemas 
más sencillos de una sola partícula. Por un lado, el sistema se 
desplaza como si toda la masa estuviera concentrada en un solo punto,
el centro de masas. Por otra parte, podemos describir el movimiento 
relativo de rotación y vibración de una partícula ficticia sometida 
a un potencial central. Esta simplificación no se puede aplicar en 
general a problemas de tres o más cuerpos.

Para átomos hidrogenoides expresamos el hamiltoniano con la siguiente
expresión
\begin{equation}
    \hat{H}=\hat{K}_N(\mathbf{R}_N) + \hat{K}_e(\mathbf{r}_e) + \hat{V}_{eN}(r_{eN})
\end{equation}
donde el primer y segundo término en el lado derecho de la ecuación 
corresponden a la energía cinética del núcleo y del electrón, 
respectivamente, y el último término es el potencial electróstático 
entre núcleo y electrón. Desarrollando los diferentes términos,
obtenemos
\begin{equation}
    \hat{H}=-\frac{\hbar^2}{2\mu}\nabla_r^2+ \frac{\hat{l}}{2\mu r^2} - \frac{Ze^2}{4\pi \varepsilon_0r}
\end{equation}
El operador $\hat{H}$ conmuta con $\hat{l}_z$ y $\hat{l}^2$. Estos
operadores tienen funciones propias comunes y los estados estacionarios
se pueden caracterizar por $E$, $l^2$ y $l_z$. 

\section{Soluciones de la ecuación de Schrödinger}
Dado que la energía potencial es centrosimétrica (independiente
de la componente angular), la ecuación de Schrödinger es separable
en componentes radial y angular. Así, las funciones de onda
hidrogenoides, denominadas \textbf{orbitales de Mulliken}, son de la forma
\begin{equation}
    \psi=R_{n,l}(r)Y^{m_l}_l(\theta,\phi)
\end{equation}
Los orbitales son funciones de onda monoelectrónicas para un
electrón en un átomo.


\subsection{Solución radial}
En primer lugar, la parte radial de la ecuación de Schrödinger es
\begin{equation}
    \bigg\{-\frac{\hbar^2}{2\mu r^2}\frac{\partial}{\partial r}\bigg(r^2\frac{\partial}{\partial r}\bigg) 
    +  \frac{l(l+1)\hbar^2}{2\mu r^2}
    - \frac{Ze^2}{4\pi \epsilon_0r}  \bigg\}
    R(r)
    = ER(r)
\end{equation}
Aunque no vamos a desarrollar la solución de esta ecuación, la función
radial tiene la siguiente estructura
\begin{equation}
    R(r)=(\textrm{polinomio en }r)\times(\textrm{exponencial en }r)
\end{equation}
Estas funciones radiales se expresan con mayor facilidad 
en función de la magnitud adimensional $\rho$, que definimos como
\begin{equation}
    \rho = \frac{2Zr}{na_0}
\end{equation}
En concreto la forma funcional, dependiente de los números
cuánticos principal ($n$) y angular ($l$), es la siguiente
\begin{equation}
  R_{n,l} = N_{n,l}\rho^l L^{2l+1}_{n+l}(\rho)\exp(-\rho/2)
\end{equation}
donde $n=1,2,3...$ y $l=0,...,n-1$. 
%\begin{equation}
%    R_{nl}(r) = -\bigg\{ \frac{4Z^3}{n^4a^3}\frac{(n-l-1)!}{[(n+l)!]^3} %\bigg\}^{1/2} \bigg(\frac{2Zr}{na}^l\bigg)\exp\big(\frac{-Zr}{na}\big)
%    L^{2l+1}_{n+l}\bigg(\frac{2Zr}{na}\bigg%)
%\end{equation%}
donde $L$ es un polinomio asociado de Laguerre. A pesar de la
complejidad de esta función, los polinomios dependientes de $n$ 
y $l$ son relativamente sencillos. 

Mostramos las 
funciones resultantes para los primeros números cuánticos en
la Tabla \ref{tb:radial_hydrogenoid}. La parte exponencial de
la función garantiza que la función decae hasta cero cuando
el electrón se aleja del núcleo, el término $\rho^l$ hace
que, para $l>0$, la función de onda sea cero en el núcleo, 
mientras que el polinomio de Laguerre genera una serie de 
oscilaciones entre valores positivos y negativos, que resulta
en la presencia de $(n-l)$ nodos. 

\begin{table}[b!]
    \centering
    \begin{tabular}{|c|c|c|c|}
    \hline
         Orbital & $n$  & $l$ &$R_{n,l}$ \\
    \hline
    \hline
         1s & 1 & 0 & 
         $2\bigg(\frac{Z}{a}\bigg)^{3/2}\mathrm{e}^{-\rho/2}$ \\
         2s & 2 & 0 & 
         $\frac{1}{8^{1/2}}\bigg(\frac{Z}{a}\bigg)^{3/2}(2-\rho) \mathrm{e}^{-\rho/2}$\\
         2p & 2 & 1 & 
         $\frac{1}{24^{1/2}}\bigg(\frac{Z}{a}\bigg)^{3/2}\rho \mathrm{e}^{-\rho/2}$\\
         3s & 3 & 0 & 
         $\frac{1}{243^{1/2}}\bigg(\frac{Z}{a})^{3/2}(6-6\rho+\rho^2)\mathrm{e}^{-\rho/2}$ \\
         3p & 3 & 1 &
         $\frac{1}{486^{1/2}}\bigg(\frac{Z}{a}\bigg)^{3/2}(4-\rho)\rho \mathrm{e}^{-\rho/2}$ \\
         3d & 3 & 2 & $\frac{1}{2430^{1/2}}\bigg(\frac{Z}{a}\bigg)^{3/2}\rho^2\mathrm{e}^{-\rho/2}$ \\
    \hline
    \end{tabular}
    \caption{Funciones de onda radiales de átomos hidrogenoides. En estas expresiones, 
    usamos $a=4 \pi \varepsilon_0\hbar^2/\mu e^2$}. %$e'=e/\sqrt{4\pi\varepsilon_0}$,
% y $E'_h=e'^2/a$.
    \label{tb:radial_hydrogenoid}
\end{table}

Asimismo llegamos a expresiones compactas para la energía de 
los estados ligados de los átomos hidrogenoides
\begin{equation}
    %E_n=-\frac{Z^2}{2n^2}\frac{\mu e'^4}{\hbar^2}=
    %-\frac{Z^2}{2n^2}\frac{\mu e'^4}{a}= -\frac{Z^2}{2n^2}E'_h
    E_n=-\frac{Z^2\mu e^4}{32\pi^2\varepsilon_0^2\hbar^2 n^2}
    \label{eq:Eh}
\end{equation}
que dependen exclusivamente del número cuántico $n$. 
En el caso del hidrógeno, podemos aproximar la masa reducida
a la masa del electrón $\mu\simeq m_e$. 

\subsection{Solución de la parte angular}
A continuación, nos centraremos en la parte angular, 
\begin{equation}
    \Lambda^2Y=-l(l+1)Y
\end{equation}
que ya resolvimos en el tema anterior. Las funciones angulares $Y$ 
son los armónicos esféricos, especificados por los números cuánticos
angular ($l$) y magnético ($m_l$). Estas funciones dependen únicamente
de la dirección y son independientes de la distancia. 

\section{Orbitales atómicos del hidrógeno}
A partir de la función de onda podemos derivar una 
propiedades de los orbitales del átomo de hidrógeno, 
caracterizados por los tres números cuánticos $n$, $l$ y
$m$. Cuando un electrón ocupa un orbital descrito por la
función de onda $\psi_{nlm}$, decimos que se
encuentra en el estado $|nlm\rangle$.

Como venimos diciendo, el número cuántico principal
es el número $n$, que puede tomar valores enteros $n=1,2,3...$
y determina la energía del electrón, descrita por la ecuación \ref{eq:Eh}.
Todas las energías de los estados ligados del átomo son negativas,
es decir, inferiores a la del electrón infinitamente separado
del núcleo que corresponde al cero en energía.

Los otros dos números cuánticos, $l$ y $m$, proceden de la solución
angular y especifican el momento angular del electrón en torno al 
núcleo. Un electrón con número cuántico $l$ tiene un momento angular
de magnitud $\{l(l+1)\}^{1/2}\hbar$, con $l=0,1,2,...n-1$. 
Las contribuciones al momento angular, sin embargo, no pueden
determinar con precisión de acuerdo con el principio de
incertidumbre de Heisenberg. La contribución en el eje $z$ del 
momento angular sí que se puede determinar con precisión, y su valor 
será $l_z=m_l\hbar$, con $m=0,\pm 1,...\pm l$.

\subsection{Capas y subcapas}
En general, las funciones con un número cuántico principal común $n$
tienen la misma energía y su región de máxima densidad a distancias
muy similares. Esto resulta en una estructura de capas. Los orbitales
con $n=1,2,3,4...$ resultan en las capas $K$, $L$, $M$ y $N$, respectivamente.
Los orbitales con el mismo número cuántico principal pero diferente 
número cuántico $l$ forman subcapas correspondientes a $l=0,1,2,3,4...$,
llamadas $s$, $p$, $d$, $f$, $g$... Para una capa con número cuántico 
$n$, hay un total de $n$ subcapas (de 0 hasta $n-1$). El número de 
orbitales atómicos en una capa correspondiente al número cuántico 
principal $n$ es $n^2$, con lo cual el un átomo hidrogenoide
la degeneración energética es de $n^2$.

\subsection{Geometría de los orbitales atómicos}
El orbital ocupado en el estado fundamental es el que tiene $n=1$
y $l=m_l=0$. El valor de $R$ lo podemos tomar de la Tabla~
\ref{tb:radial_hydrogenoid} y la función de onda angular en 
este caso es una constante, $Y_{00}=1/\sqrt{4\pi}$.
Esto resulta en una función de onda 
\begin{equation}
    \psi=\frac{1}{(\pi a_0^3)^{1/2}}\mathrm{e}^{-r/a_0}
\end{equation}
con simetría esférica que decae exponencialmente desde un máximo
que se encuentra en el núcleo ($r=0$). Por tanto la \textbf{densidad 
electrónica} $|\psi_{nlm}|^2=R^2_{nl}|Y_{lm}|^2$. 
será máxima en el núcleo mismo y decaerá al alejarnos de él. 
En el caso de los orbitales p ($l=1$) la función angular no 
es constante y la probabilidad de encontrar el electrón a 
una distancia $r$ difiere en función de la orientación.

La función de onda nos dice, a partir del valor de $\psi_{nlm_l}^2$,
la probabilidad de encontrar un electrón en una región determinada.
Si queremos calcular la probabilidad de encontrar a un electrón
en una región correspondiente a un casquete esférico ubicado 
a una distancia $r$ del núcleo y con grosor $dr$ y volumen 
$4\pi r^2dr$ entonces debemos calcular la \textbf{función de distribución 
radial}
\begin{equation}
    P(r)=4\pi r^2R^2_{nl}(r)
\end{equation}
$P(r)$ nos permite conocer la densidad de probabilidad de que el electrón
se encuentre en un casquete esférico de radio $r$. Al multiplicar
esta densidad de probabilidad por $dr$ obtenemos la probabilidad de
encontrar el electrón en cualquier lugar del casquete esférico 
correspondiente. Para un orbital 1s, el valor de la función de
distribución radial es 
\begin{equation}
    P(r)=\frac{4Z^3}{a_0^3}r^2\exp{(-2Zr/a_0)}
\end{equation}
A partir de esta expresión podemos ver fácilmente que en el núcleo, donde
$r=0$ y por tanto $r^2=0$, $P(0)=0$. En segundo lugar, al alejarnos mucho
del núcleo, $r\xrightarrow{}\infty$, $P(r)$ también tiende a cero. Finalmente,
la función tienen un máximo a valores intermedios de $r$, que pueden determinarse
mediante diferenciación. En particular, en el caso del orbital 1s del
hidrógeno el máximo se encuentra en $r=a_0$. Para el orbital 2s, el radio
más probable está en $5.2a_0$. 

%En el caso de los orbitales p, con $l=1$, 
%$\vec{l}^2=2\hbar^2$. La contribución en el eje $z$ del momento angular
%para los orbitales p es $l_z=-\hbar$ (para $m=-1$), $l_z=0$ 
%(para $m=0$) y $l_z=\hbar$ (para $m=1$). El valor definido de $l_z$ implica
%que no podemos determinar el ángulo $\varphi$ con respecto al eje $z$, que
%todas las orientaciones con respecto al eje $z$ son igualmente probables y 
%por tanto la nube electrónica es simétrica y que estos orbitales tienen simetría
%axial.

\section{Efecto Zeeman}
En 1896 Zeeman descubre que la aplicación de un campo magnético conlleva un
desdoblamiento de líneas espectrales. El denominado efecto Zeeman es la 
modificación del espectro atómico por la aplicación de un campo magnético
fuerte. Surge de la interacción entre el campo magnético aplicado y los
momentos magnéticos debidos al momento angular orbital y de spin. 
En ausencia de un campo magnético la energía de un átomo de hidrógeno es
independiente del número cuántico $m$. Pero en presencia de un campo magnético
la energía de un átomo de hidrógeno depende de $m$ y es por tanto diferente 
para las posibles orientaciones del vector momento angular. 

El efecto Zeeman resulta en una pérdida de la degeneración energética
de los orbitales. Supongamos un estado con valores de los números 
cuánticos $n$  y $l$. A partir de la Ecuación\ref{eq:Eh} todos los
orbitales tendrían la misma energía. Sin embargo, dado que este estado
tiene $2l+1$ valores de $m$, en presencia de un campo magnético estará 
dividido en $2l+1$ niveles. 

Un estado s de un átomo de hidrógeno no experimenta este desdoblamiento.
Pero el estado p se divide en tres subniveles, correspondientes
a las diferentes orientaciones de $l_z$. De acuerdo con esto, al aplicar 
un campo magnético a un átomo de hidrógeno en su estado fundamental (1s), 
no deberíamos observar ningún desdoblamiento. Sin embargo, experimentalmente
sí que se pueden observar dos bandas diferentes, lo cual indica un momento
magnético adicional.

Stern y Gerlach realizaron un afamado experimento haciendo pasar un 
haz de átomos de Ag a través de un campo magnético, en el que observaron
un desdoblamiento. Por tanto, los átomos de Ag debían tener un momento
magnético cuya orientación con respecto al eje $z$ estaba cuantizada. 
Estas dos bandas contradecían las $2l+1$ orientaciones posibles del
momento angular determinado por $l$. Goudsmith y Uhlenbeck proponen 
en 1925 un momento angular intrínseco del electrón, $s$. El electrón 
debía tener un movimiento de giro alrededor de sí mismo, que fue 
denominado de \textbf{espin}. La magnitud correspondiente es el momento 
angular intrínseco 
\begin{equation}
    |\vec{s}|=\sqrt{s(s+1)}\hbar
\end{equation}
y su componente en el eje $z$ es $s_z=m_s\hbar$, donde $m_s=-s,...,s$.
En el caso del electrón, $s=1/2$, lo que resulta en dos orientaciones
posibles: $m_s=+1/2$, o $\alpha$ o espin hacia arriba y $m_s=-1/2$ o
$\beta$ o espin hacia abajo. A partir de esta nueva contribución,
se definen los espin orbitales, que incluyen tanto las contribuciones
radial y angular de la función de onda como la de espin, es decir,
$\psi(r,\theta,\varphi)\alpha(\omega)$ y 
$\psi(r,\theta,\varphi)\beta(\omega)$.