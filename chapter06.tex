\chapter{Métodos Aproximados}
Como comentamos en el Tema anterior, la ecuación de 
Schrödinger sólo se puede resolver de manera exacta 
para átomos mono-electrónicos. Cuando un sistema tiene 
dos o más electrones es necesario recurrir
a métodos aproximados, que son principalmente dos: el
método variacional y los métodos perturbativos. 

\section{El teorema variacional}
Sea $\hat{H}$ el operador de Hamilton y $\psi$ una función 
de onda arbitraria que utilizamos para definir un sistema, 
el \textbf{teorema variacional} nos dice que el valor 
esperado de la energía al aplicar el operador
sobre la función es siempre superior a la verdadera 
energía del estado fundamental ($E_0$). Matemáticamente, 
podemos escribir este teorema de la siguiente forma:
\begin{equation}
    E=\frac{\int{\psi^\star\hat{H}\psi d\tau}}{\int{\psi^\star \psi d\tau}} \geq E_0
    \label{eq:evar}
\end{equation}
Al cociente de la Ecuación \ref{eq:evar} lo llamamos
\textbf{integral variacional} y a la función de onda arbitraria
se la llama \textbf{función de prueba}. El principio establece
que si modificamos esta función de tal manera que minimicemos
la energía, entonces nos acercaremos al máximo a la función 
de onda real. La igualdad en la Ecuación \ref{eq:evar} se 
cumple sólo en el caso en el que la función de onda de prueba 
sea idéntica a la función de onda real. El teorema variacional
proporciona una manera objetiva de aproximar la verdadera función
de onda. Este principio es la base de todos los métodos modernos
de  cálculo de estructura electrónica.

Supongamos que partimos de las funciones $\{\psi_k\}$
que son funciones propias del hamiltoniano , de tal manera que
a cada una de las funciones de onda corresponde un valor propio
para la energía,
$\hat{H}\psi_k=E_k\psi_k$. Supongamos asimismo que este conjunto
de funciones de onda forma un conjunto ortonormal. Así, la función
de onda para el estado del sistema se podrá expresar como
combinación lineal de las funciones $\psi_k$
\begin{equation}
    \psi=\sum_kc_k\psi_k
\end{equation}
y la energía correspondiente a ese estado será
\begin{equation}\begin{split}
    E=\frac{\int{\psi^\star\hat{H}\psi d\tau}}{\int{\psi^\star \psi d\tau}}=  
    \frac{\sum_i\sum_jc_i^\star c_j\int{\psi_i^\star\hat{H}\psi_j d\tau}}{\sum_i\sum_jc_i^\star c_j\int{\psi_i^\star\psi_j d\tau}}= \\
 = \frac{\sum_i\sum_jc_i^\star c_jE_{j}\delta_{ij}}{\sum_i\sum_jc_i^\star c_j\delta_{ij}}=
    \frac{\sum_ic_i^\star c_iE_{i}}{\sum_ic_i^\star c_i}
\end{split}
\end{equation}
A partir del teorema variacional podemos utilizar dos tipos
de técnicas aproximadas: el método variacional simple y el 
método de variaciones lineal.

\subsection{El método variacional simple}
En este método construimos una función de onda de prueba con 
una serie de parámetros libres y minimizamos la energía $E$
con respecto a estos parámetros. 

Usaremos como ejemplo el caso del oscilador armónico, que 
hemos resuelto de manera exacta en el Tema 3. Para describir
el estado fundamental de este sistema, con hamiltoniano
\begin{equation}
    \hat{H}=-\frac{\hbar^2}{2m}\frac{\partial^2}{\partial x^2} + 
    \frac{1}{2}kx^2
\end{equation}
podemos usar como función de prueba la función
\begin{equation}
 \phi=\cos\lambda x   
\end{equation}
con $-\frac{\pi}{2\lambda}<x<\frac{\pi}{2\lambda}$. En
este ejemplo, $\lambda$ es el parámetro variacional. 
Dada la definición de la integral variacional (Ecuación 
\ref{eq:evar}), los pasos para encontrar los valores de 
$\lambda$ que minimizan la energía serán:
\begin{enumerate}
    \item Resolver la integral del denominador y el numerador:
    operamos por separado la parte correspondiente al numerador 
    y la energía potencial ($\hat{V}=1/2k\hat{x}^2$) y cinética 
    ($\hat{K}$).
    \begin{align}
    & \int{\phi^\star\phi d\tau } = \frac{\pi}{2\lambda} \\
    & \int{\phi^\star\hat{x}^2\phi d\tau} =  \frac{\pi^3}{24\lambda^3}-\frac{\pi}{4\lambda^3} \\
    & \int{\phi^\star\hat{K}\phi d\tau}
    =\frac{\lambda^2\hbar^2}{2m}\int{\phi^\star\phi d\tau}
    \end{align}
    Por tanto, la energía en función del parámetro $\lambda$ es
    \begin{equation}
        E(\lambda) = \frac{\lambda^2\hbar^2}{2m} + \frac{k}{\lambda^2}\bigg(\frac{\pi^2}{24}-\frac{1}{4}\bigg)
    \end{equation}

    \item Minimizar $E$ con respecto a $\lambda$
    \begin{equation}
        \frac{dE}{d\lambda} = 0 \Rightarrow 
        \lambda^2_{min}=\frac{2^{1/2}}{\hbar}k^{1/2}m^{1/2}\bigg(\frac{\pi^2}{24}-\frac{1}{4}\bigg)^{1/2}
    \end{equation}
    
    \item Calcular la energía para la mejor aproximación al
    estado fundamental.
    \begin{equation}
        E(\lambda_{min})=1.14\frac{1}{2}\hbar\sqrt{\frac{k}{m}}
    \end{equation}
    En este ejemplo, dado que la energía del oscilador armónico
    es $1/2\hbar\omega$ estamos sobreestimando la energía en un
    14 \%.
\end{enumerate}

\subsection{El método de variaciones lineal}
En el método de variaciones lineal partimos de un conjunto de
funciones linealmente independientes, 
 $\{\varphi_1$, $\varphi_2$, ..., $\varphi_N\}$, 
 que pueden ser o no ortonormales entre sí,  y nos limitamos a
 valores de $\phi_j$ y $c_j$ reales. 
 Podemos escribir una función de onda aproximada como
combinación lineal de este conjunto:
\begin{equation}
    \phi=\sum_{j=1}^{N}c_j\varphi_j
\end{equation}
De este modo, podremos calcular la energía como 
\begin{equation}
    E=\frac{\int\phi^\star\hat{H}\phi d\tau}{\int\phi^\star\phi d\tau}
\end{equation}
y al ser $\phi$ una combinación lineal de funciones $\varphi_k$,
la energía dependerá de los coeficientes $c_k$.
Reorganizando los términos de esta expresión podemos obtener 
la siguiente expresión
\begin{equation}
    E\sum_{j=1}^N\sum_{i=1}^N c_j c_iS_{ji}=\sum_{j=1}^N\sum_{i=1}c_jc_iH_{ji}
\end{equation}
donde estamos usando la \textbf{integral de solapamiento}
$S_{ji}=\int \varphi^\star\varphi d\tau=S^\star_{ij}=S_{ij}$.
De acuerdo con el principio variacional, la condición que nos 
permite llegar a la combinación lineal de mínima energía es
\begin{equation}
    \frac{\partial E}{\partial c_k}=0
    \label{eq:var_deriv}
\end{equation}

Dado que el método variacional lineal se vuelve complicado
incluso para sistemas muy pequeños, trabajaremos con el ejemplo
más sencillo, correspondiente a un sistema que aproximamos
con una función de onda 
\begin{equation}
    \psi=c_1\phi_1 + c_2\phi_2
\end{equation}
Si desarrollamos la integral variacional, obtenemos lo siguiente
\begin{equation}
\begin{split}
    E&=\frac{\int\psi^\star\hat{H}\psi d\tau}{\int\psi^\star\psi d\tau}  
    = \frac{\int( c_1\phi_1+ c_2\phi_2)\hat{H}(c_1\phi_1+ c_2\phi_2)d\tau}{\int (c_1\phi_1+ c_2\phi_2)(c_1\phi_1+ c_2\phi_2)d\tau}=\\
    &= \frac{c_1^2\int\phi_1\hat{H}\phi_1 d\tau +
    c_2^2\int\phi_2\hat{H}\phi_2 d\tau +
    c_1c_2\int\phi_1\hat{H}\phi_2 d\tau +
    c_2c_1\int\phi_2\hat{H}\phi_1 d\tau}
    {c_1^2\int\phi_1\phi_1 d\tau +
    c_2^2\int\phi_2\phi_2 d\tau +
    c_1c_2\int\phi_1\phi_2 d\tau +
    c_2c_1\int\phi_2\phi_1 d\tau} =\\
    & = \frac{c_1^2H_{11} + c_2^2H_{22} + c_1c_2H_{12} + c_2c_1H_{21}}{c_1^2 + c_2^2 + c_1c_2S_{12} + c_2c_1S_{21}}
    \end{split}
\end{equation}
Podemos reorganizar esta ecuación para la energía como
\begin{equation}
    E({c_1^2S_{11} + c_2^2S_{22} + c_1c_2S_{12} + c_2c_1S_{21}})=c_1^2H_{11} + c_2^2H_{22} + c_1c_2H_{12} + c_2c_1H_{21}
    \label{eq:var_exp}
\end{equation}
%En este desarrollo hemos asumido que las funciones 
%son ortonormales y por tanto  $S_{ij}=\delta_{ij}$.
%Asimismo, estamos definiendo 
%$\alpha_k=\int \phi_k\hat{H}\phi_kd\tau$
%y 
%$\beta =\int \phi_1\hat{H}\phi_2d\tau= \int
%\phi_2\hat{H}\phi_1d\tau$.
En este caso, al tener únicamente dos funciones de onda, 
la Ecuación \ref{eq:var_deriv} pasa a ser
\begin{align}
    \frac{\partial E}{\partial c_1}&=0       & \frac{\partial E}{\partial c_2}&=0
\end{align}
Podemos por tanto derivar con respecto a $c_1$ y $c_2$
la Ecuación \ref{eq:var_exp} y así obtenener
\begin{align}
 E(2c_1S_{11} +2c_2S_{12})    &=2c_1H_{11} +2c_2H_{12} \\
 E(2c_2S_{22} +2c_1S_{21})    &=2c_2H_{22} +2c_1H_{21} 
\end{align}
Esta expresión se puede reescribir en forma matricial
\begin{equation}
\begin{pmatrix}
 H_{11}-ES_{11} &  H_{12}-ES_{12}  \\ 
 H_{21}-ES_{21} &  H_{22}-ES_{22}
\end{pmatrix}
\begin{pmatrix}
c_{1}  \\ 
c_{2}
\end{pmatrix}=
\begin{pmatrix}
0\\
0
\end{pmatrix}
\end{equation}
que tiene solución no trivial ($c_1=c_2=0$) cuando resolvemos 
el determinante secular 
\begin{equation}
\begin{vmatrix}
 H_{11}-ES_{11} &  H_{12}-ES_{12}  \\ 
 H_{21}-ES_{21} &  H_{22}-ES_{22}
\end{vmatrix}= 0
\end{equation}
El determinante secular arroja dos valores posibles
para la energía. Entre ellas, la más pequeña es 
la que nos quedaremos como aproximación para la energía
del estado fundamental.

\section{Los métodos perturbativos}
En este método, asumimos que el hamiltoniano exacto de 
un sistema puede aproximarse como suma de un término 
fundamental, $H^{(0)}$, cuyas funciones propias 
$\psi^{(0)}$ y valores propios $E^{(0)}$ son conocidos,
más otros términos que tienen menor peso en el valor total 
\begin{equation}
    \hat{H}=\hat{H}^{(0)} + \lambda\hat{H}^{(1)} + 
    \lambda^2 \hat{H}^{(2)} + ...
    \label{eq:H_pert}
\end{equation}
En esta ecuación, los términos $\hat{H}^{(1)}$ y 
$\hat{H}^{(2)}$ representan la diferencia entre el
hamiltoniano real y el modelo. En general,
$\lambda \ll 1$, de modo que las potencias
$\lambda^i\hat{H}^{(i)}$ decrecen sustancialmente 
a medida que aumenta $i$. En los desarrollos que 
incluimos a continuación, consideramos únicamente los
términos correspondientes a $\hat{H}^{(0)}$ y 
$\hat{H}^{(1)}$.

Del mismo modo que hemos hecho para el hamiltoniano, 
podemos escribir las funciones de onda para el estado
del sistema que nos interesa describir y su energía 
como suma de un término fundamental y sus perturbaciones
\begin{align}
    \psi =& \psi_k^{(0)} + \lambda\psi_k^{(1)} + \lambda^2\psi_k^{(2)} + ... \label{eq:psi_pert}\\
    E =& E_k^{(0)} + \lambda E_k^{(1)} + \lambda^2E_k^{(2)}+ ...
    \label{eq:E_pert}
\end{align}
En esta ecuación, $E_k^{(1)}$ es la corrección de primer orden
para la energía,  $E_k^{(2)}$ es la corrección de segundo orden 
para la energía, etc. Como sabemos que podemos obtener la energía
a partir de $\hat{H}\psi = E\psi$, podemos desarrollar esta
expresión en función de los términos de las Ecuaciones
\ref{eq:H_pert}, \ref{eq:psi_pert} y \ref{eq:E_pert}:
\begin{equation}
\begin{split}
    \hat{H}\psi =& (\hat{H}^{(0)} + \lambda \hat{H}^{(1)} + ...)(\psi^{(0)} + \lambda\psi^{(1)} +  + \lambda\psi^{(2)} + ...) = \\
    =& \hat{H}^{(0)}\psi^{(0)} + 
    \lambda (\hat{H}^{(1)}\psi^{(0)} + \hat{H}^{(0)}\psi^{(1)}) + \\ 
    &+ \lambda^2 (\hat{H}^{(2)}\psi^{(0)} + \hat{H}^{(1)}\psi^{(1)} + \hat{H}^{(2)}\psi^{(0)} ) =  \\
    =& E^{(0)}\psi^{(0)} + \lambda (E^{(1)}\psi^{(0)} + E^{(0)}\psi^{(1)}) + \\
    &+ \lambda^2 (E^{(2)}\psi^{(0)} + E^{(1)}\psi^{(1)} + E^{(2)}\psi^{(0)})
\end{split}
\end{equation}
Podemos agrupar los términos en esta ecuación en función del
exponente de $\lambda$ para obtener
\begin{align}
    \lambda^0\mathrm{:} & \hat{H}^{(0)}\psi^{(0)}= E^{(0)}\psi^{(0)}\label{eq:HE_pert0}\\
    \lambda^1\mathrm{:}&
    \hat{H}^{(1)}\psi^{(0)} + \hat{H}^{(0)}\psi^{(1)} = E^{(1)}\psi^{(0)} + E^{(0)}\psi^{(1)} \label{eq:HE_pert1}\\
    \lambda^2\mathrm{:}  &     
     \hat{H}^{(0)}\psi^{(2)} + \hat{H}^{(1)}\psi^{(1)} = E^{(2)}\psi^{(0)} + E^{(1)}\psi^{(1)}+ E^{(0)}\psi^{(2)}
\end{align}

A partir de este conjunto de ecuaciones, podemos empezar a trabajar
con la función de onda del estado fundamental, $\psi_0$. La Ecuación
\ref{eq:HE_pert0} pasa a ser
\begin{equation}
    \hat{H}^{(0)}\psi_0^{(0)} = E_0^{(0)}\psi_0^{(0)}
\end{equation}
donde tenemos la energía y la función de onda conocidas. La segunda ecuación
, correspondiente al exponente $\lambda$ (Ec. \ref{eq:HE_pert1}), 
es para el estado fundamental
\begin{equation}
    \hat{H}^{(1)}\psi_0^{(0)} + \hat{H}^{(1)}\psi_0^{(1)} = 
    E_0^{(0)}\psi_0^{(1)} + E_0^{(1)}\psi_0^{(0)}
\end{equation}
Podemos expresar la perturbación de la función de onda $\psi_0^{(1)}$ 
como combinación lineal de las funciones propias de $\hat{H}^{(0)}$,
de modo que $\psi_0^{(1)}=\sum_nc_n\psi_n^{(0)}$. Sustituyendo
dentro de la expresión anterior, obtenemos
\begin{equation}
    \hat{H}^{(1)}\psi_0^{(0)} + \sum_nc_n\hat{H}^{(0)}\psi_n^{(0)} = 
    \sum_nc_nE_0^{(0)}\psi_n^{(0)} + E_0^{(1)}\psi_0^{(0)}
\end{equation}
Si multiplicamos esta ecuación por la izquierda por $\psi_0^{(0)}^\star$
e integramos llegamos a
\begin{equation}
\begin{split}
  \int \psi_0^{(0)}^\star \hat{H}^{(1)}\psi_0^{(0)}d\tau +  \sum_nc_n\int \psi_0^{(0)}^\star\hat{H}^{(0)}\psi_n^{(0)}d\tau = \\
    = \sum_nc_nE_0^{(0)}\int \psi_0^{(0)}^\star\psi_n^{(0)}d\tau + \int \psi_0^{(0)}^\star E_0^{(1)}\psi_0^{(0)}d\tau
\end{split}
\end{equation}
Dada la condición de ortonormalidad, el segundo término a la izquierda de
la igualdad y el primero por la derecha serán $0$ para $n\neq 0$. Por otra
parte, dada la condición de normalización, el último término es igual a 
$E_0^{(1)}$. De este modo llegamos a una expresión compacta para el
valor promedio de la perturbación que obtenemos a partir de la función 
de onda sin perturbar
\begin{equation}
    E_0^{(1)}= \int \psi_0^{(0)}^\star \hat{H}^{(1)}\psi_0^{(0)}d\tau 
\end{equation}

Haciendo otra manipulación, podemos obtener los valores de los 
coeficientes $c_k$ de la combinación lineal. Si multiplicamos la
Ecuación \ref{eq:HE_pert1} por $\psi_k^{(0)}^\star$ para $k\neq 0$,
obtenemos
\begin{equation}
    \begin{split}
        \int \psi_k^{(0)}^\star \hat{H}^{(1)}\psi_0^{(0)}d\tau +  \sum_nc_n\int \psi_k^{(0)}^\star\hat{H}^{(0)}\psi_n^{(0)}d\tau = \\
    = \sum_nc_nE_0^{(0)}\int \psi_k^{(0)}^\star\psi_n^{(0)}d\tau + \int \psi_k^{(0)}^\star E_0^{(1)}\psi_0^{(0)}d\tau
    \end{split}
\end{equation}
Esta expresión se puede simplificar usando las condiciones de
ortogonalidad y normalización de las funciones de onda 
$\{\psi_n^{(0)}\}$ para alcanzar la siguiente ecuación:
\begin{equation}
    c_k=-\frac{\int \psi_k^{(0)}\hat{H}^{(1)}\psi_0^{(0)}}{E_k^{(0)} - E_0^{(0)}}
\end{equation}

Finalmente, aunque no haremos esta derivación de manera explícita,
a partir de los términos correspondientes a $\lambda^2$
podemos derivar la expresión para la corrección de segundo orden de
la energía
\begin{equation}
    E_0^{(2)} = \sum_{n\neq 0}\frac{|\int\psi_n^{(0)}^\star\hat{H}^{(1)}\psi_0^{(0)}d\tau|^2}{E_0^{(0)} - E_n^{(0)}} = 
    \sum_{n\neq 0}\frac{|H_{n0}^{(1)}|^2}{E_0^{(0)} - E_n^{(0)}}
\end{equation}

%\section{Método de Rayleigh-Schrödinger}
%Este método permite obtener $\psi^{(i)}_k$ como combinación 
%lineal de las funciones funciones propias sin 
%perturbar del hamiltoniano $\hat{H}^{(0)}$ (funciones de orden cero ):
%\begin{equation}
%    \psi_k^{(i)} = \sum_{j=0}^\infty \psi_j^{(0)}c_{kj}^{(i)}
%\end{equation}
%El subíndice $k$ en $c_{kj}^{(i)}$ representa el estado estacionario
%que tratamos de corregir, $(i)$ es el orden de la perturbación y $j$
%las funciones de orden cero para todos los estados estacionarios.
%Sustituyendo la expresión para la función de onda en la Ecuación
%\ref{eq:H_pert} obtenemos
%\begin{equation}
%    \sum c_{kj}(\hat{H}^{(0)}-E_k^{(0)})=
%    -(\hat{H}^{(1)}-E_k^{(1)}))\psi_k^{(0)}
%\end{equation}
%Hay dos casos a considerar.
%\begin{enumerate}
%    \item Caso en el que $n=k$:
%    \begin{equation}
%        E_n^{(1)} = \int \psi_n^{(0)}^\star \hat{H}^{(1)}\psi_n^{(0)} d\tau
%    \end{equation}
%    En este caso la corrección de primer orden se obtiene promediando
%    la perturbación $\hat{H}^{(1)}$ sobre las correspondientes
%    funciones de onda sin perturbar. Para evaluar la corrección de 
%    primer orden a la energía basta con conocer la función de onda
%    sin perturbar. Por tanto, la calcularemos usando la 
%    expresión
%    \begin{equation}
%        E_n\simeq E_n^{(0)} +  E_n^{(1)} = E_n^{(0)} +  \int \psi_n^{(0)}^\star \hat{H}^{(1)}\psi_n^{(0)} d\tau
%    \end{equation}
%
%    \item Caso en el que $n\neq k$: 
%    \begin{equation}
%        c_{kn} = -\frac{\int \psi_n^{(0)}^\star {\hat{H}}^{(1)}\psi_k^{(0)} d\tau}
%        {E_k^{(0)} - E_n^{(0)}}
%    \end{equation}
%    En este caso los coeficicentes $c_{kn}$ se calculan usando esta
%    ecuación. Para corregir a primer orden la energía de un estado
%    sólo necesitamos evaluar $\hat{H}^{(1)}$ en este estado y
%    conocer la función de onda sin perturbar. Las correcciones
%    se vuelven más y más complicadas al aumentar el orden de la
%    perturbación.
%\end{enumerate}
%Las ecuaciones que hemos introducido proporcionan la 
%siguiente expresión para la corrección de primer 
%orden de la función de onda:
%\begin{equation}
%    \psi_{k}^{(1)} = \sum_{n\neq k}\frac{\int {\psi_n^{(0)}}^\star \hat{H}^{(1)}\psi_k^{(0)} d\tau}{E_k^{(0)} - E_n^{(0)}}
%\end{equation}
%Haciendo $\lambda=1$ usando solamente la corrección de primer orden de la
%función de onda obtenemos como aproximación a la función de onda 
%perturbada
%\begin{equation}
%    \psi_k = \psi_k^{(0)} +  
%    \sum_{n\neq k}\frac{\int {\psi_n^{(0)}}^\star\hat{H}^{(1)}\psi_k^{(0)} d\tau}
%    {E_k^{(0)} - E_n^{(0)}}\psi_n^{(0)}
%\end{equation}

\section{Tratamiento perturbativo del átomo del Helio}
Bajo la aproximación de masa nuclear infinita ($m_N/m_e\rightarrow \infty$)
el hamiltoniano del átomo de dos electrones es
\begin{equation}
    \hat{H} = -\frac{1}{2}\hat{\nabla}^2_1 - \frac{Z}{r_1} 
            -\frac{1}{2}\hat{\nabla}^2_2 - \frac{Z}{r_2} + \frac{1}{r_{12}},
\end{equation}
donde $r_1$ y $r_2$ son las coordenadas de los electrones respecto 
del núcleo común, y $r_{12}=|\mathbf{r}_2 - \mathbf{r}_1|$ es la
distancia entre los electrones. En ausencia del término de repulsión 
coulombiana entre electrones el problema de orden cero es separable
en el movimiento independiente de dos electrones hidrogenoides
\begin{equation}
    \hat{H}^{(0)}\psi^{(0)}(\mathbf{r}_1,\mathbf{r}_2) = E^{(0)}\psi^{(0)}
\end{equation}
donde
\begin{align}
    \psi^{(0)} = & \psi^{(1)}(\mathbf{r}_1) \psi^{(2)}(\mathbf{r}_2) = \psi_{n_1, l_1, m_1}(1)\psi_{n_2, l_2, m_2}(2)\\
    E^{(0)} = & E_{n_1} + E_{n_2} = \bigg(-\frac{Z^2}{2n_1^2} -\frac{Z^2}{2n_2^2}\bigg)E_h&
\end{align}
En esta ecuación, $Z=2$, $n_1=n_2=1$ y $E_h=\frac{e^2}{4\pi \varepsilon_0 a_0}=4.360\times 10^{-18}$ J (1 hartree).
En este estado fundamental los dos electrones ocupan un orbital 1s de modo
que la energía de orden cero del helio será $E^{(0)}=-4$ hartree. La 
corrección perturbativa de orden uno será 
$E^{(1)}=5Z/8E_h=1.25$ hartree.