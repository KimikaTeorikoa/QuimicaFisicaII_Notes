\chapter{Átomos Polielectrónicos}
En este tema, veremos los conceptos más importantes para determinar 
la estructura electrónica en sistemas con múltiples electrones. Primero 
haremos una discusión cualitativa, introducremos el principio de 
exclusión de Pauli y la regla de Hund para el llenado de orbitales.

\subsection{La función de onda para átomos polielectrónicos}
Un \textbf{átomo polielectrónico} es aquél que tiene más de un electrón.
En un átomo con un sólo electrón, en el que se considera al núcleo
fijo en el origen de coordenadas, la ecuación de Schrödinger depende
de las tres coordenadas, ya sean cartesianas o esféricas, del electrón.
Del mismo modo para un un átomo con $N$ electrones, la correspondiente
ecuación de Schrödinger contiene $3N$ coordenadas, tres por cada uno 
de los electrones presentes en el átomo. A la hora de escribir la 
ecuación de Schrödinger de un átomo polielectrónico debemos tener 
en cuenta la atracción coulombiana del núcleo hacia todos y cada 
uno de los electrones, así como las repulsiones interelectrónicas.
Para este problema la ecuación de Schrödinger no tiene una solución 
analítica es lo que hace inviable la solución analítica, ni siquiera
en el caso del más sencillo de los átomos polielectrónicos, el átomo de 
helio. Por ello debemos recurrir a los métodos aproximados.

El modelo aproximado que se utiliza para describir las propiedades 
de los electrones en los átomos se denomina \textbf{aproximación orbital}. 
Así, suponemos en una primera aproximación que cada electrón ocupa su
“propio” orbital. Esto nos permite escribir la función
de onda $\psi(\mathbf{r}_1,\mathbf{r}_2,...\mathbf{r}_N)$ como producto de las
funciones de onda de los orbitales
\begin{equation}
    \psi(\mathbf{r}_1, \mathbf{r}_2, ..., \mathbf{r}_N) = \psi(\mathbf{r}_1)\psi(\mathbf{r}_2)...\psi(\mathbf{r}_ N)
\end{equation}
donde $\mathbf{r}_i$ es el vector que determina la posición del electrón $i$.
Podemos considerar a los orbitales individuales similares a orbitales
hidrogenoides, pero sometidos a cargas nucleares influidas por la 
presencia del resto de los electrones en el átomo. La aproximación 
orbital sería exacta si no hubiese interacciones entre electrones.
Esta descripción es sólo aproximada, pero es un modelo útil para explicar
las propiedades químicas de los átomos y el punto de partida de 
descripciones más complejas de la estructura atómica. 

%Una consecuencia muy importante de estas interacciones es que los orbitales con igual valor de $n$ pero diferentes valores de $l$ ya no son degenerados en un átomo polielectrónico.
\section{El átomo de Helio}
La aproximación orbital nos permite expresar la estructura electrónica
de los átomos en función de \textbf{configuraciones}. Por ejemplo, en el caso 
de los átomos hidrogenoides, para el estado fundamental hablaríamos de la 
configuración $1s^1$. El caso del helio (He) es más complejo, al tener dos 
electrones. El primer orbital en llenarse sería el $1s^1$, pero al tener
una carga superior, $Z=2$, el orbital es más compacto. El segundo electrón
ocupa también el orbital $1s$, con lo cual alcanzamos la configuración del
estado fundamental del átomo de He, $1s^2$.

A la hora de escribir el hamiltoniano el término correspondiente a la
energía potencial para el He es la suma de tres contribuciones, dos 
negativas debidas a la atracción entre cada uno de los electrones y
el núcleo y una positiva debida a la repulsión entre los electrones:
\begin{equation}
        V= -\frac{2e^2}{4\pi \varepsilon_0r_1} -\frac{2e^2}{4\pi \varepsilon_0r_2} + \frac{e^2}{4\pi \varepsilon_0r_{12}} 
\end{equation}
Así, el hamiltoniano tendrá la forma
\begin{equation}
    \hat{H}=  -\frac{\hbar^2}{2m_e}\nabla_1^2 -\frac{\hbar^2}{2m_e}\nabla_2^2 -\frac{2e^2}{4\pi \varepsilon_0r_1} -\frac{2e^2}{4\pi \varepsilon_0r_2} + \frac{e^2}{4\pi \varepsilon_0r_{12}}
\end{equation}
Si aplicamos el operador sobre la función de onda, $\hat{H}\psi=E\psi$.
\begin{align}
    -\frac{\hbar^2}{2m_e}\nabla_1^2\psi 
    -\frac{\hbar^2}{2m_e}\nabla_2^2\psi 
    -\frac{Ze^2}{4\pi \varepsilon_0r_1}\psi 
    -\frac{Ze^2}{4\pi \varepsilon_0r_2}\psi 
    + \frac{e^2}{4\pi \varepsilon_0r_{12}}\psi 
    = E\psi
    \label{eq:he_exact}    
\end{align}
Los cuatro primeros términos en el lado izquierdo de la ecuación 
dependen o bien de las coordenadas del $e^-_1$ o de las coordenadas 
del $e^-_2$. El último término, que representa la repulsión interelectrónica,
en cambio, incluye la distancia $r_{12}$ y por tanto depende de las distancia entre ambos electrones. Este término impide separar la ecuación en términos 
dependientes de las coordenadas del $e^-_1$ o de las coordenadas 
del $e^-_2$, y así, resolver analíticamente la ecuación. 

Una aproximación muy drástica consiste en eliminar de la Ecuación 
\ref{eq:he_exact} el término correspondiente a las repulsiones 
interelectrónicas. Así, nuestro hamiltoninano se podrá escribir
de la siguiente forma
\begin{align}
    -\frac{\hbar^2}{2m_e}\big(\nabla_1^2\psi + \nabla_2^2\psi\big)
    -\frac{Ze^2}{4\pi \varepsilon_0r_1}\psi 
    -\frac{Ze^2}{4\pi \varepsilon_0r_2}\psi 
    = E\psi
\label{eq:he_aprox}    
\end{align}
La Ecuación \ref{eq:he_aprox} sí tiene una solución exacta, que 
corresponde a la superposición de los estados individuales de 
cada uno de sus electrones. Dentro de la aproximación orbital, 
podemos escribir 
\begin{equation}
    \psi(\mathbf{r}_1,\mathbf{r}_2)=\psi_1(\mathbf{r}_1)\psi_2(\mathbf{r}_2)
\end{equation}
donde $\psi_1$ y $\psi_2$ son los estados de los electrones  
1 y 2 respectivamente. Dado que hemos eliminado la repulsión
interelectrónica, $\psi_1$ y $\psi_2$ son las soluciones de 
las siguientes ecuaciones monoelectrónicas,
\begin{align}
    -\frac{\hbar^2}{2m_e}\nabla_1^2\psi_1 -\frac{Ze^2}{4\pi \varepsilon_0r_1}\psi_1
    = E_1\psi_1\\
    -\frac{\hbar^2}{2m_e}\nabla_2^2\psi_2 -\frac{Ze^2}{4\pi \varepsilon_0r_2}\psi_2
    = E_2\psi_2
\label{eq:he_aprox}    
\end{align}
La energía total del sistema Ε es igual a la suma de las 
energías de los electrones individuales $E=E_1 + E_2$. Como
cabe sospechar, esta es una mala aproximación, que sobrestima
considerablemente la estabilidad del átomo de He, al ignorar 
la repulsión entre los electrones.


\section{Principio de Pauli}
El principio de Pauli establece lo siguiente: 
\begin{displayquote}
Cuando intercambiamos los índices posiciones de dos fermiones 
la función de onda total cambia de signo, mientras que cuando
cambiamos los índices de dos bosones idénticos la función de onda
mantiene su signo.
\end{displayquote}
Por tanto en el caso de los \textbf{bosones} (por ejemplo, los fotones)
la función de onda tendrá que ser
\textbf{simétrica}
\begin{equation}
    \psi_s(\mathbf{r}_1,\mathbf{r}_2)=\psi(\mathbf{r}_2,\mathbf{r}_1)
\end{equation}
Dado que los electrones son \textbf{fermiones}, el principio de
Pauli impone que la función de onda sea \textbf{antisimétrica}
\begin{equation}
    \psi(\mathbf{r}_1,\mathbf{r}_2)=-\psi(\mathbf{r}_2,\mathbf{r}_1)
\end{equation}

\subsection{Partículas no interaccionantes}
Para dos partículas que no interaccionan podemos escribir la
ecuación de Schrödinger como
\begin{equation}
    \hat{H}\psi=(\hat{H}_1 + \hat{H}_2)\psi=E\psi
\end{equation}
%El hamiltoniano es la suma de dos contribuciones, una para cada electrón
%\begin{equation}
%    \hat{H}=-\frac{\hbar^2}{2m_e}\nabla_1^2 -\frac{Ze^2}{4\pi \varepsilon_0r_1} 
%    -\frac{\hbar^2}{2m_e}\nabla_2^2 -\frac{Ze^2}{4\pi \varepsilon_0r_2}
%\label{eq:h1_h2}    
%\end{equation}
La función de onda correspondiente a la partícula 1 en el orbital
$\psi_\alpha$ y la partícula 2 en el orbital $\psi_\beta$ se puede
escribir como
\begin{equation}
    \psi(\mathbf{r}_1, \mathbf{r}_2) = \psi_\alpha (\mathbf{r}_1)\psi_\beta (\mathbf{r}_2)
\end{equation}
Aplicando el hamiltoniano sobre esta función de onda
obtendremos
\begin{align}
    &\hat{H}_1\psi_\alpha(\mathbf{r}_1)=E_1\psi_\alpha(\mathbf{r}_1)\\
    &\hat{H}_2\psi_\beta(\mathbf{r}_2)=E_2\psi_\beta(\mathbf{r}_2)\\
    &E = E_1+ E_2
\end{align}
Sin embargo, la función de onda $\psi=\psi_\alpha\psi_\beta$ 
no satisface el principio de indiscernibilidad. Mediante combinaciones
lineales se pueden obtener funciones simétricas, $\psi_s$,
y antisimétricas, $\psi_a$,
\begin{align}
    \psi_s(\mathbf{r}_1, \mathbf{r}_2) &= \frac{1}{2^{1/2}} \psi_\alpha(\mathbf{r}_1) \psi_\beta(\mathbf{r}_2) +  \psi_\alpha(\mathbf{r}_2) \psi_\beta(\mathbf{r}_1)\\
    \psi_a(\mathbf{r}_1, \mathbf{r}_2) &=\frac{1}{2^{1/2}}  \psi_\alpha(\mathbf{r}_1) \psi_\beta(\mathbf{r}_2) -  \psi_\alpha(\mathbf{r}_2) \psi_\beta(\mathbf{r}_1)
\end{align}
La función de onda antisimétrica se puede escribir también en forma
del denominado \textbf{determinante de Slater} (introducidos por
J. C. Slater en 1929)
\begin{equation}
    \psi_a(\mathbf{r}_1,\mathbf{r}_2) = \frac{1}{2^{1/2}}
    \begin{vmatrix} 
    \psi_\alpha(\mathbf{r}_1) & \psi_\beta(\mathbf{r}_1)   \\
    \psi_\alpha(\mathbf{r}_2) & \psi_\beta(\mathbf{r}_2)  \\
    \end{vmatrix}
\end{equation}
Estas expresiones se pueden generalizar para $N$ funciones de onda.
Por tanto para un sistema con $N$ bosones, la función de onda 
será
\begin{equation}
    \psi_s(\mathbf{r}_1,\mathbf{r}_2, ..., \mathbf{r}_N) = \frac{1}{(N!)^{1/2}}
    \sum\hat{P}\psi(\mathbf{r}_1,\mathbf{r}_2 ... \mathbf{r}_N) 
\end{equation}
mientras que para fermiones, la función de onda se escribirá como
el correspondiente determinante de Slater:
\begin{equation}
    \psi_a(\mathbf{r}_1,\mathbf{r}_2, ..., \mathbf{r}_N) = \frac{1}{(N!)^{1/2}}
    \begin{vmatrix} 
    \psi_\alpha(\mathbf{r}_1) & \psi_\beta(\mathbf{r}_1) & ... & \psi_\omega(\mathbf{r}_1)  \\
    \psi_\alpha(\mathbf{r}_2) & \psi_\beta(\mathbf{r}_2) & ... & \psi_\omega(\mathbf{r}_2)  \\
    \vdots & \vdots & \vdots& \vdots\\
    \psi_\alpha(\mathbf{r}_N) & \psi_\beta(\mathbf{r}_N) & ... & \psi_\omega(\mathbf{r}_N)  \\
    \end{vmatrix}
\end{equation}

En el caso de partículas no interaccionantes, se puede hablar no sólo del
estado total del sistema $\psi$, sino también de los estados unipartícula
(spin-orbitales). Es decir, podemos decir que una partícula se encuentra 
en el estado $\psi_\alpha$ y otra en $\psi_\beta$, (pero no podemos 
especificar cual de las $N$ partículas es la que ocupa cada estado).
En el caso de partículas interaccionantes, no podemos estrictamente hablar
de estados unipartícula, porque
\begin{equation}
    \psi_a(\mathbf{r}_1,\mathbf{r}_2, ..., \mathbf{r}_N) \neq \psi_\alpha(\mathbf{r}_1)\psi_\beta(\mathbf{r}_2)...\psi_\omega(\mathbf{r}_N)
\end{equation}
 Por eso la descripción del estado en términos de spin-orbitales es una aproximación: Aproximación Orbital.
 
\subsection{Partículas interaccionantes}
Supongamos que tenemos dos electrones que se encuentran en el mismo
orbital, $\psi_\alpha = \psi_\beta$. Si escribimos la función de onda total
antisimétrica $\psi_a$ veremos que es cero (dado que encontramos dos columnas 
idénticas en el determinante de Slater). Por tanto, debemos tener en consideración
la función de onda incluyendo el orbital $\psi(1)\psi(2)$ y 
también el \textbf{spin}, que puede ser $\alpha$
o $\beta$ para cada uno de los electrones. Para la contribución de spin
tenemos cuatro combinaciones diferentes posibles, correspondientes a 
los dos electrones con spin $\alpha$, los dos
con spin $\beta$ y uno con $\alpha$ y otro con $\beta$,
que al ser indistinguibles, escribimos en forma de combinaciones lineales
normalizadas
\begin{align}
\small
    &\alpha(1)\alpha(2) \\
    &\beta(1)\beta(2) \\
    &1/\sqrt{2}[\alpha(1)\beta(2)+\beta(1)\alpha(2)] = \sigma_{+}(1,2)\\
    &1/\sqrt{2}[\alpha(1)\beta(2)-\beta(1)\alpha(2)] = \sigma_{-}(1,2)
\end{align}
Entre estas combinaciones, sólo la función de onda correspondiente a la
última, $\psi(1)\psi(2)\sigma_{-}(1,2)$, es antisimétrica respecto al 
intercambio de índices, y por tanto apropiada para fermiones. 

En el caso de un átomo con dos electrones en un mismo orbital
$1s$, el determinante de Slater se escribirá como
\begin{equation}
\begin{split}
    \psi = 1s(1)1s(2)\frac{1}{2^{1/2}}[\alpha(1)\beta(2)-\beta(1)\alpha(2)] = \\
=    \frac{1}{2^{1/2}}
    \begin{vmatrix} 
    1s\alpha(1) & 1s\beta(1)   \\
    1s\alpha(2) & 1s\beta(2)  \\
    \end{vmatrix}
    \end{split}
\end{equation}
donde vemos que los dos electrones aparecen con espines opuestos. 
Para orbitales diferentes, el principio de exclusión de Pauli es 
irrelevante, aunque la función de onda global también debe ser
antisimétrica.

El estado de un electrón en un átomo
se define por medio de 4 números cuánticos: tres de ellos, $n$, $l$ y $m$, definen 
su estado orbital; el cuarto, $m_s$ ,su estado de espín.
El principio de exclusión de Pauli establece que: en un átomo polielectrónico
no puede haber 2 electrones con los 4 números cuánticos iguales.
En un mismo orbital sólo pueden estar localizados 2 electrones con diferente
estado de espín. El resultado de este principio es que en un orbital 
sólo quepan 2 electrones con espines opuestos o apareados ($\updownarrows$).

\section{Penetración y apantallamiento}
Al contrario de lo que veíamos en el caso de los átomos hidrogenoides,
los orbitales $2s$ y $2p$ ya no son degenerados. Un electrón en un 
átomo polielectrónico experimenta una repulsión electrostática 
ejercida por los demás electrones. Este efecto se puede incorporar
como una carga negativa neta localizada en el núcleo, que reduce la
carga nuclear desde $Z$ a $Z_{ef}$, la carga nuclear efectiva.
Hablamos entonces de un apantallamiento, y la constante de apantallamiento
$\sigma$ es 
\begin{equation}
    Z_{ef}= Z-\sigma
\end{equation}
Esta constante es diferente para los orbitales $s$ y $p$, dado
que los orbitales $s$ tienen mayor penetración que los orbitales
$p$ y por tanto experimentan menor apantallamiento. Por ello, 
se dice que los electrones en orbitales $s$ están más estrechamente
ligados.

Como resultado de la penetración y el apantallamiento de los orbitales,
para una misma capa (mismo número cuántico $n$) la energía de las 
subcapas (mismo $l$) se distribuye de acuerdo con
\begin{equation*}
    s<p<d<f
\end{equation*}
Así, en el caso del átomo de litio (Li) la configuración más
estable, el estado fundamental, será $1s^2s^1$, dado que los
orbitales $2s$ son más bajos en energía que los orbitales $2p$.
Los electrones más externos son los denominados \textbf{electrones 
de valencia}.

\subsection{El orden de ocupación}
Por extensión de este argumento, definimos el orden de ocupación,
o principio \textit{Aufbau}, que define el siguiente orden para el 
llenado de los orbitales:
\begin{equation*}
    1s < 2s < 2p < 3s<  3p <4s < 3d < 4p < 5s < 4d < 5p < 6s
\end{equation*}
donde cada orbital puede ser ocupado por dos $e^{-}$. 
Dada la multiplicidad de los orbitales $p$, para su llenado podrían
llenarse tanto con spin paralelo ($\upuparrows$) como con spin antiparalelo 
($\updownarrows$). Esta última configuración resulta más estable por estar
los electrones, en promedio, más alejados el uno del otro.
Así, por ejemplo para el carbono (C) la configuración de más baja energía 
es [He] $2s^2$ $2p_x^1$ $2p_y^1$. Esta misma norma se cumple 
siempre que hay orbitales degenerados disponibles para su ocupación,
lo cual resulta en la \textbf{regla de máxima multiplicidad de Hund},
que establece que:
\begin{displayquote}
Los electrones ocupan orbitales diferentes de una subcapa antes 
de ocupar doblemente un mismo orbital.
\end{displayquote}
La consecuencia de este principio, en el caso del nitrógeno (N) es
que su configuración fundamental sea [He] $2s^2$ $2p_x^1$ $2p_y^1$ $2p_z^1$
y para el oxígeno (O), [He] $2s^2$ $2p_x^2$ $2p_y^1$ $2p_z^1$.

La regla de máxima multiplicidad de Hund es consecuencia de una
propiedad mecanocuántica, llamada correlación de spin. Los electrones
con spin paralelo tienen tendencia a estar más separados y se repelen
menos que cuando tienen spin desapareado. Así, el átomo se comprime
ligeramente y aumenta la interacción de los electrones con el núcleo.
Es por este motivo que los electrones desapareados en los átomos de C,
N y O tienen el mismo spin.