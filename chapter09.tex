\chapter{Moléculas Diatómicas y Lineales}
En este tema y en el siguiente discutiremos los aspectos 
fundamentales de la estructura electrónica molecular. 
La principal novedad de las moléculas con respecto a 
los sistemas atómicos es que los electrones 
interaccionan con varios núcleos que, a su vez, 
se mueven. El estado del sistema depende así de las
variables de posición de los electrones y de los 
núcleos, lo que complica notablemente el problema. 
Tanto es así, que ni siquiera para la molécula más sencilla se puede resolver la ecuación de Schrödinger
de forma exacta. Por tanto el estudio de moléculas se
basa en la utilización de diversas aproximaciones:
\begin{itemize}
\item La aproximación de Born-Oppenheimer: establece
que el movimiento de núcleos y electrones se puede
separar en virtud de su gran diferencia de masa.
\item La teoría de enlace de Valencia: parte del 
concepto de un par de electrones compartidos.
\item La teoría de orbitales moleculares: se forman
a partir de combinaciones lineales de orbitales atómicos. De modo similar al caso atómico, un espin-orbital molecular es una función de onda de la molécula de un $e^{-}$.
\end{itemize}

\section{La aproximación de Born-Oppenheimer}
Mientras que la ecuación de Schrödinger para un átomo de hidrógeno
se puede resolver de forma exacta, no es posible obtener una solución
exacta para ninguna molécula, ya que incluso para la más sencilla de 
ellas hay que resolver un problema de tres cuerpos (dos núcleos y un
$e^{-}$). La primera aproximación que usamos es la de Born-Oppenheimer,
propuesta por Max Born y J. Robert Oppenheimer en 1927:
\begin{displayquote}
    Como los núcleos, mucho más pesados que los electrones, se desplazan
    a una velocidad relativamente más lenta, pueden considerarse fijos 
    en relación a los electrones que llevan a cabo sus desplazamientos.
\end{displayquote}
Podemos considerar los núcleos fijos en una ubicación arbitraria, y 
resolver así la ec. de Schrödinger para la función de onda de los $e^-$
únicamente. La aproximación de Born-Oppenheimer nos permite seleccionar 
una separación internuclear en una molécula diatómica y resolver luego 
la ec. de Schrödinger para los $e^-$ con esa separación nuclear.
Luego elegimos una separación diferente y repetimos el cálculo, y así
sucesivamente. De este modo podemos explorar cómo varía la energía de la
molécula con la longitud de enlace y obtener una curva de energía potencial 
molecular. Se llama curva de energía potencial porque la energía cinética
de los núcleos fijos es cero. Una vez calculada la curva se pueden
identificar, (i) la longitud de enlace de equilibrio, $R_e$, que es la 
separación internuclear en el punto mínimo de la curva, y (ii) la 
energía de disociación del enlace, $D_o$, que está relacionada con el 
valor mínimo, $D_e$.

Para una molécula con $N$ núcleos ($A=1,...,N$) y $n$ electrones ($i=1,...,n$)
podemos escribir el hamiltoniano con la siguiente expresión
\begin{equation}
 \begin{split}  
    V= -\frac{\hbar^2}{2}\sum_{A=1}^N\frac{\nabla^2_A}{M_A} 
    -\frac{\hbar^2}{2m_e}\sum_{i=1}^n
    \nabla^2_i 
    -\sum_{A=1}^N\sum_{i=1}^{n}\frac{Z_Ae^2}{4\pi \varepsilon_0r_{iA}} 
    +\\\sum_{i>j}^n \frac{e^2}{4\pi\varepsilon_0r_{ij}} 
    +\sum_{A>B}^N\frac{Z_AZ_Be^2}{4\pi \varepsilon_0R_{AB}} 
\end{split}
\end{equation}
donde estamos usando la distancia
entre electrón y núcleo, $r_{iA}=|\vec{r}_i-\vec{R}_A|$, la
distancia interelectrónica, $r_{ij}=|\vec{r}_i-\vec{r}_j|$ y la
distancia entre núcleos, $r_{AB}=|\vec{R}_B-\vec{R}_A|$.
Los estados estacionarios son solución de la
ecuación
\begin{equation}
    \hat{H}\Phi(\mathbf{r},\mathbf{R}) = \{
    \hat{K}_N + \hat{K}_e 
    +\hat{V}_{eN} 
    +\hat{V}_{ee}
    +\hat{V}_{NN} 
    \}\Phi(\mathbf{r},\mathbf{R}) = E\Phi(\mathbf{r},\mathbf{R})
\end{equation}
Dentro de esta expresión hay tres términos, 
\begin{equation}
    \hat{H}_e=\hat{K}_e 
    +\hat{V}_{eN} 
    +\hat{V}_{ee}
\end{equation}
que englobamos en el denominado hamiltoniano electrónico. En la
aproximación de Born-Oppenheimer:
debido a la gran diferencia de masa
entre electrones y núcleos ($me/MA << 10^{-3}$) se considera el movimiento 
de las nubes electrónicas 
independiente del movimiento de los 
núcleos. Así, podemos separar la
función de onda en una parte nuclear
y una parte electrónica
\begin{equation}
    \Phi=\Psi_e\Psi_N
\end{equation}
Al fijar los núcleos, podemos omitir
los términos que representa la energía
cinética nuclear de la ecuación
del hamiltoniano molecular
\begin{equation}
    (\hat{V}_{NN}+\hat{H}_{e})\Psi_e = 
    U\Psi_e
\end{equation}

De modo que la ecuación de Schrödinger para el movimiento nuclear es
\begin{align}
    \hat{H}_N\psi_N(\mathbf{R})=E\psi_N(\mathbf{R})\\
    [\hat{K}_N + U(\mathbf{R})]\psi_N(\mathbf{R}) = E\psi_N(\mathbf{R})
\end{align}
Así, con la aproximación de Born-Oppenheimer el cálculo de energía de un sistema molecular se puede separar en un problema electrónico y en un problema de movimiento nuclear.

\section{Teoría de enlace de valencia}
Esta teoría, desarrollada por Hetiler y London, fue la primera teoría mecanocuántica de enlace que se desarrolló. Incluye conceptos tales como el apareamiento de espin, la superposición de orbitales, los enlaces $\sigma$ y $\pi$ y la hibridación.
Según la teoría del enlace de valencia, el enlace se forma cuando un $e^-$ de un orbital atómico de un átomo aparea su espin con el de un $e^-$ en un orbital atómico de otro átomo.
Para comprender cómo se forma el enlace a partir de este apareamiento, vamos a analizar las funciones de onda de los dos electrones que lo forman.
Consideraremos el enlace químico más simple posible, el de la molécula de hidrógeno, H$_2$.

\subsection{Molécula de hidrógeno}
Tenemos dos núcleos y dos electrones: 
Orb. At. A y B → cuatro spin-orbitales: $1s_A\alpha$, $1s_A \beta$, $1s_B \alpha$, $1s_B \beta$.
Con 2 $e^-$ y 4 spin -orbitales  tenemos 6 estados bielectrónicos. Por un lado tenemos los estados iónicos en
los que los dos electrones están en los espinorbitales de uno u otro átomo:
\begin{equation*}
\begin{split}
    1s_A\alpha(1)1s_A\beta(2)\\
    1s_B\alpha(1)1s_B\beta(2)
\end{split}
\end{equation*}
Por otro lado tenemos los estados covalentes 
\begin{equation*}
\begin{split}
    1s_A\alpha(1)1s_B\alpha(2)\\
    1s_A\alpha(1)1s_B\beta(2)\\
    1s_A\beta(1)1s_B\alpha(2)\\
    1s_A\beta(1)1s_B\beta(2)\\
\end{split}
\end{equation*}
Heitler y London descartan los términos iónicos, que
corresponderían a
la configuración H$^+$-H$^-$ y sólo 
consideran los covalentes.