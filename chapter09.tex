\chapter{Moléculas Diatómicas y Lineales}
En este tema y en el siguiente discutiremos los aspectos 
fundamentales de la estructura electrónica molecular, primero
para las moléculas diatómicas y lineales y a continuación
para moléculas poliatómicas. La principal novedad de las 
moléculas con respecto a lo que ya conocemos acerca de los
átomos es que los electrones interaccionan simultáneamente
con varios núcleos que, a su vez, se encuentran en
movimiento. El estado del sistema dependerá así tanto 
de las posiciones de los electrones como de las de los 
núcleos, lo que complica notablemente el problema. 
Tanto es así, que ni siquiera para la molécula más 
sencilla podemos resolver la ecuación de Schrödinger
de forma exacta. 

Para solventar este problema, utilizamos diversas 
aproximaciones:
\begin{itemize}
\item La aproximación de Born-Oppenheimer, que establece
que el movimiento de núcleos y electrones se puede
separar en virtud de su gran diferencia de masa.
\item La teoría de enlace de Valencia, que parte del 
concepto de la compartición de pares de electrones.
\item La teoría de orbitales moleculares, que asume que
estos se forman a partir de combinaciones lineales de 
orbitales atómicos. 
\end{itemize}

\section{La aproximación de Born-Oppenheimer}
Mientras que la ecuación de Schrödinger para un átomo 
de hidrógeno se puede resolver de forma exacta, no es 
posible obtener una solución exacta para ninguna molécula,
ya que incluso para la más sencilla de ellas (el catión
H$_2^\+$ hay que resolver un problema de tres cuerpos 
(i.e. dos núcleos y un $e^{-}$). 

La primera aproximación que usamos es la de Born-Oppenheimer,
propuesta por Max Born y J. Robert Oppenheimer en 1927, en
la que se asume que se asume que los núcleos, al ser muchas
veces más pesados que los electrones, se mueven lentamente
y pueden tratarse como cuerpos estacionarios mientras los 
electrones se mueven bajo su influencia.
Esta aproximación es razonablemente buena para los estados 
fundamentales de las moléculas. 

Operativamente, usando esta aproximación podemos seleccionar
una separación internuclear y resolver la ecuación de
Schrödinger para los $e^-$, usando la distancia entre los
núcleos de manera paramétrica. A continuación elegimos 
una separación diferente y repetimos el cálculo, y así
sucesivamente. Esto permite explorar cómo varía la energía 
de la molécula en función de la distancia entre los átomos
y obtener una \textit{curva de energía potencial molecular}.
Se llama curva de energía potencial porque la energía cinética
de los núcleos fijos es cero. Una vez calculada la curva se
pueden identificar, (i) la longitud de enlace de equilibrio,
$R_e$, que es la separación internuclear en el punto mínimo de
la curva, y (ii) la energía de disociación del enlace, $D_o$,
que está relacionada con el valor mínimo, $D_e$.

\subsection{Ecuación de Schrödinger dentro de la 
aproximación de Born-Oppenheimer}
Para una molécula con $M$ núcleos ($A=1,...,M$) y 
$N$ electrones ($i=1,...,N$) podemos escribir el hamiltoniano
como una suma de términos correspondientes a la energía cinética
y potencial electrostática para núcleos y electrones
\begin{equation}
 \begin{split}  
\hat{H}=& -\frac{\hbar^2}{2}\sum_{A=1}^M\frac{\nabla^2_A}{M_A}
    -\frac{\hbar^2}{2m_e}\sum_{i=1}^N\nabla^2_i
    +\\
&+\sum_{A=1}^M\sum_{A>B}^M\frac{Z_AZ_Be^2}{4\pi\varepsilon_0R_{AB}}
+\sum_{i=1}^N\sum_{i>j}^N \frac{e^2}{4\pi\varepsilon_0r_{ij}}
-\sum_{A=1}^M\sum_{i=1}^{N}\frac{Z_Ae^2}{4\pi\varepsilon_0r_{iA}}
\end{split}
\end{equation}
En esta ecuación, $r_{iA}=|\vec{r}_i-\vec{R}_A|$ es la distancia
entre el electrón $i$ y el núcleo $A$, $r_{ij}=|\vec{r}_i-\vec{r}_j|$ 
es la distancia interelectrónica, y $r_{AB}=|\vec{R}_B-\vec{R}_A|$
es la distancia entre núcleos.

Considerando la aproximación de Born-Oppenheimer podemos fijar
los núcleos en una posición, con lo cual la energía cinética 
asociada a los núcleos es cero y la repulsión entre ellos es
una constante que no tiene efecto sobre las funciones propias. 
Los términos restantes forman el hamiltoniano electrónico
\begin{equation}
    \hat{H}_\mathrm{elec}= -\frac{\hbar^2}{2m_e}\sum_{i=1}^N\nabla^2_i 
    +\sum_{i=1}^N\sum_{i>j}^N \frac{e^2}{4\pi\varepsilon_0r_{ij}}
    -\sum_{A=1}^M\sum_{i=1}^{N}\frac{Z_Ae^2}{4\pi\varepsilon_0r_{iA}} 
\end{equation}
La solución de la ecuación de Schrödinger para el hamiltoniano
electrónico 
\begin{equation}
\hat{H}_\mathrm{elec}\Phi_\mathrm{elec} = E_\mathrm{elec}\Phi_\mathrm{elec}
\end{equation}
son las funciones de onda electrónicas
\begin{equation}
    \Phi_\mathrm{elec}=\Phi_\mathrm{elec}({\mathbf{r}_i}; {\mathbf{R_A}})
\end{equation}
que dependen explicitamente de las coordenadas electrónicas y
dependen de manera paramétrica de las coordenadas nucleares.
Para diferentes posiciones de los núcleos $\Phi_\mathrm{elec}$
es una función diferente. Para obtener la energía total para
una separación internuclear dada, tenemos que considerar
la constante que obtenemos a partir de la repulsión entre sus
cargas positivas
\begin{equation}
    E_\mathrm{tot}= E_\mathrm{elec} + \sum_{A=1}^M\sum_{A>B}^M\frac{Z_AZ_Be^2}{4\pi\varepsilon_0R_{AB}}
\end{equation}

Una vez resuelto el problema electrónico es posible resolver 
el problema del movimiento nuclear. Como los electrones se 
mueven mucho más rápido que los núcleos, es razonable
remplazar las coordenadas electrónicas por sus posiciones 
promedio, descritas por la función de onda electrónica. Esto
nos permite obtener el hamiltoniano nuclear, que expresa la 
energía de los núcleos ante el campo promedio generado por los
electrones
\begin{equation}
 \begin{split}  
    \hat{H}_\mathrm{nucl}=&-\frac{\hbar^2}{2}\sum_{A=1}^M\frac{\nabla^2_A}{M_A}
    +\sum_{A=1}^M\sum_{A>B}^M\frac{Z_AZ_Be^2}{4\pi\varepsilon_0R_{AB}} 
    +
    \\
    &+\Bigg\langle-\frac{\hbar^2}{2m_e}\sum_{i=1}^N\nabla^2_i 
    -\sum_{A=1}^M\sum_{i=1}^{N}\frac{Z_Ae^2}{4\pi\varepsilon_0r_{iA}} 
    +\sum_{i=1}^N\sum_{i>j}^N \frac{e^2}{4\pi\varepsilon_0r_{ij}} \Bigg\rangle =
    \\
    =&-\frac{\hbar^2}{2}\sum_{A=1}^M\frac{\nabla^2_A}{M_A} + \sum_{A=1}^M\sum_{A>B}^M\frac{Z_AZ_Be^2}{4\pi\varepsilon_0R_{AB}} + E_\mathrm{elec}(\{ \mathbf{R}_A\})
    \\
    =&-\frac{\hbar^2}{2}\sum_{A=1}^M\frac{\nabla^2_A}{M_A} + E_\mathrm{tot}(\{ \mathbf{R}_A\})
\end{split}
\end{equation}
En esta ecuación hemos sustituido en primer lugar
la energía electrónica, y en el siguiente paso
la energía total, que representa un potencial 
para el movimiento nuclear, es decir una 
curva de energía potencial. Las soluciones
para esta expresión son las funciones nucleares de
la ecuación de Schrödinger
\begin{equation}
    \hat{H}_\mathrm{nucl}\Phi_\mathrm{nucl} = E\Phi_\mathrm{nucl}
\end{equation}
describen la vibración, rotación y traslación
de una molécula y su valor propio, $E$,
representa la energía incluyendo todas  
las contribuciones para la energía: 
electrónica, vibracional, rotacional y
translacional. La función de onda total
dentro de esta aproximación es
\begin{equation}
    \Phi=\Phi_\mathrm{elec}\Phi_\mathrm{nucl}
\end{equation}

%\begin{equation}
%    \hat{H}\Phi(\mathbf{r},\mathbf{R}) = \{
%    \hat{K}_N + \hat{K}_e 
%    +\hat{V}_{eN} 
%    +\hat{V}_{ee}
%    +\hat{V}_{NN} 
%    \}\Phi(\mathbf{r},\mathbf{R}) = E\Phi(\mathbf{r},\mathbf{R})
%\end{equation}
%Dentro de esta expresión hay tres términos, 
%\begin{equation}
%    \hat{H}_e=\hat{K}_e 
%    +\hat{V}_{eN} 
%    +\hat{V}_{ee}
%\end{equation}
%que englobamos en el denominado hamiltoniano electrónico. En la
%aproximación de Born-Oppenheimer:
%debido a la gran diferencia de masa
%entre electrones y núcleos ($me/MA << 10^{-3}$) se considera el movimiento 
%de las nubes electrónicas 
%independiente del movimiento de los 
%núcleos. Así, podemos separar la
%función de onda en una parte nuclear
%y una parte electrónica
%\begin{equation}
%    \Phi=\Psi_e\Psi_N
%\end{equation}
%Al fijar los núcleos, podemos omitir
%los términos que representa la energía
%cinética nuclear de la ecuación
%del hamiltoniano molecular
%\begin{equation}
%    (\hat{V}_{NN}+\hat{H}_{e})\Psi_e = 
%    U\Psi_e
%\end{equation}
%
%De modo que la ecuación de Schrödinger para el movimiento nuclear es
%\begin{align}
%    \hat{H}_N\psi_N(\mathbf{R})=E\psi_N(\mathbf{R})\\
%    [\hat{K}_N + U(\mathbf{R})]\psi_N(\mathbf{R}) = E\psi_N(\mathbf{R})
%\end{align}
%Así, con la aproximación de Born-Oppenheimer el cálculo de energía de un sistema molecular se puede separar en un problema electrónico y en un problema de movimiento nuclear.

\section{Teoría de enlace de valencia}
Esta teoría, desarrollada por Heitler y London, fue la primera
teoría mecanocuántica de enlace que se desarrolló. Incluye 
conceptos tales como el \textit{apareamiento de espín}, 
la \textit{superposición de orbitales}, 
\textit{los enlaces $\sigma$ y $\pi$} y la
\textit{hibridación}.
Según la teoría del enlace de valencia, el enlace se forma 
cuando un electrón de un orbital atómico de un átomo aparea su 
espín con el de un electrón en un orbital atómico de otro átomo.
Para comprender cómo se forma el enlace a partir de este
apareamiento, vamos a analizar las funciones de onda de los 
dos electrones que lo forman. Consideraremos el caso más
sencillo, correspondiente al enlace químico 
en la molécula de hidrógeno, H$_2$.

\subsection{Molécula de hidrógeno}
Partimos de dos orbitales atómicos 1s que llamamos A y B.
En total, forman cuatro espín-orbitales: $1s_A\alpha$, 
$1s_A \beta$, $1s_B\alpha$ y $1s_B \beta$. 
Con 2 $e^-$ y 4 espín-orbitales tenemos un total de 6 estados
bielectrónicos. Por un lado tenemos los \textit{estados 
iónicos} en los que los dos electrones  están en los
espín-orbitales de uno u otro átomo, 
1s$_A\alpha$(1)1s$_A\beta$(2),
y 1s$_B\alpha$(1)1$s_B\beta$(2), 
y por otro, los \textit{estados covalentes}:
1s$_A\alpha$(1)1s$_B\alpha$(2), 1s$_A\alpha$(1)1s$_B\beta$(2), 
1s$_A\beta$(1)1s$_B\alpha$(2) y 1s$_A\beta$(1)1s$_B\beta$(2).
Heitler y London descartan los términos iónicos, que 
corresponderían a la configuración H$^+$-H$^-$ y sólo 
consideran los covalentes.

Como los dos electrones son indiscernibles, una función de onda
aceptable para el sistema debe tener densidades de probabilidad 
idénticas para los dos electrones. Una función del tipo
\begin{equation}
    \psi=\mathrm{1s}_A(1)\mathrm{1s}_B(2)
\end{equation}
no cumple el criterio de indiscernibilidad, al estar la densidad
del electrón 1 situada en el orbital $\psi_A$ y la del electrón
2 en el orbital $\psi_\mathrm{B}$. Sin embargo, una
superposición (no normalizada) de las dos funciones 
\begin{equation}
    \psi_\pm=\mathrm{1s}_A(1)\mathrm{1s}_B(2) \pm
    \mathrm{1s}_A(2)\mathrm{1s}_B(1)
\end{equation}
sí permite describir el estado del sistema satisfaciendo la 
condición de indiscernibilidad. 

El Principio de Pauli 
requiere que la función de onda completa de dos electrones, 
incluyendo el espín, cambie de signo cuando se intercambian 
los electrones. La contribución de espín puede expresarse
como una serie de combinaciones 
\begin{gather*}
\alpha(1)\alpha(2)\\
\beta (1)\beta (2)\\
\alpha(1)\beta(2) + \alpha(2)\beta(1)\\
\alpha(1)\beta(2) − \alpha(2)\beta(1)
\end{gather*}
Sólo una de estas combinaciones es antisimétrica, por lo cual
irá asociada a la función de onda orbital simétrica
\begin{equation}
    \Psi=N
    \big[\mathrm{1s}_A(1)\mathrm{1s}_B(2) +\mathrm{1s}_A(2)\mathrm{1s}_B(1)\big]
    \frac{1}{\sqrt{2}}\big[\alpha(1)\beta(2) − \alpha(2)\beta(1)\big]
    \label{eq:unpaired}
\end{equation}
mientras que las combinaciones simétricas podrán ir asociadas
a la forma antisimétrica
\begin{equation}
    \Psi=N
    \big[\mathrm{1s}_A(1)\mathrm{1s}_B(2) -\mathrm{1s}_A(2)\mathrm{1s}_B(1)\big]
\begin{cases}
\alpha(1)\alpha(2)\\
\beta (1)\beta (2)\\ 
\frac{1}{\sqrt{2}}\big[\alpha(1)\beta(2) + \alpha(2)\beta(1)\big]
\end{cases}
\end{equation}

En la primera de las funciones (Eq. \ref{eq:unpaired})
nos encontramos con electrones con los espines apareados.
Esta combinación es el estado de menor energía
correspondiente a la formación de enlace químico,
dado que hay una elevada densidad de 
probabilidad para la presencia del electrón entre ambos
núcleos por la interferencia constructiva entre las
contribuciones de los orbitales. El enlace es de tipo 
$\sigma$ y tiene simetría cilíndrica alrededor del eje
internuclear. Se asemeja a un par de electrones en 
un orbital tipo s.

\subsection{Otras moléculas diatómicas}
Podemos aplicar estos conceptos a moléculas diatómicas
homonucleares, como la molécula de N$_2$. Tomamos en 
cuenta la configuración electrónica de valencia
2s$^2$2p$_x^1$p$_y^1$p$_z^1$
Por convención, se toma como eje internuclear el eje $z$ ,
y así cada átomo tiene un orbital 2p$_z$ que apunta 
hacia un orbital 2p$_z$ del otro átomo, mientras que 
los orbitales 2p$_x$ y 2p$_y$ son perpendiculares al eje. 
Por apareamiento de los dos electrones de los dos orbitales 
2p$_z$ se forma un enlace $\sigma$. Podemos escribir la
parte orbital de la función de onda como
\begin{equation}
 \psi =2\mathrm{p}_{zA}(1)2\mathrm{p}_{zB}(2) + 2\mathrm{p}_{zA}(2)2\mathrm{p}_{zB}(1)
\end{equation} 

Los restantes orbitales 2p de los átomos de N no pueden
formar enlaces $\sigma$ al no poseer simetría
cilíndrica alrededor del eje internuclear. En cambio se 
unen para forman dos enlaces $\pi$. El enlace $\pi$ se
forma a partir del apareamiento de los espines de 
electrones en dos orbitales p contiguos.
La molécula de N$_2$ posee dos enlaces $\pi$, uno formado
por el apareamiento de espines de dos orbitales 2p$_x$
contiguos y el otro por el apareamiento de espines en
dos orbitales 2p$_y$ contiguos. El apareamiento de
electrones lleva a un orden de enlace 3, correspondiente 
a un enlace $\sigma$ y dos enlaces $\pi$.

%Para otras moléculas poliatómicas, se mantiene la
%misma lógica en virtud de la cual se forman enlaces
%$\sigma$ a partir del apareamiento de electrones 
%que ocupan orbitales atómicos con simetría cilíndrica
%y enlaces $\pi$ cuando se emparejan electrones con la
%simetría adecuada. Sin embargo, las predicciones de 
%la teoría empiezan a aparecer al no ser capaz de
%predecir por ejemplo el ángulo de la molécula de 
%agua o la tetravalencia del carbono. Para explicarla, 
%se introducen conceptos como la promoción de electrones
%y la hibridación.

\section{Teoría de Orbitales Moleculares}
Según esta teoría no tratamos a los electrones como
parte de un enlace en particular, sino que se distribuyen
por toda la molécula. Esta teoría está mucho más desarrollada
que la teoría de enlace de valencia. Para ilustrarla,
utilizamos la especie molecular más simple de todas, el 
ion molecular de hidrógeno, H$^+_2$, que cuenta con un sólo
electrón y dos núcleos idénticos. 

El hamiltoniano del único electrón en el H$^+_2$
\begin{equation}
\hat{H}=-\frac{\hbar^2}{2m_e}\nabla^2_1-
\frac{e^2}{4\pi \varepsilon_0r_{1A}} - 
\frac{e^2}{4\pi \varepsilon_0r_{1B}} +
\frac{e^2}{4\pi \varepsilon_0R_{AB}}
\end{equation}
donde el primer término corresponde a la energía cinética
del electrón, el segundo y tercero, a su interacción 
con los núcleos, y el cuarto a la repulsión internuclear.
Las funciones propias de esta ecuación son los denominados
\textit{orbitales moleculares}, a partir de cuyo cuadrado
$|\Psi|^2$ podemos entender la distribución del electrón
en la molécula. 

\subsection{Combinaciones lineales de orbitales atómicos}
En el caso del H$^+$ el problema se puede
resolver analíticamente dentro de la aproximación de 
Born-Oppenheimer, pero las soluciones son complicadas.
Por tanto, recurrimos a una aproximación más simple que
se puede emplear también en el caso de sistemas más 
complejos. Estas soluciones se obtienen como combinación
lineal de los orbitales atómicos de la molécula 
(\textit{Linear Combination of Atomic Orbitals}, LCAO)
\begin{equation}
\Psi_i=\sum_jc_{ij}\phi_j
\end{equation}
donde $\phi_j$ es cada uno de los orbitales atómicos y 
$c_{ij}$ es un coeficiente con la contribución de cada
orbital atómico $j$ en el orbital molecular $\Psi_i$.

En el caso del ion molecular H$_2^+$ comenzamos con sendos 
orbitales atómicos 1s centrados cada uno en un núcleo
\begin{gather*}
    \chi_A =1s_A\\
    \chi_B=1s_B
\end{gather*}
Escribimos la función de onda total como una
superposición de los dos orbitales atómicos, sumando o 
restando los orbitales atómicos para producir orbitales
moleculares de la simetría de inversión apropiada.
\begin{equation}
\Psi_\pm = %N(a\pm b)
\begin{cases}
\sigma_g=\{2(1+S)\}^{-1/2}(1s_A+1s_B)\\
\sigma_u=\{2(1-S)\}^{-1/2}(1s_A-1s_B)
\end{cases}
\end{equation}
donde hemos usado el hecho de que los orbitales atómicos
están normalizados. Definiendo  $S=\int\chi_A^\star\chi_Bdq=\int\chi_B^\star\chi_Adq$, como
la integral de solapamiento, obtenemos los correspondientes
orbitales moleculares, que son ortonormales entre sí. 

\subsection{Orbitales enlazantes y antienlazantes}
A partir de estas funciones de onda, obtenemos expresiones 
para la densidad
\begin{align}
    \Psi^2_+ &= N^2(1s^2_A+1s^2_B+21s_A1s_B)\\
    \Psi^2_- &= N^2(1s^2_A+1s^2_B-21s_A1s_B) 
    \label{eq:ungerade}
\end{align}
La primera de estas expresiones resulta en una acumulación
de la densidad electrónica en la región internuclear. Se
trata por tanto de un orbital molecular \textit{enlazante}.
Por el contrario, en el caso de la Ecuación \ref{eq:ungerade} 
la densidad se acumula fuera de la región internuclear. Se
trata de un \textit{orbital antienlazante}. 

La acumulación de densidad electrónica en distintas regiones 
hace que en los orbitales enlazantes y antienlazantes, los 
núcleos se vean atraídos hacia el interior o el exterior del
enlace, respectivamente. Allá donde los orbitales atómicos
solapan e interfieren constructivamente, la energía es inferior
a la de los dos átomos por separado dado que el electrón
puede interaccionar simultáneamente con dos núcleos.
Para separaciones pequeñas el
espacio  entre los núcleos es demasiado pequeño y no se 
acumula densidad electrónica significativa. Además la 
repulsión entre núcleos (que es proporcional a $1/R$) aumenta.
Como consecuencia, la energía de la molécula aumenta a 
distancias cortas y la curva de energía potencial pasa por 
un valor mínimo. 

La energía del orbital 1$\sigma$ es 
\begin{equation}
    E_{1\sigma}=E_{H1s} + \frac{e^2}{4\pi\varepsilon_0R}
    -\frac{j+k}{1+S}
\end{equation}
donde $S$ es la integral de solapamiento, $j$ mide
la interacción entre un núcleo y la densidad 
electrónica centrada en el otro núcleo y $k$
mide la interacción entre el núcleo y el exceso
de probabilidad en la región internuclear
que emerge del solapamiento. 

La diferencia entre la energía a la distancia
internuclear de equilibrio, ($R = R_e$), y la 
de las especies atómicas infinitamente separadas,
($R=\infty$), se denomina energía de disociación
de equilibrio y se denota $D_e$. El valor de $D_e$
nos da una idea de la estabilidad de la molécula, 
ya que, cuanto mayor es $D_e$, más difícil será 
romper la molécula en dos átomos separados. Aunque 
la energía calculada a partir de esta expresión de
acuerdo con la Teoría de Orbitales Moleculares no 
es particularmente precisa, es cualitativamente 
correcta.

El orbital molecular enlazante resultante de la 
interferencia constructiva es denominado 1$\sigma$,
pues es el orbital $\sigma$ de menor energía. Un 
electrón que ocupa un orbital $\sigma$ se denomina 
electrón $\sigma$, y si se trata del único electrón 
presente en la molécula (como en el estado fundamental
del H$^+$), indicamos la configuración de la molécula 
como 1$\sigma^1$. 

El orbital antienlazante tiene un plano nodal 
internuclear donde se anulan los orbitales atómicos
1s$_A$ y 1s$_B$ de forma exacta. Este orbital si 
es ocupado, contribuye a la disminución de la 
cohesión entre dos átomos y a la elevación de 
la energía de la molécula en relación con los 
átomos separados. Específicamente, la expresión para
la energía de este orbital 2$\sigma$ es
\begin{equation}
    E_{2\sigma}=E_{H1s} + \frac{e^2}{4\pi\varepsilon_0R}
    -\frac{j-k}{1-S}
\end{equation}
En la superficie de energía potencial se ve el efecto
desestabilizador de un electrón antienlazante. Este efecto
es debido en parte a la exclusión de un electrón 
antienlazante de la región internuclear, que se
distribuye por fuera de la región enlazante. 

Un efecto importante que también observamos en
la superficie de energía para los orbitales $1\sigma$
y $2\sigma$ es que el orbital antienlazante es más
antienlazante que el enlazante es enlazante, lo
cual se traduce en la expresión
\begin{equation}
    |E_{-}-E_{H1s}|>|E_+-E_{H1s}|
\end{equation}
Esto se debe en parte a la repulsión entre los 
núcleos, que eleva la energía de ambos orbitales 
moleculares. Los orbitales antienlazantes que hemos
denominado $2\sigma$ también se pueden indicar con 
un asterisco, y nombrarse 2$\sigma^\star$. 

Para moléculas diatómicas homonucleares se 
describe el orbital molecular identificando 
su simetría de inversión: el comportamiento de 
la función de onda al ser invertida en el centro 
de inversión de la molécula. Si consideramos cualquier
punto del orbital enlazante $\sigma$ y lo proyectamos
a través del centro de la molécula y a igual distancia
del otro lado, obtenemos un valor idéntico de la función
de onda. Tiene, por tanto, simetría par y o, en alemán, 
\textit{gerade}. Esta simetría se indica con un subíndice $g$,
y el orbital como $\sigma$ pasa a ser $\sigma_g$. 
Análogamente, el orbital antienlazante $2\sigma$ tiene
simetría impar o \textit{ungerade}, que  se indica con el 
subíndice $u$, con lo que el orbital $\sigma^\star$ pasa
a ser $\sigma_u$. Los orbitales que hasta ahora hemos
denominado 1$\sigma$ y 2$\sigma$ pueden también 
describirse como 1$\sigma_g$ y 1$\sigma_u$.

\subsection{Moléculas diatómicas homonucleares}
 Los orbitales moleculares se construyen combinando los orbitales
 atómicos disponibles. Los electrones proporcionados por los
 átomos se acomodan en los orbitales en la configuración que
 minimice su energía siempre y cuando esta esté sujeta a las restricciones del Principio de exclusión de Pauli, que no más
 de un electrón puede ocupar un único orbital y que cuando dos 
 lo ocupen deben tener sus espines apareados.
 Al igual que para los átomos, si hay varios OM degenerados
 disponibles, se agrega un electrón por cada orbital antes de
 ocupar doblemente un orbital, para así minimizar la repulsión 
 interelectrónica. También se tiene en cuenta la regla de máxima
 multiplicidad de Hund, en virtud de la cual si los electrones 
 ocupan distintos orbitales degenerados, se disponen 
 preferentemente con espines paralelos.

Consideremos en primer lugar la molécula H$_2$, que es la más
simple entre las moléculas diatómicas. En ella, cada átomo de
hidrógeno contribuye un orbital 1s, igual que hemos visto para
el ión H$_2^+$, y forman dos orbitales moleculares, el 
1$\sigma_g$ y el 1$\sigma_u$. En general, a partir de $N$ 
orbitales atómicos se forman $N$ orgitales moleculares.

El \textit{diagrama de orbitales moleculares} 
permite visualizar el tipo de configuración electrónica que se
va a adoptar. Como en el caso del H$_2$ hay dos electrones
que acomodar, ambos se alojan en el orbital molecular 1$\sigma_g$
apareando sus espines, como requiere el principio de exclusión
de Pauli. La configuración resultante es 1$\sigma^2$, en la
que el enlace químico se forma a partir de dos electrones en el
orbital molecular $\sigma$ enlazante. La función de onda de 
este estado fundamental escrita como un determinante es
\begin{equation}
   \Phi_+ =
   \frac{1}{\sqrt{2}}
   \begin{vmatrix}
\sigma_g\alpha(1) & \sigma_g\alpha(2) \\ 
\sigma_g\beta(1) & \sigma_g\beta(2) 
\end{vmatrix}=
\frac{1}{\sqrt{2}}\sigma_g(1)\sigma_g(2)\big[\alpha(1)\beta(2)-\alpha(2)\beta(1)\big]
\end{equation}

Los resultados que se obtienen a partir de esta aproximación 
son buenos en la zona correspondiente al mínimo en la superficie
de energía, pero no
en lo que respecta a la energía de disociación, al contrario
de lo que sucedía con la teoría de enlace de valencia.
Un método que mejora los problemas derivados de ambas 
aproximaciones es el de \textit{interacción de configuraciones (CI)}.

Otro caso interesante es el de la molécula de helio (He$_2$). 
En este caso, tenemos dos electrones apareados tanto en
el orbital $1\sigma_g$ como en el $1\sigma_u$. La configuración
resultante es $1\sigma_g^2$ y $1\sigma_u^2$. Dado que la 
desestabilización del antienlazante es superior a la
estabilización del antienlazante, la molécula de He$_2$ 
tiene mayor energía que los átomos por separado y es por
tanto inestable.

\subsection{Moléculas homonucleares del segundo periodo}
Cuando llegamos a las moléculas que implican a átomos del
segundo periodo, sólo participan en el enlace los orbitales
atómicos de la capa de valencia, es decir los 2s y 2p.
Contribuyen a un orbital molecular los orbitales atómicos
que tengan la simetría adecuada. Por ejemplo, para formar
un orbital molecular $\sigma$ intervienen los orbitales
atómicos que tienen simetría cilíndrica alrededor del eje
internuclear, es decir los orbitales 2s y los 2p$_z$ de 
los dos átomos, para formar cuatro orbitales moleculares.

Para formar orbitales $\pi$ enlazantes o antienlazantes, 
de nuevo, se superponen orbitales que tienen la simetría
adecuada. En concreto, los dos orbitales 2p$_x$ se 
superponen dando dos orbitales $\pi_x$ y los dos orbitales 
2p$_y$ se superponen dando dos orbitales $\pi_y$ . Los dos
orbitales enlazantes son $\pi_u$ mientras que los
antienlazantes son $\pi_g^\star$.
Cuando un orbital s se superpone con un orbital p$_x$ de otro átomo, no existe solapamiento neto entre ambos orbitales.

Para construir el diagrama de niveles de energía de moléculas diatómicas homonucleares del segundo periodo, formamos ocho 
orbitales moleculares a partir de los ocho orbitales de la 
capa de valencia (cuatro de cada átomo).

\subsection{Moléculas diatómicas heteronucleares}
Cuando dos átomos diferentes forman enlaces, nos vamos
a encontrar con el fenómeno de la polaridad. 
Un enlace polar entre dos átomos A y B consiste en dos 
electrones en un orbital de la forma
\begin{equation}
    \Psi=c_A\psi_A + c_B\psi_B\label{eq:polar}
\end{equation}
donde $c_A\neq c_B$. Esto supone que el orbital atómico
de menor energía tiene la mayor contribución al orbital 
enlazante, así los electrones no están igualmente repartidos
entre el núcleo A y B. Esto resulta en que los átomos 
tengan cargas parciales negativas y positivas.

Otra consecuencia de la diferencia entre los orbitales 
implicados es que no existe el centro de inversión, y 
por tanto, desaparecen las simetrías \textit{gerade} y 
\textit{ungerade}. Al entrar en juego átomos 
cuya naturaleza puede ser muy distinta, tendremos que 
utilizar consideraciones energéticas que nos permitan 
inferir qué orbitales participarán en el enlace y de qué 
forma se van a combinar. Sólo orbitales atómicos de 
energías razonablemente similares contribuyen sustancialmente
a un orbital molecular dado.

Consideremos el ejemplo de la molécula de HF. Para construir 
los orbitales moleculares, se parte del orbital 1s del hidrógeno
y los orbitales 1s, 2s y 2p del flúor. Los orbitales 1s y 2s
del átomo de F tienen una energía demasiado baja para participar
en el enlace y forman así orbitales moleculares $\sigma$
no enlazantes. Asimismo, se forma una combinación de los 
orbitales atómicos 1s del H y 2p$_z$ del F, que resultan en
un orbital molecular enlazante 3$\sigma$ y otro antienlazante, 
4$\sigma$, que queda vacante en el estado fundamental. 
Recuperando la descripción de polaridad de la Ecuación
\ref{eq:polar}, en este caso el orbital se forma siguiendo
\begin{equation}
    \sigma=c_11s_H + c_22p_{z,F}
\end{equation}
y $c_2>c_1$. Finalmente, los orbitales de simetría $\pi$, 
son exclusivamente los orbitales 2p$_x$ y 2p$_y$. Estos 
orbitales no pueden mezclarse con el orbital 1s$_H$ y
quedan como orbitales 1$\pi_x$ y 1$\pi_y$ no enlazantes degenerados. 

En muchos otros casos de moléculas diatómicas nos
encontramos con que ambos átomos disponen de orbitales
atómicos s y p. Éste es el caso de moléculas como el 
CN, NO, CO o FCl. Para estas moléculas, las energías 
de los orbitales de valencia en uno y otro átomo no son muy diferentes. Así, cabe esperar que los orbitales s de uno de
los átomos se combinen con los orbitales s del otro átomo, 
y que los orbitales p hagan lo propio, al igual que sucedía
en el caso de moléculas homonucleares. La diferencia con 
el caso homonuclear es que los coeficientes de la mezcla 
no satisfacen exactamente la relación $c_A = \pm c_B$.
Un ejemplo interesante es el de la molécula de CO, que 
tiene hasta 10 electrones de valencia. Su configuración
electrónica es 
$(\sigma_s)^2(\sigma_s^{\star})^2(\pi)^4(\sigma_p)^2$. 
Como sus seis electrones enlazantes la molécula tiene un
enlace triple similar al del N$_2$.