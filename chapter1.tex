\chapter{Antecedentes de la Mecánica Cuántica}
A finales del s. XIX se producen una serie de cambios muy 
importantes  en la Física debido, en primer término, a una
serie de observaciones experimentales que la Física Clásica
no era capaz de explicar.

\section{Radiación del Cuerpo Negro}
La primera de estas limitaciones parte de un fenómeno que 
podemos observar cotidianamente. Cuando calentamos un metal 
a muy elevadas temperaturas vemos que empieza a cambiar de 
color, adquiriendo tonalidades primero rojizas y luego azuladas.
Esto se debe a que  el cuerpo calentado emite una radiación
electromagnética. No todas las frecuencias están igualmente 
presentes en la radiación. Si estudiamos cómo cambia la distribución
de longitudes de onda, $\lambda$,  a medida que va aumentando
la temperatura, veremos que el máximo de la distribución se 
desplaza hacia longitudes de onda más bajas. Esta observación la
captura la \textbf{ley de desplazamiento de Wien},
\begin{equation}
\lambda_\mathrm{max}T=\mathrm{constante}.
\label{eq:Wien}
\end{equation}

Definimos el \textbf{cuerpo negro} como un objeto que es capaz de 
absorber todas las longitudes de onda uniformemente. Podemos pensar
en él como un dispositivo experimental con una cavidad reflectante
en el interior, que se prepara en equilibrio a una temperatura $T$,
y en el que hay apenas un pequeño orificio, que nos permite medir 
la radiación. \textbf{Lord Rayleigh} trató este problema de manera
clásica, asumiendo que el campo electromagnético se podía describir
como una colección de osciladores con todas las frecuencias o
longitudes de onda posibles. Si había una radiación de una longitud
de onda determinada, esta radiación se correspondería con un oscilador
excitado. A partir de una descripción que asumía que la temperatura
distribuiría su energía de acuerdo con el \textbf{principio de
equipartición de la energía}, llegó con la ayuda de \textbf{James
Jeans} a la siguiente relación:
\begin{equation}
\rho=\frac{8\pi k_BT}{\lambda^4}
\label{eq:rayleigh-jeans}
\end{equation}
Esta ecuación, conocida como la \textbf{ecuación de Rayleigh-Jeans},
expresa la contribución a la energía de la radiación para cada una de las 
longitudes de onda $\lambda$, o la densidad de estados, $\rho$. 
Como vemos la dependencia con  $\lambda$ es inversa con la
cuarta potencia.  A valores elevados de $\lambda$, la predicción
es acertada, pero a valores bajos, correspondiente a radiación
de muy alta energía, la densidad aumentaría monotónicamente. 
Esto  conduciría a la denominada ``catástrofe 
ultravioleta'', porque de acuerdo con este crecimiento monotónico
la densidad nunca dejaría de aumentar. Este resultado revelaba un
fallo en la descripción clásica del sistema.

\textbf{Max Planck} pudo resolver este problema asumiendo que la
energía
de cada oscilador estaba limitada a una serie de valores discretos,
es decir, estaba cuantizada. La energía de cada oscilador de 
frecuencia $\nu$ debía ser un múltiplo entero de la cantidad
$h\nu$. Es decir, 
\begin{equation}
E=nh\nu
\end{equation}
donde $n=1,2,3...$. De esta manera la expresión resultante
para la densidad de estados era la denominada \textbf{distribución
de Planck},
\begin{equation}
    \rho=\frac{8\pi h c}{\lambda^5(\mathrm{e}^{hc/\lambda k_BT}-1)}\
\label{eq:planck}
\end{equation}
En la teoría de Planck, la constante $h$ era indeterminada,
pero ajustando a los datos experimentales se pudo recuperar 
su valor de $h=6.626\times 10^{-34}$J$\cdot$s. De acuerdo con
esta nueva interpretación ``cuántica'', los osciladores se 
excitan sólo cuando hay suficiente energía disponible para 
alcanzar la energía $h\nu$.

\section{El Efecto Fotoeléctrico}
La segunda evidencia experimental que iba más allá de lo que la 
Física clásica era capaz de predecir es el denominado efecto
fotoeléctrico. El efecto fotoeléctrico indica que los fotones
se comportan como partículas que son capaces de colisionar
con otras partículas. 
En este caso, el experimento consistía en 
exponer a un material determinado a una radiación ultravioleta.
Hasta alcanzar una determinada frecuencia de radiación, $\nu_0$, 
no se observaba nada. Pero a partir de $\nu_0$ empezaban a 
liberarse electrones. La frecuencia $\nu_0$ era además 
característica de cada material.

Otro aspecto
interesante a resaltar de este proceso de ionización es que la
energía cinética de esos electrones dependía de la frecuencia
de la radiación incidente, y era independiente de la intensidad
de esta radiación, lo cual era contrario a las predicciones de
Maxwell. Incluso con muy bajas intensidades de radiación
se producía la ionización del material con tal de que se cumpliese
la condici\'on $\nu>\nu_0$.

¿De qué manera se resolvió este problema? Ya en el s. XX, 
\textbf{Albert Einstein} propuso una solución a esta paradoja,
el mismo año en que describió el movimiento Browniano y la 
Teoría de la Relatividad. Para liberar un electrón, teorizó, es
necesario un \textbf{cuanto de energía} superior a un umbral $\nu_0$, y la
energía excedente se libera en forma de energía cinética con la
que sale despedido el electrón. Así define la llamada \textbf{"work
function"},
\begin{equation}
\frac{1}{2}m_ev^2=h\nu - \Phi
\end{equation}
donde $h\nu$ es la energía de la radiación incidente, $\Phi$ es
el umbral necesario para producir la emisión del electrón y
en la lado izquierdo de la ecuación nos encontramos con la 
expresión familiar de la energía cinética. En el caso de que
$h\nu<\Phi$ no se produce liberación de electrones, pero si
por el contrario $h\nu>\Phi$ entonces sí que se produce
ionización y con una cantidad de energía que podemos calcular.

\section{Hipótesis de de Broglie}
De modo que ya hemos establecido el carácter de partícula de las
ondas. Experimentos similares sirvieron para establecer
el comportamiento como onda de los electrones. El experimento
crucial lo realizaron \textbf{Davisson} y \textbf{Germer}, 
que observaron la difracción de electrones en un cristal. 
La difracción es la interferencia causada por un objeto 
interpuesto en el camino de una onda, y al hacer este experimento
se observó que también los electrones, comportándose como ondas,
difractaban.

Otras partículas también son capaces de difractar, como han 
revelado después experimentos en partículas $\alpha$ o en
el hidrógeno molecular, confirmando que las partículas tienen
características de ondas. 

¿Pero cómo racionalizamos esta dualidad de ondas y cuerpos?
Una primera indicación nos la da la relación del físico Francés
\textbf{Louis de Broglie}, que propuso que una partícula que viajase
con un momento lineal $p$ tendría una longitud de onda asociada
\begin{equation}
\lambda=\frac{h}{p}\label{eq:debroglie}
\end{equation}
Los cuerpos de gran tamaño tienen momentos lineales enormes, debido
a su gran masa. Por eso sus longitudes de onda son indetectables
y las propiedades oscilatorias de los cuerpos macroscópicos no 
pueden ser observados.	

\section{Principio de Incertidumbre de Heisenberg}
Otra de las grandes rupturas que establece la Mecánica Cuántica 
con la Mecánica Clásica es nuestra capacidad de medir una 
determinada propiedad con precisión arbitraria. Dentro de la 
formulación de la Mecánica Cuántica es muy importante el Principio
de Incertidumbre, enunciado por \textbf{Werner Heisenberg} en 1927
y que le valió un Premio Nobel en 1932 por "la creación de la Mecánica
Cuántica". 
Establece que es imposible determinar simultáneamente el momento 
y la posición de una partícula. 

Aunque veremos este principio en 
mayor profundidad más adelante en el curso, la precisión de una
medida viene determinada por
\begin{equation}
    \Delta x\Delta q\geq 1/2\hbar\label{eq:heiss}
\end{equation}
donde $\Delta p=(\langle p^2\rangle-\langle p \rangle^2)^{1/2}$ y
$\Delta q=(\langle q^2\rangle-\langle q\rangle^2)^{1/2}$. En realidad
el principio de incertidumbre no sólo afecta al momento y a la posición, sino a 
cualquier par de \textbf{variables complementarias}. Del mismo modo
que en la Ecuación \ref{eq:heiss} para posición y momento, para 
la duración de un proceso cuántico $t$ y su energía $E$ podemos
escribir
\begin{equation}
    \Delta E\Delta t\geq 1/2\hbar.
\end{equation}

\section{Espectros atómicos}
La última evidencia experimental que comentaremos procede de los 
espectros atómicos. Si nosotros obtenemos un espectro de emisión
de un elemento como el hidrógeno, con lo que nos encontramos
es que el espectro es discontinuo. Por tanto, la emisión de un
átomo se produce sólo a unos determinados valores de la longitud
de onda. De paso podemos decir que en el caso de una molécula 
también el espectro de absorción, que corresponde a frecuencias
que permiten a una molécula cambiar de estado, es discontinuo.

Esto hizo que tuvieran que reformularse los modelos atómicos que 
se empleaban, y que parecían incompatibles con esta discontinuidad.
El primer modelo que tuvo en consideración esta propiedad fue
el de \textbf{Niels Bohr}, que partía de dos suposiciones: 
\begin{itemize}[i]
    \item La primera es que un electrón orbita alrededor del
    núcleo en un conjunto discreto de estados estacionarios, 
    correspondientes a órbitas circulares que satisfacen que la
    fuerza centrífuga y la atracción entre el núcleo y el electrón
    son iguales.
    \begin{equation}
        \frac{e^2}{4\pi \epsilon_0 r^2}=m_e\frac{v^2}{r}
    \end{equation}
    \item La segunda suposición era la que introducía la cuantización
    y asumía que las órbitas permitidas debían satisfacer la relación
    $L=n\hbar=m_evr$, donde $L$ es una magnitud denominada ``acción''.
\end{itemize}
Reorganizando estas ecuaciones podemos calcular el radio de las órbitas,
\begin{equation}
    r_n=\bigg(\frac{4\pi \epsilon_0\hbar^2}{m_ee^2}\bigg)n^2=a_0n^2    
\end{equation}
y la correspondiente energía de los diferentes niveles
\begin{equation}
    \begin{split}
%    A & = \frac{\pi r^2}{2} \\
% & = \frac{1}{2} \pi r^2
    E_n= & \frac{1}{2}m_ev^2 - \frac{1}{4\pi \epsilon_0}\frac{e^2}{r}=\\
    & -\frac{m_e}{2\hbar^2}\bigg(\frac{e^2}{4\pi\epsilon_0}\bigg)\bigg(\frac{1}{n^2}\bigg)
    \end{split}
\end{equation}
Para el hidrógeno, esta descripción es satisfactoria, y fue capaz de
explicar transiciones entre niveles de acuerdo con la siguiente 
expresión
\begin{equation}
    \tilde{\nu}=R_H\bigg(\frac{1}{n^2_1} -\frac{1}{n^2_2}\bigg) 
    \label{eq:Rydberg}
\end{equation}
donde $R_H$=109677 cm$^{-1}$ es la constante de Rydberg del 
átomo de hidrógeno. La Ecuación \ref{eq:Rydberg} explica las líneas
espectrales de Lyman, Balmer, y Paschen, donde $n_1=1,2, 3$ y
$n_2=2,3,4$, respectivamente.
