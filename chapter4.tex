\chapter{Momento angular}
En este tema, vamos a examinar los conceptos básicos de los
operadores momento angular y sus componentes, analizando sus
relaciones de conmutación. Asimismo, se verán las funciones
conocidas como armónicos esféricos, que son las funciones
propias de la parte angular del operador de energía cinética.

\section{Movimiento rotacional en física clásica}
Desde un punto de vista clásico, para un sistema que gira 
en un plano (es decir, en 2D), describimos el \textbf{momento
angular} como un vector cuya magnitud nos da la velocidad 
a la que la partícula gira y cuya dirección indica el eje
de rotación. El momento angular, $l$, se define como
\begin{equation}
    l=I\omega
\end{equation}
donde estamos usando el \textbf{momento de inercia} $I$ y
la velocidad angular $\omega$. El momento angular tiene 
en el momento lineal $p$ su contrapartida translacional, mientras que el momento de inercia $I$ y la velocidad
angular $\omega$ juegan un papel equiparable al de la masa
($m$) y la velocidad ($v$). Para una partícula de masa $m$, 
el momento de inercia se define como $I=mr^2$.

%Para acelerar una partícula tenemos que aplicarle un $\textbf{torque}$, 
%$\mathbf{T}$, y la correspondiente ecuación de Newton es 
%\begin{equation}
%    \frac{d\mathbf{J}}{dt}=\mathbf{T}
%\end{equation}
%Si aplicamos un torque $\mathbf{T}$ durante un periodo de tiempo $\tau$
%la energía rotacional es 
%\begin{equation}
%    E_k=\frac{T^2\tau^2}{2I}
%\end{equation}
%que no está cuantizada.

En tres dimensiones, retenemos la naturaleza vectorial
del momento angular y lo definimos como 
$\mathbf{l}= I\bm{\omega}$
o bien como $\mathbf{l} = \mathbf{r}\times\mathbf{p}$.
Podemos reescribir el momento angular usando el siguiente
determinante
\begin{align}
    \mathbf{l}&=\mathbf{r}\times\mathbf{p} =
    \begin{vmatrix}
\mathbf{i} & \mathbf{j} & \mathbf{k}  \\ 
x & y & z \\ 
p_x & p_y & p_z  \notag
\end{vmatrix} = \\
&= (yp_z-zp_y)\mathbf{i} +
(zp_x-xp_z)\mathbf{j} +
(xp_y-yp_x)\mathbf{k}
\end{align}
donde tenemos las contribuciones $l_x$, $l_y$ y $l_z$
multiplicadas por los correspondientes vectores unitarios.

De acuerdo con la mecánica clásica, la energía cinética 
puede expresarse como $E=1/2mv^2=p^2/2m$. Análogamente en 
el caso del movimiento rotacional podemos expresar la 
energía como 
\begin{equation}
    E=1/2I\omega^2=\frac{L^2}{2I}\label{eq:classicrote}
\end{equation}
donde de nuevo podemos establecer una identificación 
entre masa y momento de inercia, y momentos lineal y
angular, para el movimiento translacional y rotacional,
respectivamente.

\section{Descripción mecanocuántica del momento angular}
En Mecánica Cuántica la descripción del movimiento rotacional
implica la definición de una función de onda que reproduzca 
la trayectoria trazada por una partícula sobre la superficie
de una esfera. Sobre esta función de onda pueden actuar
una serie de operadores. El operador $\hat{l}$ no tiene 
funciones propias, lo cual significa que no existe ningún 
estado para el cual esté determinado. Sólo se puede determinar 
su valor promedio. Por el contrario, sí que podemos encontrar
funciones propias como las componentes en $x$, $y$ y
$z$ del momento angular
\begin{equation}
\hat{l}=\hat{l}_x+\hat{l}_y+\hat{l}_z
\end{equation}
que escribimos como
\begin{subequations}
    \begin{align}
\hat{l}_x = 
\frac{\hbar}{\mathrm{i}}\bigg(y\frac{\partial}{\partial z} 
- z\frac{\partial}{\partial y}\bigg) \\
\hat{l}_y = 
\frac{\hbar}{\mathrm{i}}\bigg(z\frac{\partial}{\partial x}
-  x\frac{\partial}{\partial z}\bigg) \\
\hat{l}_z = 
\frac{\hbar}{\mathrm{i}}\bigg(x\frac{\partial}{\partial y} 
- y\frac{\partial}{\partial x}\bigg) 
\end{align}
\end{subequations}

Es importante establecer las relaciones de conmutación
de estos operadores. En concreto, el valor de los 
conmutadores de las distintas componentes del momento
angular es 
\begin{subequations}
    \begin{align}
        [\hat{l}_x,\hat{l}_y] = \mathrm{i}\hbar \hat{l}_z\\
        [\hat{l}_y,\hat{l}_z] = \mathrm{i}\hbar \hat{l}_x\\
        [\hat{l}_z,\hat{l}_x] = \mathrm{i}\hbar \hat{l}_y
    \end{align}
\end{subequations}
Por tanto, las distintas componentes del momento angular no
conmutan entre sí. Son observables complementarios y
de acuerdo con el principio de incertidumbre esto significa
que no podemos especificar más de una componente del momento
angular.

Otro operador de interés es el cuadrado del momento angular
\begin{equation}
    \hat{l}^2=\hat{l}_x^2+\hat{l}_y^2+\hat{l}_z^2=
    \hbar^2\Lambda^2
\end{equation}
que sí conmuta con los tres componentes del momento angular
\begin{equation}
    [ \hat{l}^2,  \hat{l}_q] = 0,\mathrm{~ para~}q=x, y, z
\end{equation}
Por tanto, $\hat{l}^2$ y $\hat{l}_q$ forman un conjunto de 
operadores compatibles.

Si usamos coordenadas esféricas 
\begin{subequations}
\begin{align}
    x=&r\sin\theta\cos\phi\\
    y=&r\sin\theta\sin\phi\\
    z=&r\cos\theta
\end{align}
\end{subequations}
donde $r$ es el radio, $\theta$ es la colatitud y 
$\phi$ es el ángulo azimutal, podemos obtener expresiones
compactas para las componentes del momento angular
\begin{subequations}
    \begin{align}
        \hat{l}_x&= 
        \mathrm{i}\hbar\bigg(
        \sin{\phi}\frac{\partial}{\partial \theta} +
        \cot{\theta}\cos{\phi} \frac{\partial}{\partial\phi}
        \bigg) \\
        \hat{l}_y&= 
        -\mathrm{i}\hbar\bigg(
        \cos{\phi}\frac{\partial}{\partial \theta} -
        \cot{\theta}\sin{\phi} \frac{\partial}{\partial\phi}
        \bigg) \\
        \hat{l}_z&=  -\mathrm{i}\hbar\frac{\partial}{\partial \phi}
    \end{align}\label{eq:lz}
\end{subequations}
así como para el cuadrado del momento angular
\begin{equation}
    \hat{l}^2= -\hbar^2\bigg(
    \frac{\partial^2}{\partial \theta^2} +
        \cot{\theta}\frac{\partial}{\partial \theta} + 
        \frac{1}{\sin^2\theta}\frac{\partial^2}{\partial\phi^2}
        \bigg)
\end{equation}

\section{Hamiltoniano para el movimiento rotacional}
El hamiltoniano para la rotación en tres
dimensiones se escribe como
\begin{equation}
    \hat{H}=-\frac{\hbar^2}{2m}\nabla^2 +V,
    \mathrm{~ donde~}
    \nabla^2=\frac{\partial^2}{\partial x^2} + \frac{\partial^2}{\partial y^2} +\frac{\partial^2}{\partial z^2}
\end{equation}
De nuevo, la energía potencial para la partícula confinada
en una esfera es cero, por lo que la energía cinética es
igual a la energía total
\begin{equation}
    -\frac{\hbar^2}{2m}\nabla^2\psi=E\psi
\end{equation}
En coordenadas esféricas podemos expresar el operador
de la siguiente manera
\begin{equation}
    %\nabla^2=\frac{\partial^2}{\partial r^2} + 
    %\frac{2}{r}\frac{\partial}{\partial r} +
    %\frac{1}{r^2}\Lambda^2
    \nabla^2=\frac{1}{r^2}\frac{\partial}{\partial r}r^2\frac{\partial}{\partial r} + \frac{1}{r^2}\Lambda^2
\end{equation}
donde estamos definiendo el operador
\begin{equation}
    \Lambda^2=\frac{1}{\sin^2\theta}\frac{\partial^2}{\partial\phi^2}+
    \frac{1}{\sin\theta}\frac{\partial}{\partial\theta}\sin\theta\frac{\partial}{\partial\theta}
\end{equation}
En el caso del movimiento esférico podemos cancelar los 
términos dependientes de la distancia, con lo que obtenemos
\begin{equation}
\hat{H}\psi=-\frac{\hbar^2}{2m}\frac{1}{r^2}\Lambda^2\psi=-\frac{\hbar^2}{2I}\Lambda^2\psi=E\psi
\label{eq:rothamiltonian}
\end{equation}
o bien $\Lambda^2\psi=-\varepsilon\psi$ si definimos 
$\varepsilon=2IE/\hbar^2$.

La función de onda se puede escribir separando la parte
dependiente de $\theta$ de la parte dependiente de $\phi$,
de modo que
\begin{equation}
    \psi(\theta, \phi)=\Theta(\theta)\Phi(\phi)
\end{equation}
Si aplicamos el operador de Legendre, $\Lambda^2$, sobre
esta expresión obtenemos
\begin{equation}
    \Lambda^2\psi=\frac{1}{\sin^2\theta}\frac{\partial^2(\Theta\Phi)}{\partial\phi^2}+
    \frac{1}{\sin\theta}\frac{\partial}{\partial\theta}\sin\theta\frac{\partial(\Phi\Theta)}{\partial\theta}=-\varepsilon\Theta\Phi
\end{equation}
Esta expresión se puede reorganizar si sacamos de las 
derivadas los términos que son constantes 
\begin{equation}
    \frac{\Theta}{\sin^2\theta}\frac{d^2(\Phi)}{d\phi^2} + 
    \frac{\Phi}{\sin\theta}\frac{d}{d\theta}\sin\theta\frac{d(\Theta)}{d\theta}+
    \varepsilon\Theta\Phi=0
\end{equation}
Podemos simplificar esta expresión dividiendo entre $\Theta\Phi$ y 
multiplicando por $\sin^2\theta$, lo cual resulta en
\begin{equation}
    \frac{1}{\Phi}\frac{d^2(\Phi)}{d\phi^2} + 
    \frac{\sin\theta}{\Theta}\frac{d}{d\theta}\sin\theta\frac{d(\Theta)}{d\theta}+
    \varepsilon\sin^2\theta=0\label{eq:rotor}
\end{equation}
Para resolver esta ecuación podemos reparar en que el
primer término a la izquierda de la misma depende 
exclusivamente de $\phi$ mientras que los demás 
dependen únicamente de $\theta$. Por tanto ambos deben
ser iguales a una constante. Esta ecuación resolver
usando el método de separación de variables igualando
el término dependiente de $\phi$ a una constante 
$-m^2_l$ y el término dependiente de $\theta$ a $m^2_l$.

Las soluciones para la ecuación 
\begin{equation}
    \frac{1}{\Phi}\frac{d^2(\Phi)}{d\phi^2}=-m^2_l
\end{equation}
tienen la forma general
\begin{equation}
    \Phi=A\mathrm{e}^{\mathrm{i}m_l\phi}
\end{equation}
donde $A$ no es más que la constante de normalización. A 
continuación introducimos las condiciones de contorno cíclicas.
El dominio del ángulo azimutal es $0\leq \phi\leq2\pi$
y por tanto en estos límites la función de onda debe ser
igual para ser continua, $\Phi(0)=\Phi(2\pi)$, y por tanto
los valores aceptables del \textbf{número cuántico magnético},
$m_l$, son $\pm 1,\pm 2...$. Para que la función esté
normalizada, tenemos que definir el valor de
la constante $A$
\begin{equation}
    |A|^2\int_0^{2\pi}\mathrm{e}^{-im_l\phi}\mathrm{e}^{im_l\phi}d\phi=
    2\pi|A|^2
\end{equation}
y por tanto $|A|=(1/2\pi)^{1/2}$ y la función de onda es
\begin{equation}
    \Phi(\phi)=\bigg(\frac{1}{2\pi}\bigg)^{1/2}\mathrm{e}^{im_l\phi}
\end{equation}

Habiendo resuelto la parte correspondiente al ángulo 
azimutal, nos queda pendiente resolver la función de onda
dependiente de la colatitud, $\theta$, en la Ecuación
\ref{eq:rotor}, que resulta mucho más compleja 
matemáticamente
\begin{equation}
    \Theta(\theta)=\bigg\{ 
    \bigg(
    \frac{2l+1}{2}
    \bigg)
    \frac{(l-|m_l|)!}{(l+|m_l|)!}
    \bigg\}^{1/2}
    P_l^{|m_l|}(\cos\theta)
\end{equation}
que depende del número cuántico del el \textbf{número 
cuántico del momento angular orbital}, $l$, y el número cuántico magnético, $m_l$, que ya aparecía en la solución
para $\Phi(\phi)$. 

Ahora que tenemos las dos contribuciones angulares, 
podemos escribir la función de onda al completo,
$Y_{l,m_l}=\Theta(\theta)\Phi(\phi)$. 
En este curso, nos limitaremos a decir que las 
soluciones de la ecuación de Schrödinger son los
\textbf{armónicos esféricos}, $Y_{l,m_l}$, 
que mostramos en la Tabla \ref{tb:legendre}. Estas 
funciones dependen de dos números cuánticos, cuyos 
valores están restringidos a $l=0,1,2,...$ y 
$m_l=l, l-1,...,0,...,-l$. El número cuántico del 
momento angular $l$ es un número entero no negativo, y 
para cada valor de $l$ hay $2l+1$ valores del número 
cuántico magnético $m_l$. El número de nodos en la 
densidad $|Y_{l,m_l}|^2$ para $m_l=0$ varía en función 
del número $l$. Además, para un valor determinado de $l$,
la probabilidad de encontrar a una partícula se propaga 
a lo largo del plano $xy$ al ir aumentando $m_l$.

\begin{table}[t!]
    \centering
    \begin{tabular}{c|c|c}
     $l$ & $m_l$ & $ Y_{l,m_l}$ \\
     \hline
     & & \\
    0 & 0       & $\bigg(\dfrac{1}{4\pi}\bigg)^{1/2}$\\ 
     & & \\
    1 & 0       & $\bigg(\dfrac{3}{4\pi}\bigg)^{1/2}\cos{\theta}$ \\ 
      & $\pm 1$ & $\mp\bigg(\dfrac{3}{8\pi}\bigg)^{1/2}\sin{\theta}\mathrm{e}^{\pm \mathrm{i}\phi}$ \\ 
     & & \\
    2 & 0       & $\bigg(\dfrac{5}{16\pi}\bigg)^{1/2}(3\cos^2\theta-1)$ \\ 
      & $\pm 1$  & $\mp\bigg(\dfrac{15}{8\pi}\bigg)^{1/2}\cos\theta\sin\theta\mathrm{e}^{\pm \mathrm{i}\phi}$  \\ 
      & $\pm 2$  & $\bigg(\dfrac{15}{32\pi}\bigg)^{1/2}\sin^2\theta\mathrm{e}^{\pm 2\mathrm{i}\phi}$ \\ 
    ... &  &
    \end{tabular}
    \caption{Armónicos esféricos, dependientes de los números
    cuánticos $l$ y $m_l$.}
    \label{tb:legendre}
\end{table}

\section{Cuantización de los niveles energéticos y
del momento angular}
Los niveles energéticos para la rotación en tres 
dimensiones se restringen a los siguientes valores
\begin{equation}
    E=l(l+1)\frac{\hbar^2}{2I}, 
    \mathrm{~ para~}l=0,1,2,...
\end{equation}
Claramente, los niveles energéticos están cuantizados 
en función de $l$ y son independientes del número 
cuántico $m_l$. Las $2m_l+1$ funciones de onda
correspondientes al número cuántico $l$ serán 
degeneradas.

Como hemos visto en la Ecuación \ref{eq:classicrote} el 
módulo del momento angular se relaciona con la energía
mediante la expresión $E=L^2/2I$. Análogamente, podemos
deducir que la magnitud del momento angular en la 
descripción cuántica es $\{l(l+1)\}^{1/2}\hbar$ y que va
a estar cuantizado con $l=0,1,2,...$

Como hemos mostrado con anterioridad nuestra
función de onda $Y_{l,m_l}=\Theta(\theta)\Phi(\phi)$
es separable. La componente $z$ del momento angular
$\hat{l}_z$ depende de $\phi$ (ver Ecuación \label{eq:lz})
de modo que a partir de la expresión para $\Phi(\phi)$
podemos calcular su valor
\begin{equation}
    l_z=m_l\hbar
\end{equation}
Cuando $m_l>0$ la proyección en el eje $z$ del momento
angular  es positiva mientras que si $m_l<0$  es negativa.
Recordemos que al conocer $l_z$ no es posible determinar 
las restantes componentes del momento angular $l_x$ y $l_y$.

\section{El rotor rígido}
A continuación, extendemos nuestra discusión sobre el 
movimiento rotacional al caso de dos partículas de masas 
$m_1$ y $m_2$ con vectores de posición $\mathbf{r}_1$ y 
$\mathbf{r}_2$ que se mantienen a una distancia constante,
$R$. En Mecánica Clásica, definimos la posición del centro
de masas como 
\begin{equation}
    \mathbf{R}_\mathrm{cm}= \frac{m_1\mathbf{r}_1+m_2\mathbf{r}_2}{M}
\end{equation}
La \textbf{masa reducida} de un sistema, $\mu$, se define
de la siguiente manera
\begin{equation}
    \frac{1}{\mu}=\frac{1}{m_1}+\frac{1}{m_2}
\end{equation}
o bien
\begin{equation}
    \mu=\frac{m_1m_2}{m_1+m_2}
\end{equation}
En el caso de que las masas sean prácticamente iguales 
($m_1\simeq m_2$) la masa reducida es aproximadamente 
igual a la mitad de la masa ($\mu\simeq m/2$), mientras
que si la masa de una partícula excede mucho a la otra
($m_1>>m_2$) la masa reducida es equiparable
a la masa inferior ($\mu\simeq m_2$).

Para este tipo de sistema, en ausencia de un potencial 
externo, en Mecánica Cuántica escribimos el hamiltoniano
como 
\begin{equation}
    H=-\frac{\hbar^2}{2m_1}\nabla_1^2-\frac{\hbar^2}{2m_2}\nabla_2^2
\end{equation}
Podemos expresar esta función como la suma de dos 
contribuciones, una para el movimiento del centro de 
masas y otra para el movimiento relativo de las partículas,
\begin{equation}
    -\frac{\hbar^2}{2M}\nabla_\mathrm{cm}^2\Psi 
    -\frac{\hbar^2}{2\mu}\nabla^2\Psi=E_\mathrm{total}\Psi
\end{equation}
Cada una de las contribuciones de esta ecuación puede 
escribirse independientemente en virtud de la separabilidad 
de la función de onda, $\Psi=\psi_\mathrm{cm}\psi$,
\begin{subequations}
    \begin{align}
        -\frac{\hbar^2}{2m}\nabla^2_\mathrm{cm}\psi_\mathrm{cm}&=
        E_\mathrm{cm}\psi_\mathrm{cm} \\
        -\frac{\hbar^2}{2\mu}\nabla^2\psi&=
        E\psi
    \end{align}
\end{subequations}
La energía total del sistema es por tanto la suma de 
las dos contribuciones, $E_\mathrm{total}=E_\mathrm{cm}+E$.
La primera corresponde al desplazamiento del centro de 
masas de la partícula, $E_\mathrm{cm}$, y se describe 
con un modelo sencillo como el de la partícula 
libre o la partícula en una caja. La otra contribución,
$E$, corresponde
al movimiento relativo de las dos 
partículas, y como la distancia entre
ellas se mantiene constante, sólo nos 
tenemos que preocupar por la 
contribución angular. 

De manera análoga a como hemos visto en el caso 
del movimiento rotacional para una sola partícula 
(Ecuación \ref{eq:rothamiltonian}), podemos
escribir
\begin{equation}
\frac{1}{R^2}\Lambda^2\psi=-\frac{2\mu E}{\hbar^2}\psi
\label{eq:rotor}
\end{equation}
De nuevo, las soluciones de esta ecuación son los 
armónicos esféricos, exactamente como hemos visto con
anterioridad, pero sustituyendo el momento de
inercia por $I=\mu R^2$. La energía del rotor rígido es 
\begin{equation}
    E_J=\frac{\hbar^2}{2I}J(J+1)=hBJ(J+1)
\end{equation}
donde el número cuántico $J$ remplaza a $l$, y 
definimos la constante rotacional, $B=h/(8\pi^2I)$. 
El resto de las propiedades que hemos visto para el
movimiento rotacional se aplican exactamente de la 
misma manera al rotor rígido.

\section{Movimiento en presencia de un potencial central}
Finalmente, nos ocupamos de la situación más general,
en la que el movimiento rotacional ocurre en presencia
de un potencial central y puede variar la distancia 
entre las dos partículas, como puede ser el caso de un
electrón en un campo electrostático. En este caso debemos
mantener la contribución radial del hamiltoniano, que
se escribirá como
\begin{equation}
    \hat{H} = -\frac{\hbar^2}{2\mu}\nabla^2 + \hat{V}
    \label{eq:central}
\end{equation}
En esta expresión hemos descartado ya la contribución
traslacional a la energía. En la Ecuación \ref{eq:central},
\begin{equation}
        \nabla^2=\frac{1}{r^2}\frac{\partial}{\partial r}r^2\frac{\partial}{\partial r} + \frac{1}{r^2}\Lambda^2
\end{equation}
donde como se puede ver mantenemos los términos radiales.
De la misma manera que venimos haciendo con las diferentes
contribuciones angulares, podemos separar nuestra función
de onda
\begin{equation}
    \psi(r,\theta, \phi)=R(r)Y_{l,m_l}(\theta, \phi)
\end{equation}
Como sabemos por nuestro trabajo sobre el movimiento 
rotacional, $\Lambda^2Y_{l,m_l}=-l(l+1)Y_{l,m_l}$, de modo 
que podemos escribir el hamiltoniano como la siguiente
ecuación diferencial radial
\begin{equation}
    \bigg\{-\frac{\hbar^2}{2\mu}\frac{1}{r^2}\frac{d}{dr}\bigg(r^2\frac{d}{dr}\bigg) + \frac{l(l+1)\hbar^2}{2\mu r^2} + V(r)\bigg\}R(r)=ER(r)
\end{equation}

Este tratamiento matemático que hemos desarrollado
para el movimiento rotacional nos prepara para resolver
de manera exacta sistemas sencillos como el átomo de 
hidrógeno, donde un único electrón experimenta la
influencia del potencial electrostático ejercido por 
el núcleo.