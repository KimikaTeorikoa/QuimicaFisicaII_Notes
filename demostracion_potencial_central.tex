% ****** Start of file apssamp.tex ******
%
%   This file is part of the APS files in the REVTeX 4.1 distribution.
%   Version 4.1r of REVTeX, August 2010
%
%   Copyright (c) 2009, 2010 The American Physical Society.
%
%   See the REVTeX 4 README file for restrictions and more information.
%
% TeX'ing this file requires that you have AMS-LaTeX 2.0 installed
% as well as the rest of the prerequisites for REVTeX 4.1
%
% See the REVTeX 4 README file
% It also requires running BibTeX. The commands are as follows:
%
%  1)  latex apssamp.tex
%  2)  bibtex apssamp
%  3)  latex apssamp.tex
%  4)  latex apssamp.tex
%
\documentclass[notitlepage, amsmath,amssymb,
 aps,12pt,tightenlines]{revtex4-1}
%\documentclass[%
% reprint,
%superscriptaddress,
%groupedaddress,
%unsortedaddress,
%runinaddress,
%frontmatterverbose, 
%preprint,
%showpacs,preprintnumbers,
%nofootinbib,
%nobibnotes,
%bibnotes,
 %amsmath,amssymb,
 %aps,
%pra,
%prb,
%rmp,
%prstab,
%prstper,
%floatfix,
%]{revtex4-1}
\usepackage[spanish, es-tabla]{babel}
\usepackage[utf8x]{inputenc}
\usepackage[T1]{fontenc}
\usepackage{txfonts}
\usepackage{palatino}
\usepackage{graphicx}% Include figure files
\usepackage{dcolumn}% Align table columns on decimal point
\usepackage{bm}% bold math
%\usepackage{hyperref}% add hypertext capabilities
%\usepackage[mathlines]{lineno}% Enable numbering of text and display math
%\linenumbers\relax % Commence numbering lines

%\usepackage[showframe,%Uncomment any one of the following lines to test 
%%scale=0.7, marginratio={1:1, 2:3}, ignoreall,% default settings
%%text={7in,10in},centering,
%%margin=1.5in,
%%total={6.5in,8.75in}, top=1.2in, left=0.9in, includefoot,
%%height=10in,a5paper,hmargin={3cm,0.8in},
%]{geometry}

\begin{document}

\title{Desarrollo matemático de la función de onda total en presencia de un potencial central}% Force line breaks with \\

\author{David De Sancho}
 \email{david.desancho@ehu.eus}
 \affiliation{Departamento de Ciencia y Tecnología de Polímeros\\
 Universidad del País Vasco\\
 Donostia - San Sebasti\'an}

%\date{\today}% It is always \today, today,
             %  but any date may be explicitly specified

\maketitle

\subsection*{Introducción}
Tras obtener las soluciones de la ecuación de
Schrödinger para el movimiento rotacional, los
armónicos esféricos $Y_{l,m_l}(\theta ,\phi)$,
podemos pasar a resolver el problema
de la partícula sometida a un potencial
central. Este problema es todavía más relevante para
nosotros como químicos, porque es precisamente el 
problema al que nos enfrentamos cuando estudiamos
un átomo hidrogenoide. 
Para resolver este problema, utilizaremos los resultados 
del movimiento rotacional, pero tendremos que 
incorporar la dependencia espacial, descrita por
la función radial $R(r)$.

\subsection*{Hamiltoniano}
El Hamiltoniano para el movimiento rotacional en presencia
de un potencial central $V(r)$ es
\begin{equation}
    \hat{H}=-\frac{\hbar^2}{2\mu}\nabla^2 +V(r),
\end{equation}
done estamos usando la masa reducida, $\mu$, y la laplaciana, 
$\nabla^2$, correspondiente al término de energía cinética.
Este operador se puede expresar en coordenadas 
esféricas  de la siguiente manera
\begin{equation}
    %\nabla^2=\frac{\partial^2}{\partial r^2} + 
    %\frac{2}{r}\frac{\partial}{\partial r} +
    %\frac{1}{r^2}\Lambda^2
    \nabla^2=\frac{1}{r^2}\frac{\partial}{\partial r}r^2\frac{\partial}{\partial r} + \frac{1}{r^2}\Lambda^2
\end{equation}
En esta ecuación, estamos definiendo a su vez el operador de Legendre,
que incluye toda la dependencia con los ángulos $\theta$
y $\phi$, y ya empleamos en nuestro tratamiento del movimiento
rotacional,
\begin{equation}
    \Lambda^2=\frac{1}{\sin^2\theta}\frac{\partial^2}{\partial\phi^2}+
    \frac{1}{\sin\theta}\frac{\partial}{\partial\theta}\sin\theta\frac{\partial}{\partial\theta}
\end{equation}

\subsection*{Separación de la función de onda}
Como hemos hecho en el caso de la función de onda angular
para sus correspondientes componentes, $\Theta(\theta)$ y
$\Phi(\phi)$, la función de onda  $\psi(r,\theta,\phi)$ 
se puede escribir separando la parte radial de la función
de onda $R(r)$ y la parte angular, $Y_{l,m_l}(\theta, \phi)$
\begin{equation}
    \psi(r,\theta,\phi) = R(r) Y_{l,m_l}(\theta, \phi)
    \label{eq:sep}
\end{equation}

\subsection*{Desarrollo matemático}
A continuación, aplicamos el hamiltoniano a la función de onda
de la Ecuación \ref{eq:sep}. Así, tenemos
\begin{equation}
\begin{split}
    \hat{H}\psi(r,\theta,\phi)=&
    \bigg\{
    -\frac{\hbar^2}{2\mu} \frac{1}{r^2}\frac{\partial}{\partial r}r^2\frac{\partial}{\partial r} 
    -\frac{\hbar^2}{2\mu} \frac{1}{r^2}\Lambda^2
    +V(r)\bigg\}\psi(r,\theta,\phi)= \\
    =&
    \bigg\{
    -\frac{\hbar^2}{2\mu} \frac{1}{r^2}\frac{\partial}{\partial r}r^2\frac{\partial}{\partial r} 
    -\frac{\hbar^2}{2\mu} \frac{1}{r^2}\Lambda^2
    +V(r)\bigg\}R(r) Y_{l,m_l}(\theta, \phi)=\\
    &=ER(r) Y_{l,m_l}(\theta, \phi)
\end{split}
\end{equation}
Esta expresión se puede reorganizar si sustituimos la función de onda expresada 
como producto de $R(r)$ y $Y(\theta, \phi)$ y ssacamos de las 
derivadas los términos que no
dependen de la variable con la que estamos derivando
\begin{equation}
   \begin{split}
       \hat{H}\psi(r,\theta,\phi)=&\hat{H}R(r) Y_{l,m_l}(\theta, \phi)=\\ 
       =&-\frac{\hbar^2}{2\mu} \frac{1}{r^2}\frac{\partial}{\partial r}r^2\frac{\partial}{\partial r} R(r) Y_{l,m_l}(\theta, \phi)  
       -\frac{\hbar^2}{2\mu} \frac{1}{r^2}\Lambda^2R(r) Y_{l,m_l}(\theta, \phi)
      + V(r)R(r) Y_{l,m_l}(\theta, \phi)
   \end{split} 
\end{equation}
En esta expresión podemos ir resolviendo término a término para ver
que en todos ellos encontramos la componente $Y_{l,m_l}$
\begin{equation}
   \begin{split}
       \hat{H}\psi(r,\theta,\phi)=&\hat{H}R(r) Y_{l,m_l}(\theta, \phi)=\\ =&\bigg\{-\frac{\hbar^2}{2\mu} \frac{1}{r^2}\frac{\partial}{\partial r}r^2\frac{\partial}{\partial r} R(r) 
 +  \frac{l(l+1)\hbar^2}{2\mu r^2} R(r) 
 + V(r)R(r) \bigg\}Y_{l,m_l}(\theta, \phi)=\\
     &=ER(r) Y_{l,m_l}(\theta, \phi)
   \end{split}
   \label{eq:beforelast}
\end{equation}
Para llegar a esta expresión hemos usado los resultados que obtuvimos
en el caso del movimiento rotacional para $\Lambda^2$ y la energía,
$E=l(l+1)\hbar^2/2I$ (transparencias 6 y 14)
\begin{equation}
    \Lambda^2Y_{l,m_l}=-\frac{2IE}{\hbar^2}Y_{l,m_l}= -l(l+1)Y_{l,m_l}
    \label{eq:rotor}
\end{equation}
Si dividimos ambos lados de la Ecuación \ref{eq:beforelast} 
por $Y_{l,m_l}$, obtenemos la ecuación radial 
\begin{equation}
    \bigg\{-\frac{\hbar^2}{2\mu}\frac{1}{r^2}\frac{d}{dr}\bigg(r^2\frac{d}{dr}\bigg) + \frac{l(l+1)\hbar^2}{2\mu r^2} + V(r)\bigg\}R(r)=ER(r)
\end{equation}


\end{document}
%
% ****** End of file apssamp.tex ******