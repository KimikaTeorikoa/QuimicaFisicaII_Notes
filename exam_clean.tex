\documentclass[addpoints,spanish, 12pt,a4paper]{exam}
% Hasta donde pone \begin{document} es lo que se conoce como preámbulo (preamble)

%% Esto es de la clase exam. Si dejamos sin comentar \printanswers, se mostraran las soluciones. 
%% Si la comentamos y dejamos sin comentar \noprintanswers, pues no se muestran las soluciones.
\printanswers
%\noprintanswers

%%%%%%%%%%%%%%%%%%%%%%%%%%%%%%%%%%%%%%%%%%%%%%%%%%%%%%%%%%%%%%%%%%%%%%%%%%%%%%%%%%
% Idioma y codificación de texto
\PassOptionsToPackage{T1}{fontenc} 
\usepackage{fontenc} 
\usepackage[utf8]{inputenc}
% Cargar babel y configurar para español
\usepackage[spanish,es-lcroman, es-tabla, es-noshorthands]{babel}
%%%%%%%%%%%%%%%%%%%%%%%%%%%%%%%%%%%%%%%%%%%%%%%%%%%%%%%%%%%%%%%%%%%%%%%%%%%%%%%%%%
%\usepackage{mathptmx}
\usepackage{amsmath}
\usepackage{mathptmx}% http://ctan.org/pkg/mathptmx
%\usepackage[osf,sc]{mathpazo}
\usepackage[scaled=0.9]{helvet}
\renewcommand{\familydefault}{\sfdefault}


\usepackage{graphicx}
%\usepackage[colorinlistoftodos]{todonotes}
%\usepackage[colorlinks=true, allcolors=blue]{hyperref}

%\renewcommand{\sfdefault}{phv}

%%%%%%% Paquetes varios y alguna opción
\usepackage{graphicx} % Paquete necesario para incluir imágenes, cambiarles el tamaño, etc.
\usepackage{enumitem} % Para poder configurar las listas
\everymath{\displaystyle} % Esto es para que las expresiones se vean... grandes, que resulta diferente de si las queremos entre líneas.

%%%%%%%%%%%%%%%%%%%%%%%%%%%%%%%%%%%%%%%%%%%%%%%%%%%%%%%%%%%%%%%%%%%%%%%%%%%%%%%%%%
%%%% Cosas a configurar de la clase EXAM %%%%

\author{David De Sancho}
\pagestyle{headandfoot}
\runningheadrule
\extraheadheight{3cm}
\firstpageheader{}
    {\hspace*{10cm}\includegraphics[height=2cm]{figs/logotipo_upv_ehu.jpg}\\ 
    \large\textbf{Química Física II}\\
    Examen de los Temas 1 al 4\\
    20 de Noviembre de 2018
    }
{}

\runningheader{\small Química Física II, 1$^\mathrm{er}$ parcial}{}
{\small 20 de Noviembre, 2018}
%\runningfooter{}{{\tiny  }}{\thepage\ de \numpages}
\runningfooter{}{{\tiny  }}{\thepage/\numpages}
\pointpoints{punto}{puntos}
\bonuspointpoints{punto extra}{puntos extra}
\hqword{Pregunta}
\hpword{Puntos}
\hsword{Calificación}
\renewcommand{\solutiontitle}{\noindent\textbf{Solución:}\par\noindent}
\pointformat{(\emph{\thepoints})}
\bonuspointformat{(\emph{\thepoints})}
\pointsinrightmargin % Para poner las puntuaciones a la derecha. Se puede cambiar. Si se comenta, sale a la izquierda.
\extrawidth{-3cm} %Un poquito más de margen por si ponemos textos largos.
\marginpointname{ \emph{\points}}
%\bracketedpoints

%%%%%%%%%%%%%%%%%%%%%% FIN DEL PREÁMBULO %%%%%%%%%%%%%%%%%%%%%

\begin{document}
%\vspace{0.1in} %espacio vertical
\makebox[\textwidth]{Nombre:\enspace\hrulefill}

%%%%%%%%%%%%%%%%%%%%%%%%%%%%%%%%%%%%%%%%%
% Tabla para anotar la calificación
%%%%%%%%%%%%%%%%%%%%%%%%%%%%%%%%%%%%%%%%%

\begin{center}
    %\resizebox{\textwidth}{!}{\gradetable[h][questions]} % Esto es por si la tabla sale muy grande, para ajustarla al ancho
    \gradetable[h][questions]
\end{center}
\vspace{0.1in} % Espacio vertical


\begin{questions} % Comenzamos con las preguntas del examen
% Entre corchetes se pone la puntuación de cada una

    %Pregunta con apartados
    \question[1] Para resolver la denominada "catástrofe
    ultravioleta", Max Planck introduce por primera vez 
    la idea de la cuantización. La expresión resultante
    para la densidad de estados para una longitud de onda
    $\lambda$ es  
    \begin{equation*}
        \rho=\frac{8\pi hc}{\lambda^5(\mathrm{e}^{hc/\lambda k_BT}-1)}\
    \end{equation*}
    Demuestra que la distribución de Planck se reduce 
    a la ley de Rayleigh-Jeans, 
    $\rho=8\pi k_BT/\lambda^4$, 
    para valores elevados de $\lambda$. Ilustra gŕaficamente
    la diferencia entre ambos resultados.
    \newpage
    %\begin{solution}
    %\fontfamily{ptm}\selectfont
%
    %A medida que $\lambda$ aumenta, el exponente
    %$hc/\lambda k_BT$ decrece y a longitudes de 
    %onda muy pequeñas, $hc/\lambda k_BT\ll 1$. 
    %Definiendo $x=hc/\lambda k_BT$ podemos hacer 
    %una expansión en serie de Taylor del término $\mathrm{e}^x$
    %\begin{equation*}
    %    \mathrm{e}^x= 1 + x + \frac{1}{2!}x^2+...
    %\end{equation*}
    %Para $x\ll1$ podemos aproximar $\mathrm{e}^x= 1 + x$
    %y por tanto obtenemos 
    % \begin{align*}
    %    \lim_{\lambda\rightarrow\infty} \rho =&
    %    \frac{8\pi hc}{\lambda^5(1+x-1)} = 
    %     \frac{8\pi hc}{\lambda^5x}=
    %     \frac{8\pi hc}{\lambda^5}\frac{1}{hc/\lambda k_BT} \\
    %     = &\frac{8\pi k_BT}{\lambda^4}
    %\end{align*}
    %\vspace*{20\baselineskip}
    %\end{solution}
    
    \question[1] La función trabajo (\textit{work function})
    del cesio metálico es 2.14 eV. Calcula la energía 
    cinética y la velocidad de los electrones liberados
    por la luz de longitud de onda (a) 700 nm y (b) 300 nm.
    Interpreta el resultado.
    \newpage
    %\begin{solution}
    %    \fontfamily{ptm}\selectfont
%
    %    La conservación de la energía impone que la
    %    energía del fotón se distribuya entre la energía
    %    necesaria para la ionización y la energía cinética
    %    con la que son liberados los electrones,
    %    \begin{equation*}
    %        E=\Phi + E_K
    %    \end{equation*}
    %    La energía del fotón se puede escribir como
    %    $E=h\nu=hc/\lambda$, de modo que la energía 
    %    cinética resultante es $E_K=hc/\lambda - \Phi$.
    %    Asímismo, como la energía cinética es 
    %    $E_K=1/2m_ev^2$, la velocidad es $v=(2E_K/m_e)^{1/2}$.
    %    
    %    \begin{align*}
    %        E_K =&\frac{(6.626\times10^{-34}\textrm{J s})
    %        \cdot (3\times10^8\textrm{m s}^{-1})}{700\times 10^{-9}\mathrm{m}}& - \\ &-2.09\textrm{ eV}\cdot1.6\times 10^{-19} \textrm{J eV}^{-1}=\\
    %        =&-5.04 J
    %    \end{align*}
    %    Este valor de la energía es negativo, debido a que
    %    la longitud de la radiación incidente es demasiado
    %    larga como para ionizar el cesio.
    %    \begin{align*}
    %        E_K =&\frac{(6.626\times10^{-34}\textrm{J s})
    %        \cdot (3\times10^8\textrm{m s}^{-1})}{195\times 10^{-9}\mathrm{m}}& - \\&-2.09\textrm{ eV}\cdot1.6\times 10^{-19} \textrm{J eV}^{-1}=\\
    %        =& 3.28\times10^{-19} \mathrm{J}
    %    \end{align*}
    %    A partir de la energía podemos obtener la velocidad,
    %    \begin{equation*}
    %        v=\frac{2\cdot3.28\times10^{-19} \mathrm{J}}{9.11\times10^{-31}\mathrm{kg}}=848838\mathrm{~ m s}^{-1}
    %    \end{equation*}
    %\vspace*{33\baselineskip}
    %\end{solution}
    
    \question[1\half] Uno de los postulados de la Mecánica
    Cuántica establece que el valor 
    del observable $\Omega$ en una serie de medidas es 
    igual al valor promedio (\textit{expectation value})
    del operador. Si nuestra función de onda $\psi$ no
    es función propia del operador, pero se puede expresar
    como una combinación lineal de dos funciones propias 
    de $\hat{\Omega}$, $\psi_1$ y $\psi_2$, para las que se
    cumple $\hat{\Omega}\psi_1=\omega_1\psi_1$ y
    $\hat{\Omega}\psi_2=\omega_2\psi_2$. Si disponemos
    de un dispositivo experimental que nos permite medir
    el observable $\Omega$: (a) ¿Cuál será el resultado de
    cada medida individual? (b) ¿Cuál será el "expectation
    value"?
    \newpage
    
    %\begin{solution}
%        \fontfamily{ptm}\selectfont
%    Cuando se realiza una medida, se obtendrá el valor 
%    propio $\omega_1$ o $\omega_2$ correspondiente a una
%    de las funciones propias $\psi_1$ o $\psi_2$, 
%    respectivamente. La probabilidad de encontrar 
%    cada una de las soluciones es proporcional
%    al cuadrado del módulo del coeficiente $|c_1|^2$ 
%    o $|c_2|^2$ dentro de la combinación lineal. 
%    El promedio tras un gran número observaciones lo 
%    da el valor esperado, $\langle\Omega\rangle$:
%    \begin{align*}
%    \langle\Omega\rangle  =& 
%\int (c_1\psi_1 + c_2\psi_2)^\star
%\hat{\Omega}(c_1\psi_1 + c_2\psi_2)d\tau = \\
%=&\int (c_1\psi_1 + c_2\psi_2)^\star
%(c_1\hat{\Omega}\psi_1 + c_2\hat{\Omega}\psi_2)d\tau = \\
%=&\int c_1^\star c_1\psi_1^\star\hat{\Omega}\psi_1 d\tau + 
%\int c_2^\star c_2\psi_2^\star\hat{\Omega}\psi_2 d\tau +\\ +&\int c_1^\star c_2\psi_1^\star\hat{\Omega}\psi_2 d\tau + 
%\int c_2^\star c_1\psi_2^\star\hat{\Omega}\psi_1 d\tau = \\
%=& \int c_1^\star c_1\omega_1\psi_1^\star\psi_1 d\tau + 
%\int c_2^\star c_2\omega_2\psi_2^\star\psi_2 d\tau + \\
%+&\int c_1^\star c_2\omega_2\psi_1^\star\psi_2 d\tau + 
%\int c_2^\star c_1\omega_1\psi_2^\star\psi_1 d\tau 
%    \end{align*}
%    Los términos cruzados se cancelan, y por tanto 
%    obtenemos el siguiente "expectation value":
%    \begin{equation*}
%    \langle\Omega\rangle = |c_1|^2w_1 + |c_2|^2w_2
%    \end{equation*}
    %\vspace*{27\baselineskip}
    %\end{solution}
    
    \question[1\half] El principio de incertidumbre
    en su versión más general relaciona el error en 
    la medida de dos operadores,
    $\Delta\Omega_1\Delta\Omega_2$,
    con el valor del conmutador
    $[\hat{\Omega}_1,\hat{\Omega}_2]$. Supongamos
    que nos encontramos en el caso de una partícula
    sometida a un potencial armónico, $V=1/2kx^2$.
    Demuestra que la incertidumbre asociada a los
    operadores posición, $\hat{x}$, y hamiltoniano,
    $\hat{H}_x$, depende de la velocidad de
    desplazamiento de la partícula y es independiente
    del potencial.
    \newpage
    %\begin{solution}
%    En el caso general, el error en la determinación
%    de dos observables se relaciona con el conmutador
%    a través de 
%    $\Delta\Omega_1\Delta\Omega_2\geq1/2|[\hat{\Omega}_1,\hat{\Omega}_2]|$. Para calcular la incertidumbre 
%    en el caso del hamiltoniano y la posición, obtenemos
%    que 
%    \begin{align*}
%     [\hat{H}_x, \hat{x}] = \bigg[\bigg(\frac{p^3_x}{2m} + 1/2kx^2\bigg)x\psi - x\bigg(\frac{p^2_x}{2m} + 1/2kx^2\bigg)\psi\bigg] = \\
%     \bigg[\frac{p^2_x}{2m}x\psi + 1/2kx^3\psi - x\frac{p^2_x}{2m}\psi - 1/2kx^3\psi\bigg]=\\
%     \bigg[\frac{p^2_x}{2m}x\psi - x\frac{p^2_x}{2m}\psi\bigg] =\\
%     \bigg(\frac{\hbar}{\mathrm{i}}\bigg)^2\frac{1}{2m}
%     \bigg[\frac{\partial^2}{\partial x^2}x\psi\bigg] - x\frac{p^2_x}{2m}\psi=\\
%    \bigg(\frac{\hbar}{\mathrm{i}}\bigg)^2\frac{1}{2m}
%     \bigg[\frac{\partial}{\partial x}\bigg(\psi + x\frac{\partial}{\partial x}\psi\bigg)\bigg] - x\frac{p^2_x}{2m}\psi=\\
%     \bigg(\frac{\hbar}{\mathrm{i}}\bigg)^2\frac{1}{2m}
%     \bigg[\frac{\partial\psi}{\partial x} +
%     \bigg(\frac{\partial\psi}{\partial x} + x\frac{\partial^2\psi}{\partial x^2}\bigg)\bigg] - x\frac{p^2_x}{2m}\psi= \\
%     \bigg(\frac{\hbar}{\mathrm{i}}\bigg)^2\frac{1}{2m}
%     2\frac{\partial\psi}{\partial x} + x\frac{p^2_x}{2m}\psi - x\frac{p^2_x}{2m}\psi = 
%     -\frac{i\hbar}{m}\hat{p}_x\psi 
%    \end{align*}
%    Por tanto el conmutador y la incertidumbre en
%    posición y energía dependen de la velocidad de la
%    partícula, pero no de la constante del potencial
%    armónico.
   %     \vspace*{29\baselineskip}
   % \end{solution}

    \question[2] En el caso de la partícula en una
    caja de potencial en una dimensión, nos encontramos
    con que $V(0)=V(L)=\infty$ y $V(x)=0$ para 
    $0<x<L$. La solución para este sistema en 
    la región en la que no hay un potencial externo es
    \begin{equation*}
        \psi=A\exp(ikx) + B\exp(-ikx) 
    \end{equation*}
    Razone, para este sistema, el valor de la constante
    de normalización, el origen de la cuantización y la
    separación entre niveles energéticos.
    \newpage%\begin{solution}
    %    \vspace*{29\baselineskip}
    %\end{solution}
    
    \question[1\half] Las soluciones del oscilador armónico
    son de la forma
    \begin{equation*}
        \psi_v(x) = N_vH_v(y)\mathrm{e}^{-y^2/2}    
    \end{equation*}
    donde $H_v$ son los polinomios de Hermite y 
    hemos sustituido $y=x/\alpha$ y 
    $\alpha=(\hbar/mk)^{1/4}$. Explica, valiéndote de un
    dibujo, cómo se aplica el principio de correspondencia
    a este sistema.
    \newpage
    %\begin{solution}
    %    \vspace*{29\baselineskip}
    %\end{solution}
    
    \question[1\half] Los armónicos esféricos son las soluciones
    de la ecuación de Schrödinger para el movimiento
    rotacional y dependen de dos, números cuánticos $l$ y 
    $m$. Prueba que el armónico esférico
    \begin{equation*}
        Y_{1,-1}=(3/8\pi)^{1/2}\sin\theta\mathrm{e}^{-i\phi}
     \end{equation*}
    está normalizado y es ortogonal a    
    \begin{equation*}
        Y_{2,1}=(15/8\pi)^{1/2}\sin\theta\cos\theta\mathrm{e}^{i\phi}
    \end{equation*}
    Recuerda que para dos funciones de onda $\psi_i$ y $\psi_j$ al usar coordenadas esféricas podemos 
    escribir
    \begin{equation*}
        \int_\infty^\infty\psi_i^\star\psi_j d\tau=
    \int_0^\pi \sin\theta d\theta\int_0^{2\pi}d\phi\psi_i^\star\psi_j
    \end{equation*}
    \newpage
    %\begin{solution}
%    \fontfamily{ptm}\selectfont
%    La condición de normalización aplicada al armónico esférico $Y_{1,-1}$
%    \begin{equation*}
%        \int_0^\pi d\theta\sin\theta\int_0^{2\pi}d\phi Y_{1,-1}^{\star}Y_{1,-1}=1
%    \end{equation*}
%    Sustituyendo la expresión para $Y_{1,-1}$ nos 
%    encontramos con
%    \begin{align*}
%        \int_0^\pi d\theta\int_0^{2\pi}d\phi Y_{1,-1}^{\star}Y_{1,-1}&=
%        \frac{3}{8\pi}\int_0^\pi \sin\theta(\sin^2\theta) d\theta\int_0^{2\pi}\mathrm{e}^{i\phi}\mathrm{e}^{-i\phi}d\phi =\\
%        &=\frac{3}{8\pi}2\pi\int_0^\pi \sin\theta(\sin^2\theta) d\theta=\\
%        &=\frac{3}{4}\int_0^\pi \sin\theta(1-\cos^2\theta) d\theta=\\
%       &=\frac{3}{4}\bigg(\int_0^\pi \sin\theta d\theta -\int_0^\pi \sin\theta\cos^2\theta d\theta\bigg)=1
%    \end{align*}
%    De manera que efectivamente el armónico esférico está 
%    normalizado.
%    
%    La condición de ortogonalidad en coordenadas esféricas
%    para $Y_{1,-1}$ y $Y_{2,1}$ es
%    \begin{equation*}
%        \int_0^\pi d\theta\sin\theta\int_0^{2\pi}d\phi Y_{2,1}^{\star}Y_{1,-1}=0
%    \end{equation*}
%    Si desarrollamos esta expresión,
%    \begin{align*}
%        &\int_0^\pi d\theta\sin\theta\int_0^{2\pi}d\phi Y_{2,1}^{\star}Y_{1,-1}=\\
%        &\bigg(\frac{15}{8\pi}\bigg)^{1/2} \bigg(\frac{3}{8\pi}\bigg)^{1/2}
%        \int_0^\pi d\theta\sin\theta
%        \int_0^{2\pi}d\phi
%        (\mathrm{e}^{-i\phi}\sin\theta\cos\theta)(\mathrm{e}^{-i\phi}\sin\theta)=\\
%        &\bigg(\frac{15}{8\pi}\bigg)^{1/2} \bigg(\frac{3}{8\pi}\bigg)^{1/2}
%           \int_0^\pi d\theta\sin^3\theta\cos\theta
%           \int_0^{2\pi}d\phi \mathrm{e}^{-2i\phi}
%    \end{align*}
%   La integral correspondiente al ángulo $\phi$ es cero
%   porque es la suma de la integral sobre el coseno y el 
%   seno sobre dos periodos completos. Por tanto la 
%   integral es cero y los armónicos esféricos son
%   ortogonales.
    %\vspace*{24\baselineskip}
    %\end{solution}
    %\begin{solution}
    %\vspace*{24\baselineskip}
%    \end{solution}
    

\end{questions}

\end{document}