\documentclass[addpoints,spanish, 12pt,a4paper]{exam}
% Hasta donde pone \begin{document} es lo que se conoce como preámbulo (preamble)

%% Esto es de la clase exam. Si dejamos sin comentar \printanswers, se mostraran las soluciones. 
%% Si la comentamos y dejamos sin comentar \noprintanswers, pues no se muestran las soluciones.
\printanswers
%\noprintanswers

%%%%%%%%%%%%%%%%%%%%%%%%%%%%%%%%%%%%%%%%%%%%%%%%%%%%%%%%%%%%%%%%%%%%%%%%%%%%%%%%%%
% Idioma y codificación de texto
\PassOptionsToPackage{T1}{fontenc} 
\usepackage{fontenc} 
\usepackage[utf8]{inputenc}
% Cargar babel y configurar para español
\usepackage[spanish,es-lcroman, es-tabla, es-noshorthands]{babel}
%%%%%%%%%%%%%%%%%%%%%%%%%%%%%%%%%%%%%%%%%%%%%%%%%%%%%%%%%%%%%%%%%%%%%%%%%%%%%%%%%%
%\usepackage{mathptmx}
\usepackage{amsmath}
\usepackage{mathptmx}% http://ctan.org/pkg/mathptmx
%\usepackage[osf,sc]{mathpazo}
\usepackage[scaled=0.9]{helvet}
\renewcommand{\familydefault}{\sfdefault}


\usepackage{graphicx}
%\usepackage[colorinlistoftodos]{todonotes}
%\usepackage[colorlinks=true, allcolors=blue]{hyperref}

%\renewcommand{\sfdefault}{phv}

%%%%%%% Paquetes varios y alguna opción
\usepackage{graphicx} % Paquete necesario para incluir imágenes, cambiarles el tamaño, etc.
\usepackage{enumitem} % Para poder configurar las listas
\everymath{\displaystyle} % Esto es para que las expresiones se vean... grandes, que resulta diferente de si las queremos entre líneas.

%%%%%%%%%%%%%%%%%%%%%%%%%%%%%%%%%%%%%%%%%%%%%%%%%%%%%%%%%%%%%%%%%%%%%%%%%%%%%%%%%%
%%%% Cosas a configurar de la clase EXAM %%%%

\author{David De Sancho}
\pagestyle{headandfoot}
\runningheadrule
\extraheadheight{3cm}
\firstpageheader{}
    {\hspace*{10cm}\includegraphics[height=2cm]{figs/logotipo_upv_ehu.jpg}\\ 
    \textbf{Química Física II}\\
    Examen de los Temas 1 al 4\\
    29 de Mayo de 2019
    }
{}

\runningheader{\small Química Física II, 1$^\mathrm{er}$ parcial}{}
{\small 20 de Noviembre, 2018}
\runningfooter{}{{\tiny  }}{Página \thepage\ de \numpages}
\pointpoints{punto}{puntos}
\bonuspointpoints{punto extra}{puntos extra}
\hqword{Pregunta}
\hpword{Puntos}
\hsword{Calificación}
\renewcommand{\solutiontitle}{\noindent\textbf{Solución:}\par\noindent}
\pointformat{(\emph{\thepoints})}
\bonuspointformat{(\emph{\thepoints})}
\pointsinrightmargin % Para poner las puntuaciones a la derecha. Se puede cambiar. Si se comenta, sale a la izquierda.
\extrawidth{-3cm} %Un poquito más de margen por si ponemos textos largos.
\marginpointname{ \emph{\points}}
%\bracketedpoints

%%%%%%%%%%%%%%%%%%%%%% FIN DEL PREÁMBULO %%%%%%%%%%%%%%%%%%%%%

\begin{document}
\vspace{0.1in} %espacio vertical
\makebox[\textwidth]{Nombre:\enspace\hrulefill}

%%%%%%%%%%%%%%%%%%%%%%%%%%%%%%%%%%%%%%%%%
% Tabla para anotar la calificación
%%%%%%%%%%%%%%%%%%%%%%%%%%%%%%%%%%%%%%%%%

\begin{center}
    %\resizebox{\textwidth}{!}{\gradetable[h][questions]} % Esto es por si la tabla sale muy grande, para ajustarla al ancho
    \gradetable[h][questions]
\end{center}
\vspace{0.1in} % Espacio vertical


\begin{questions} % Comenzamos con las preguntas del examen
% Entre corchetes se pone la puntuación de cada una

    %Pregunta con apartados
    \question[2\half] Una de las observaciones que llevaron al descubrimiento
    del efecto fotoeléctrico fue que, al incidir con luz sobre superficies
    de diferentes materiales, la energía cinética de los fotoelectrones 
    dependía linealmente con la frecuencia de la radiación incidente, $\nu$.
    ¿Cómo explica esta observación? 
    ¿A partir de qué valor de la frecuencia empiezan a emitirse 
    fotoelectrones?

    \newpage
    
   % \question Calcula:
   %     \begin{parts}
   %         \part[1] $2+2=$
   %         \part[2] $\frac{1}{2}+\frac{3}{4}=$ 
   %     \end{parts}
   % \begin{solution} % Aquí ponemos la solución, es opcional.
   %     \begin{parts}
   %         \part $4$
   %         \part $\frac{5}{4}$ 
   %     \end{parts}
   % \end{solution}
    
    \question[2\half] Las funciones $e^{ikx}$ y $e^{-ikx}$ son funciones propias
    del operador momento lineal, 
    \begin{equation*}
        \hat{p}_x=-i\hbar \frac{\partial}{\partial x}
    \end{equation*}
    ¿Cuáles son sus valores propios? ¿A qué sistema corresponden? Cualquier combinación lineal de estas funciones, ¿es también función
    propia de $\hat{p}_x$?
    \newpage 
    
    
    \question[2\half] 
     Un rodamiento de masa 1 g se encuentra en una caja  de potencial de
     anchura 10 cm moviéndose a una velocidad de 1 cm/s. 
     (a) Calcula el número cuántico. 
     (b) Muestra cuál es el espaciamiento entre dos niveles para el valor del 
     número cuántico obtenido en el apartado anterior. 
     (c) ¿Qué principio ilustran los resultados?
    \newpage
    
    \question[2\half] Las primeras funciones de onda del oscilador armónico
    son 
    \begin{align*}
        & \psi_0(x)=\bigg(\frac{\alpha}{\pi}\bigg)^{1/4}e^{-\alpha x^2/2} \\
        & \psi_1(x)=\bigg(\frac{4\alpha^3}{\pi}\bigg)^{1/4}xe^{-\alpha x^2/2}
    \end{align*}
    Demuestra que las funciones son ortogonales entre sí.
    \newpage    


\end{questions}

\end{document}