\documentclass[a4paper, 11pt]{article}
\usepackage[spanish, es-tabla]{babel}

%% Language and font encodings
%\usepackage[english]{babel}
\usepackage[utf8x]{inputenc}
\usepackage[T1]{fontenc}

%% Sets page size and margins
\usepackage[a4paper,top=3cm,bottom=2cm,left=3cm,right=3cm,marginparwidth=1.75cm]{geometry}

%% Useful packages
\usepackage{amsmath}
\usepackage{mathptmx}% http://ctan.org/pkg/mathptmx
\usepackage[osf,sc]{mathpazo}
\usepackage[scaled=1]{helvet}

\usepackage{graphicx}
%\usepackage[colorinlistoftodos]{todonotes}
%\usepackage[colorlinks=true, allcolors=blue]{hyperref}

\renewcommand{\sfdefault}{phv}
\usepackage[margin=0pt,font={sf,scriptsize},labelfont=bf]{caption}
\usepackage{sansmath}
\usepackage{float} % to easily modify floats
\usepackage{etoolbox} % nice command patching
\usepackage{everyhook} % nice \every... patching
\restylefloat{figure}
\floatevery{figure}{\PushPreHook{math}{\sansmath}}
% undo the change to \everymath at the end of the figure (etoolbox)
\apptocmd{\endfigure}{\PopPreHook{math}}{}{}%\usepackage{helvet}
\usepackage{sectsty}
\allsectionsfont{\sffamily}


\title{\huge\textsf{\textbf{Qu\'imica Física II}\\
\Large \textit{Ejercicios de Química Cuántica}}}
%\author{\sf David De Sancho Sánchez}
\date{}
\begin{document}
\maketitle
\large
\begin{enumerate}
    \item La función trabajo (\textit{work
    function}) para el potasio es de 2.24
    eV. Si se ilumina el potasio metálico
    con luz de longitud de onda de 480 nm,
    calcula (a) la energía cinética máxima
    de los fotoelectrones y (b) la longitud
    de onda umbral.
    
    \item Se lanza una roca con masa de 
    50 g con velocidad de 40 m/s. ¿Cuál es
    su longitud de onda de De Broglie?
    
    \item Se mide la velocidad de un
    electrón y resulta ser de $5\times10^3$
    m/s $\pm$0.003\%. ¿Dentro de qué
    límites puede conocerse la posición de
    dicho electrón a lo largo de la
    dirección del vector velocidad?
    
    \item Suponiendo que $\psi_1$ y
    $\psi_2$ son dos funciones reales que 
    están normalizadas y son ortogonales, 
    normaliza las siguientes funciones:
    (a) $\psi_1+\psi_2$, (b) 
    $\psi_1-\psi_2$.
    
    \item ¿Cuál de las siguientes funciones
    es función propia del operador
    $\hat{D}=\frac{d}{dx}$? (a) $kx^2$
    (b) $\sin x$ (c) $\exp (kx)$ 
    (d) $\exp(kx^2)$ (e) $\exp(ikx)$.
    
    \item Los operadores para la posición y
    el momento lineal vienen dados por 
    \begin{equation}
        \begin{array}{cc}
             \hat{x}& =x  \\
            \hat{p}_x& =\frac{\hbar}{i}\big(\frac{\partial}{\partial x}\big)
        \end{array}
    \end{equation}
    (a) ¿Son estos operadores hermíticos?
    (b) ¿Conmutan estos operadores?
    (c) ¿Son los operadores citados lineales
    o no lineales?
    (d) ¿Es la función $\psi_1=A\sin (n\pi
    x/a)$, donde $A$, $n$ y $a$ son
    constantes, una función propia de ambos
    operadores?
   
   \item Normaliza la función del ejercicio
   6d en el dominio $0\leq x\leq a$.
   
   \item Determina las condiciones necesarias 
   para que la función $\psi(x)=x\exp(-\alpha
   x^2)$ sea función propia del operador 
   $\hat{\Omega}=\frac{d^2}{dx^2} + bx^2$.
   Normalizar la función y calcular el valor 
   propio de $x$ para un
   sistema descrito por dicha función.
   
   \item Calcula el valor medio del momento y
   la posición para el caso de una partícula de
   masa $m$ en una caja unidimensional de anchura
   $L$. Comprueba que se cumple el principio de
   incertidumbre.
   
   \item Sea una partícula de masa $m$ que se
   mueve en una caja de potencial de anchura $L$.
   (a) Calcula la probabilidad de que la partícula
   se encuentre entre $L/4$ y $3L/4$. (b) Evalua el
   límite clásico.
   
  \item Un rodamiento de masa 1 g se encuentra
  en una caja unidimensional de anchura 10 cm
  moviéndose a una velocidad de 1 cm/s. (a) Calcula
  el número cuántico. (b) Muestra cuál es el especiamiento
  entre dos niveles para el valor del número cuántico 
  obtenido en el apartado anterior. (c) ¿Qué ilustran los 
  resultados?
  
  \item Una masa de 45 g en un muelle oscila a la 
  frecuencia de 2.4 vibraciones por segundo con 
  una amplitud de 4 cm. (a) Calcula la constante 
  de fuerza del muelle. (b) ¿Cuál sería el número
  cuántico $v$ si el sistema se tratase 
  mecanocuánticamente?
  
  \item Calcula la frecuencia de la radiación emitida
  cuando un oscilador armónico de frecuencia $6\times 10^{13}$
  s$^{-1}$ salta del nivel 8 al 7.
  
  \item Comprueba que el armónico esférico $Y_{10}$ es una solución 
  de la ecuación $\Lambda^2Y_{l,m_l}=-l(l+1)Y_{l,m_l}$.
\end{enumerate}
\end{document}