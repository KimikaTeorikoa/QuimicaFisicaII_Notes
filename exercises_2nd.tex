\documentclass[a4paper, 11pt]{article}
\usepackage[spanish, es-tabla]{babel}

%% Language and font encodings
%\usepackage[english]{babel}
\usepackage[utf8x]{inputenc}
\usepackage[T1]{fontenc}

%% Sets page size and margins
\usepackage[a4paper,top=3cm,bottom=2cm,left=3cm,right=3cm,marginparwidth=1.75cm]{geometry}

%% Useful packages
\usepackage{amsmath}
\usepackage{mathptmx}% http://ctan.org/pkg/mathptmx
\usepackage[osf,sc]{mathpazo}
\usepackage[scaled=1]{helvet}

\usepackage{graphicx}
%\usepackage[colorinlistoftodos]{todonotes}
%\usepackage[colorlinks=true, allcolors=blue]{hyperref}

\renewcommand{\sfdefault}{phv}
\usepackage[margin=0pt,font={sf,scriptsize},labelfont=bf]{caption}
\usepackage{sansmath}
\usepackage{float} % to easily modify floats
\usepackage{etoolbox} % nice command patching
\usepackage{everyhook} % nice \every... patching
\restylefloat{figure}
\floatevery{figure}{\PushPreHook{math}{\sansmath}}
% undo the change to \everymath at the end of the figure (etoolbox)
\apptocmd{\endfigure}{\PopPreHook{math}}{}{}%\usepackage{helvet}
\usepackage{sectsty}
\allsectionsfont{\sffamily}


\title{\huge\textsf{\textbf{Qu\'imica Física II}\\
\Large \textit{Problemas de Estructura Atómica y Molecular}}}
%\author{\sf David De Sancho Sánchez}
\date{}
\begin{document}
\maketitle
\large
\begin{enumerate}
    \item Calcula la probabilidad de que un electrón descrito por la
    función de onda 1s del átomo de hidrógeno se encuentre a una distancia del 
    núcleo igual o inferior al primer radio de Bohr.
    
    \item Calcula el radio más probable al que se encontrará un
    electrón cuando ocupe un orbital 1s de un átomo 
    hidrogenoide con número atómico Z.
    
    \item Calcula la posición exacta de los nodos de las funciones
    radiales de los orbitales 1s, 2s, 2p y 3s del átomo de hidrógeno.
    
    \item Una partícula se encuentra sometida al siguiente potencial
    \begin{equation*}
\hat{V}(x)=
\begin{cases}
  \infty & \text{ si } L<x<0\\
  0  & \text{ si } 0\leq x\leq \frac{L}{4}\text{  o  }\frac{3L}{4}\leq x\leq L
\\
  V_0, & \text{si } \frac{L}{4}<x<\frac{3L}{4}
\end{cases}
\end{equation*}
    donde $V_0$ es una constante pequeña. Usa la teoría de perturbaciones
    para calcular las correcciones de primer orden de la energía, utilizando
    como hamiltoniano de orden cero el correspondiente a la partícula en 
    una caja.
    
    \item Calcula la corrección de primer orden a la energía del 
    estado fundamental de un oscilador armónico cuya energía potencial 
    es 
    \begin{equation*}
        V(x)=\frac{1}{2}kx^2 + \frac{1}{6}\gamma x^3  + \frac{1}{24}bx^4
    \end{equation*}
\end{enumerate}
\end{document}