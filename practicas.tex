%%%%%%%%%%%%%%%%%%%%%%%%%%%%%%%%%%%%%%%%%%%%%%%%%%%%%%%%%%%%%%%%%%%%%%
% How to use writeLaTeX: 
%
% You edit the source code here on the left, and the preview on the
% right shows you the result within a few seconds.
%
% Bookmark this page and share the URL with your co-authors. They can
% edit at the same time!
%
% You can upload figures, bibliographies, custom classes and
% styles using the files menu.
%
% If you're new to LaTeX, the wikibook is a great place to start:
% http://en.wikibooks.org/wiki/LaTeX
%
%%%%%%%%%%%%%%%%%%%%%%%%%%%%%%%%%%%%%%%%%%%%%%%%%%%%%%%%%%%%%%%%%%%%%%
\documentclass{tufte-book}

\hypersetup{colorlinks}% uncomment this line if you prefer colored hyperlinks (e.g., for onscreen viewing)

%%
% Book metadata
%\title{A Tufte-Style Book\thanks{Thanks to Edward R.~Tufte for his inspiration.}}
\author{Xabier L\'opez, Jon M. Matxain, David De Sancho \& Irene Casademont}
%\publisher{Publisher of This Book}
\usepackage[spanish, es-tabla]{babel}
\usepackage[utf8x]{inputenc}
\usepackage[T1]{fontenc}

\title{Qu\'imica F\'isica \\\huge{Pr\'acticas de Ordenador}}
%%
% If they're installed, use Bergamo and Chantilly from www.fontsite.com.
% They're clones of Bembo and Gill Sans, respectively.
%\IfFileExists{bergamo.sty}{\usepackage[osf]{bergamo}}{}% Bembo
%\IfFileExists{chantill.sty}{\usepackage{chantill}}{}% Gill Sans

\usepackage{microtype}

%%
% Just some sample text
\usepackage{lipsum}

%%
% For nicely typeset tabular material
\usepackage{booktabs}

%%
% For graphics / images
\usepackage{graphicx}
\setkeys{Gin}{width=\linewidth,totalheight=\textheight,keepaspectratio}
\graphicspath{{graphics/}}

% The fancyvrb package lets us customize the formatting of verbatim
% environments.  We use a slightly smaller font.
\usepackage{fancyvrb}
\fvset{fontsize=\normalsize}

%%
% Prints argument within hanging parentheses (i.e., parentheses that take
% up no horizontal space).  Useful in tabular environments.
\newcommand{\hangp}[1]{\makebox[0pt][r]{(}#1\makebox[0pt][l]{)}}

%%
% Prints an asterisk that takes up no horizontal space.
% Useful in tabular environments.
\newcommand{\hangstar}{\makebox[0pt][l]{*}}

%%
% Prints a trailing space in a smart way.
\usepackage{xspace}


\setcounter{secnumdepth}{0}

% Allows for fancy math alignments
\usepackage{amsmath}
\usepackage{bm}
%\numberwithin{equation}{chapter}
%%

% Theorems
\newtheorem{theorem}{Postulado}
 
% Prints the month name (e.g., January) and the year (e.g., 2008)
\newcommand{\monthyear}{%
  \ifcase\month\or January\or February\or March\or April\or May\or June\or
  July\or August\or September\or October\or November\or
  December\fi\space\number\year
}

% Prints an epigraph and speaker in sans serif, all-caps type.
\newcommand{\openepigraph}[2]{%
  %\sffamily\fontsize{14}{16}\selectfont
  \begin{fullwidth}
  \sffamily\large
  \begin{doublespace}
  \noindent\allcaps{#1}\\% epigraph
  \noindent\allcaps{#2}% author
  \end{doublespace}
  \end{fullwidth}
}

% Inserts a blank page
\newcommand{\blankpage}{\newpage\hbox{}\thispagestyle{empty}\newpage}

\usepackage{units}

% Typesets the font size, leading, and measure in the form of 10/12x26 pc.
\newcommand{\measure}[3]{#1/#2$\times$\unit[#3]{pc}}

% Macros for typesetting the documentation
\newcommand{\hlred}[1]{\textcolor{Maroon}{#1}}% prints in red
\newcommand{\hangleft}[1]{\makebox[0pt][r]{#1}}
\newcommand{\hairsp}{\hspace{1pt}}% hair space
\newcommand{\hquad}{\hskip0.5em\relax}% half quad space
\newcommand{\TODO}{\textcolor{red}{\bf TODO!}\xspace}
\newcommand{\ie}{\textit{i.\hairsp{}e.}\xspace}
\newcommand{\eg}{\textit{e.\hairsp{}g.}\xspace}
\newcommand{\na}{\quad--}% used in tables for N/A cells
\providecommand{\XeLaTeX}{X\lower.5ex\hbox{\kern-0.15em\reflectbox{E}}\kern-0.1em\LaTeX}
\newcommand{\tXeLaTeX}{\XeLaTeX\index{XeLaTeX@\protect\XeLaTeX}}
% \index{\texttt{\textbackslash xyz}@\hangleft{\texttt{\textbackslash}}\texttt{xyz}}
\newcommand{\tuftebs}{\symbol{'134}}% a backslash in tt type in OT1/T1
\newcommand{\doccmdnoindex}[2][]{\texttt{\tuftebs#2}}% command name -- adds backslash automatically (and doesn't add cmd to the index)
\newcommand{\doccmddef}[2][]{%
  \hlred{\texttt{\tuftebs#2}}\label{cmd:#2}%
  \ifthenelse{\isempty{#1}}%
    {% add the command to the index
      \index{#2 command@\protect\hangleft{\texttt{\tuftebs}}\texttt{#2}}% command name
    }%
    {% add the command and package to the index
      \index{#2 command@\protect\hangleft{\texttt{\tuftebs}}\texttt{#2} (\texttt{#1} package)}% command name
      \index{#1 package@\texttt{#1} package}\index{packages!#1@\texttt{#1}}% package name
    }%
}% command name -- adds backslash automatically
\newcommand{\doccmd}[2][]{%
  \texttt{\tuftebs#2}%
  \ifthenelse{\isempty{#1}}%
    {% add the command to the index
      \index{#2 command@\protect\hangleft{\texttt{\tuftebs}}\texttt{#2}}% command name
    }%
    {% add the command and package to the index
      \index{#2 command@\protect\hangleft{\texttt{\tuftebs}}\texttt{#2} (\texttt{#1} package)}% command name
      \index{#1 package@\texttt{#1} package}\index{packages!#1@\texttt{#1}}% package name
    }%
}% command name -- adds backslash automatically
\newcommand{\docopt}[1]{\ensuremath{\langle}\textrm{\textit{#1}}\ensuremath{\rangle}}% optional command argument
\newcommand{\docarg}[1]{\textrm{\textit{#1}}}% (required) command argument
\newenvironment{docspec}{\begin{quotation}\ttfamily\parskip0pt\parindent0pt\ignorespaces}{\end{quotation}}% command specification environment
\newcommand{\docenv}[1]{\texttt{#1}\index{#1 environment@\texttt{#1} environment}\index{environments!#1@\texttt{#1}}}% environment name
\newcommand{\docenvdef}[1]{\hlred{\texttt{#1}}\label{env:#1}\index{#1 environment@\texttt{#1} environment}\index{environments!#1@\texttt{#1}}}% environment name
\newcommand{\docpkg}[1]{\texttt{#1}\index{#1 package@\texttt{#1} package}\index{packages!#1@\texttt{#1}}}% package name
\newcommand{\doccls}[1]{\texttt{#1}}% document class name
\newcommand{\docclsopt}[1]{\texttt{#1}\index{#1 class option@\texttt{#1} class option}\index{class options!#1@\texttt{#1}}}% document class option name
\newcommand{\docclsoptdef}[1]{\hlred{\texttt{#1}}\label{clsopt:#1}\index{#1 class option@\texttt{#1} class option}\index{class options!#1@\texttt{#1}}}% document class option name defined
\newcommand{\docmsg}[2]{\bigskip\begin{fullwidth}\noindent\ttfamily#1\end{fullwidth}\medskip\par\noindent#2}
\newcommand{\docfilehook}[2]{\texttt{#1}\index{file hooks!#2}\index{#1@\texttt{#1}}}
\newcommand{\doccounter}[1]{\texttt{#1}\index{#1 counter@\texttt{#1} counter}}

% Generates the index
\usepackage{makeidx}
\makeindex

\begin{document}

% Front matter
%\frontmatter

% r.1 blank page
%\blankpage

%% v.2 epigraphs
%\newpage\thispagestyle{empty}
%\openepigraph{%
%The public is more familiar with bad design than good design.
%It is, in effect, conditioned to prefer bad design, 
%because that is what it lives with. 
%The new becomes threatening, the old reassuring.
%}{Paul Rand%, {\itshape Design, Form, and Chaos}
%}
%\vfill
%\openepigraph{%
%A designer knows that he has achieved perfection 
%not when there is nothing left to add, 
%but when there is nothing left to take away.
%}{Antoine de Saint-Exup\'{e}ry}
%\vfill
%\openepigraph{%
%\ldots the designer of a new system must not only be the implementor and the first 
%large-scale user; the designer should also write the first user manual\ldots 
%If I had not participated fully in all these activities, 
%literally hundreds of improvements would never have been made, 
%because I would never have thought of them or perceived 
%why they were important.
%}{Donald E. Knuth}
%



% r.3 full title page
\maketitle


%% v.4 copyright page
%\newpage
%\begin{fullwidth}
%~\vfill
%\thispagestyle{empty}
%\setlength{\parindent}{0pt}
%\setlength{\parskip}{\baselineskip}
%Copyright \copyright\ \the\year\ \thanklessauthor
%
%\par\smallcaps{Published by \thanklesspublisher}
%
%\par\smallcaps{tufte-latex.googlecode.com}
%
%\par Licensed under the Apache License, Version 2.0 (the ``License''); you may not
%use this file except in compliance with the License. You may obtain a copy
%of the License at \url{http://www.apache.org/licenses/LICENSE-2.0}. Unless
%required by applicable law or agreed to in writing, software distributed
%under the License is distributed on an \smallcaps{``AS IS'' BASIS, WITHOUT
%WARRANTIES OR CONDITIONS OF ANY KIND}, either express or implied. See the
%License for the specific language governing permissions and limitations
%under the License.\index{license}
%
%\par\textit{First printing, \monthyear}
%\end{fullwidth}

% r.5 contents
%\tableofcontents
%\listoffigures
%\listoftables

%% r.7 dedication
%\cleardoublepage
%\vfill
%\begin{doublespace}
%\noindent\fontsize{18}{22}\selectfont\itshape
%\nohyphenation
%Dedicated to those who appreciate \LaTeX{} 
%and the work of \mbox{Edward R.~Tufte} 
%and \mbox{Donald E.~Knuth}.
%\end{doublespace}
%\vfill
%\vfill


\chapter{Introducción general}
En este primer capítulo explicaremos los objetivos que queremos
alcanzar y cuáles son las herramientas que vamos a utilizar. En 
los siguientes capítulos utilizaremos a fondo estas herramientas.
Profundizaremos en la información que como químicos podemos extraer
de la Mecánica Cuántica y en las relaciones que la computación 
permite establecer entre teoría y experimentos. 

En concreto el problema en que nos vamos a centrar es la determinación del 
espectro rotovibracional del monóxido de carbono (CO). A continuación 
determinaremos las variaciones en el espectro ultravioleta en presencia de 
diferentes disolventes.

A continuación en este manual introduciremos los fundamentos teóricos
y las herramientas que vamos a utilizar para alcanzar nuestros 
objetivos. Por un lado, la mecánica cuántica; por otro, la
computación, que se usa hoy en día para resolver las ecuaciónes 
mecanocuánticas. Éstas han sido incorporadas en un \textit{software}
de cálculo, \textbf{Gaussian03}. Pero antes de describir cómo vamos 
a hacer uso de este programa, presentaremos el sistema operativo 
\textbf{Linux}.


\section{Linux}
\subsection{Qué es Linux}
Linux es una familia de sistemas operativos (\textit{OS}) de código 
abierto organizados en torno al \textit{Linux kernel}, desarrollado
inicialmente por Linus Torvalds. La \textit{Free Software Foundation} 
usa el nombre GNU/Linux para referirse a la familia de sistemas 
operativos así como a distribuciones específicas, que incluyen el 
kernel pero además multitud de programas y librerías del proyecto
GNU. A pesar de que este sistema operativo no es el más extendido
para usuarios domésticos, domina en el ámbito de la supercomputación,
y ha sido la base para el desarrollo del Android OS.

\subsection{Comandos de Linux}
A pesar de que Linux dispone de un entorno gráfico análogo al de 
Windows o Mac OS X, la interfaz más importante entre el usuario y el 
sistema operativo en Linux es la \textbf{terminal}. En la Tabla~
\ref{tb:commands} mostramos algunos de los comandos más habituales.

\begin{table}[t!]
\centering
	\small
	\begin{tabular}{l|l|l}
		Comando & Descripción & Opciones \\
		\hline
		& & \\
		\texttt{ls} & Lista de archivos y directorios & \texttt{-l}, \texttt{-lt}, \texttt{-ltr} \\ 
		\texttt{pwd} & Escribe el \textit{path} & \\ 
		\texttt{df} & Mostrar la cantidad de espacio libre en disco & \texttt{}\\ 
		\texttt{du} & Mostrar uso de disco por parte de archivos y directorios & \texttt{}\\
		\texttt{man command} & Mostrar manual del programa \texttt{command}  & \texttt{}\\ 
		\texttt{grep} & Seleccionar un argumento  & \texttt{}\\

		\texttt{bg} & Mandar proceso al \textit{background}  & \texttt{}\\ 
		\texttt{fg} & Traer proceso al \textit{foreground}  & \texttt{}\\ 
		\texttt{touch file\_name} & Crear archivo  & \\ 
		\texttt{rm file\_name} & Borrar archivo  & \\ 
		\texttt{cat file\_name} & Mostrar archivo por pantalla & \\ 
		\texttt{more file\_name} & Mostrar gradualmente el archivo & \texttt{}\\ 
		\texttt{nano file\_name} & Mostrar el archivo con el editor \texttt{nano}  & \\ 
		\texttt{mkdir folder\_name} & Crear directorio  & \\ 
		\texttt{cd folder\_name} & Cambio de directorio \texttt{cd folder\_name}& \texttt{}\\ 
		\texttt{mv file1 file2} & Mover archivo  & \texttt{}\\ 
		\texttt{cp file1 file2} & Crear copia de \texttt{file1}  & \texttt{}\\ 
    \end{tabular}
    %\caption{Armónicos esféricos, dependientes de los números
    %cuánticos $l$ y $m_l$.}
    \label{tb:commands}
\end{table}

\section{Introducción a la Química Cuántica}

\chapter{Átomos}

\section{Método de Hartree Fock para átomos}
\subsection{Las ecuaciones que debemos resolver}
\subsection{Soluciones}
\section{Input para el programa Gaussian}
\section{Output de Gaussian}

\chapter{Moléculas: Introducción}

\chapter{Moléculas. Optimización de la geometría}

\chapter{Moléculas: Cálculo de frecuencias}

\chapter{Espectro rotovibracional del CO}

\chapter{Estados excitados del CO (espectro ultravioleta-visible)}



%\begin{thebibliography}{}
%\bibitem{atkins_depaula} Atkins, P and De Paula, J. 2006. ``Physical Chemistry, 8th Edition''. Oxford University Press.
%
%\bibitem{atkins} Atkins, P and Friedman, R. 2005. ``Molecular Quantum Mechanics, 4th Edition''. Oxford University Press.
%
%\bibitem{levine} Levine, I. N. 2013. ``Quantum
%Chemistry, 7th Edition''. Pearson.
%\end{thebibliography}

\end{document}
