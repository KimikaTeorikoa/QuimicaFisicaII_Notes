%%%%%%%%%%%%%%%%%%%%%%%%%%%%%%%%%%%%%%%%%%%%%%%%%%%%%%%%%%%%%%%%%%%%%%
% How to use writeLaTeX: 
%
% You edit the source code here on the left, and the preview on the
% right shows you the result within a few seconds.
%
% Bookmark this page and share the URL with your co-authors. They can
% edit at the same time!
%
% You can upload figures, bibliographies, custom classes and
% styles using the files menu.
%
% If you're new to LaTeX, the wikibook is a great place to start:
% http://en.wikibooks.org/wiki/LaTeX
%
%%%%%%%%%%%%%%%%%%%%%%%%%%%%%%%%%%%%%%%%%%%%%%%%%%%%%%%%%%%%%%%%%%%%%%
\documentclass{tufte-book}

\hypersetup{colorlinks}% uncomment this line if you prefer colored hyperlinks (e.g., for onscreen viewing)

%%
% Book metadata
%\title{A Tufte-Style Book\thanks{Thanks to Edward R.~Tufte for his inspiration.}}
\author{M. D. Fern\'andez \& D. De Sancho}
%\publisher{Publisher of This Book}
\usepackage[spanish, es-tabla]{babel}
\usepackage[utf8x]{inputenc}
\usepackage[T1]{fontenc}
\usepackage{chemfig}

\title{Qu\'imica F\'isica II - \\Pr\'acticas \\ Experimentales}
%%
% If they're installed, use Bergamo and Chantilly from www.fontsite.com.
% They're clones of Bembo and Gill Sans, respectively.
%\IfFileExists{bergamo.sty}{\usepackage[osf]{bergamo}}{}% Bembo
%\IfFileExists{chantill.sty}{\usepackage{chantill}}{}% Gill Sans

\usepackage{microtype}

%%
% Just some sample text
\usepackage{lipsum}

%%
% For nicely typeset tabular material
\usepackage{booktabs}

%%
% For graphics / images
\usepackage{graphicx}
\setkeys{Gin}{width=\linewidth,totalheight=\textheight,keepaspectratio}
\graphicspath{{graphics/}}

% The fancyvrb package lets us customize the formatting of verbatim
% environments.  We use a slightly smaller font.
\usepackage{fancyvrb}
\fvset{fontsize=\normalsize}

%%
% Prints argument within hanging parentheses (i.e., parentheses that take
% up no horizontal space).  Useful in tabular environments.
\newcommand{\hangp}[1]{\makebox[0pt][r]{(}#1\makebox[0pt][l]{)}}

%%
% Prints an asterisk that takes up no horizontal space.
% Useful in tabular environments.
\newcommand{\hangstar}{\makebox[0pt][l]{*}}

%%
% Prints a trailing space in a smart way.
\usepackage{xspace}

\decimalpoint


\setcounter{secnumdepth}{0}

% Allows for fancy math alignments
\usepackage{amsmath}
\usepackage{bm}
%\numberwithin{equation}{chapter}
%%

% Theorems
\newtheorem{theorem}{Postulado}
 
% Prints the month name (e.g., January) and the year (e.g., 2008)
\newcommand{\monthyear}{%
  \ifcase\month\or January\or February\or March\or April\or May\or June\or
  July\or August\or September\or October\or November\or
  December\fi\space\number\year
}

% Prints an epigraph and speaker in sans serif, all-caps type.
\newcommand{\openepigraph}[2]{%
  %\sffamily\fontsize{14}{16}\selectfont
  \begin{fullwidth}
  \sffamily\large
  \begin{doublespace}
  \noindent\allcaps{#1}\\% epigraph
  \noindent\allcaps{#2}% author
  \end{doublespace}
  \end{fullwidth}
}

% Inserts a blank page
\newcommand{\blankpage}{\newpage\hbox{}\thispagestyle{empty}\newpage}

\usepackage{units}

% Typesets the font size, leading, and measure in the form of 10/12x26 pc.
\newcommand{\measure}[3]{#1/#2$\times$\unit[#3]{pc}}

% Macros for typesetting the documentation
\newcommand{\hlred}[1]{\textcolor{Maroon}{#1}}% prints in red
\newcommand{\hangleft}[1]{\makebox[0pt][r]{#1}}
\newcommand{\hairsp}{\hspace{1pt}}% hair space
\newcommand{\hquad}{\hskip0.5em\relax}% half quad space
\newcommand{\TODO}{\textcolor{red}{\bf TODO!}\xspace}
\newcommand{\ie}{textit{i.\hairsp{}e.}\xspace}
\newcommand{\eg}{\textit{e.\hairsp{}g.}\xspace}
\newcommand{\na}{\quad--}% used in tables for N/A cells
\providecommand{\XeLaTeX}{X\lower.5ex\hbox{\kern-0.15em\reflectbox{E}}\kern-0.1em\LaTeX}
\newcommand{\tXeLaTeX}{\XeLaTeX\index{XeLaTeX@\protect\XeLaTeX}}
% \index{\texttt{\textbackslash xyz}@\hangleft{\texttt{\textbackslash}}\texttt{xyz}}
\newcommand{\tuftebs}{\symbol{'134}}% a backslash in tt type in OT1/T1
\newcommand{\doccmdnoindex}[2][]{\texttt{\tuftebs#2}}% command name -- adds backslash automatically (and doesn't add cmd to the index)
\newcommand{\doccmddef}[2][]{%
  \hlred{\texttt{\tuftebs#2}}\label{cmd:#2}%
  \ifthenelse{\isempty{#1}}%
    {% add the command to the index
      \index{#2 command@\protect\hangleft{\texttt{\tuftebs}}\texttt{#2}}% command name
    }%
    {% add the command and package to the index
      \index{#2 command@\protect\hangleft{\texttt{\tuftebs}}\texttt{#2} (\texttt{#1} package)}% command name
      \index{#1 package@\texttt{#1} package}\index{packages!#1@\texttt{#1}}% package name
    }%
}% command name -- adds backslash automatically
\newcommand{\doccmd}[2][]{%
  \texttt{\tuftebs#2}%
  \ifthenelse{\isempty{#1}}%
    {% add the command to the index
      \index{#2 command@\protect\hangleft{\texttt{\tuftebs}}\texttt{#2}}% command name
    }%
    {% add the command and package to the index
      \index{#2 command@\protect\hangleft{\texttt{\tuftebs}}\texttt{#2} (\texttt{#1} package)}% command name
      \index{#1 package@\texttt{#1} package}\index{packages!#1@\texttt{#1}}% package name
    }%
}% command name -- adds backslash automatically
\newcommand{\docopt}[1]{\ensuremath{\langle}\textrm{\textit{#1}}\ensuremath{\rangle}}% optional command argument
\newcommand{\docarg}[1]{\textrm{\textit{#1}}}% (required) command argument
\newenvironment{docspec}{\begin{quotation}\ttfamily\parskip0pt\parindent0pt\ignorespaces}{\end{quotation}}% command specification environment
\newcommand{\docenv}[1]{\texttt{#1}\index{#1 environment@\texttt{#1} environment}\index{environments!#1@\texttt{#1}}}% environment name
\newcommand{\docenvdef}[1]{\hlred{\texttt{#1}}\label{env:#1}\index{#1 environment@\texttt{#1} environment}\index{environments!#1@\texttt{#1}}}% environment name
\newcommand{\docpkg}[1]{\texttt{#1}\index{#1 package@\texttt{#1} package}\index{packages!#1@\texttt{#1}}}% package name
\newcommand{\doccls}[1]{\texttt{#1}}% document class name
\newcommand{\docclsopt}[1]{\texttt{#1}\index{#1 class option@\texttt{#1} class option}\index{class options!#1@\texttt{#1}}}% document class option name
\newcommand{\docclsoptdef}[1]{\hlred{\texttt{#1}}\label{clsopt:#1}\index{#1 class option@\texttt{#1} class option}\index{class options!#1@\texttt{#1}}}% document class option name defined
\newcommand{\docmsg}[2]{\bigskip\begin{fullwidth}\noindent\ttfamily#1\end{fullwidth}\medskip\par\noindent#2}
\newcommand{\docfilehook}[2]{\texttt{#1}\index{file hooks!#2}\index{#1@\texttt{#1}}}
\newcommand{\doccounter}[1]{\texttt{#1}\index{#1 counter@\texttt{#1} counter}}

% Generates the index
\usepackage{makeidx}
\makeindex

\begin{document}

% Front matter
%\frontmatter

% r.1 blank page
%\blankpage

%% v.2 epigraphs
%\newpage\thispagestyle{empty}
%\openepigraph{%
%The public is more familiar with bad design than good design.
%It is, in effect, conditioned to prefer bad design, 
%because that is what it lives with. 
%The new becomes threatening, the old reassuring.
%}{Paul Rand%, {\itshape Design, Form, and Chaos}
%}
%\vfill
%\openepigraph{%
%A designer knows that he has achieved perfection 
%not when there is nothing left to add, 
%but when there is nothing left to take away.
%}{Antoine de Saint-Exup\'{e}ry}
%\vfill
%\openepigraph{%
%\ldots the designer of a new system must not only be the implementor and the first 
%large-scale user; the designer should also write the first user manual\ldots 
%If I had not participated fully in all these activities, 
%literally hundreds of improvements would never have been made, 
%because I would never have thought of them or perceived 
%why they were important.
%}{Donald E. Knuth}
%



% r.3 full title page
\maketitle


%% v.4 copyright page
%\newpage
%\begin{fullwidth}
%~\vfill
%\thispagestyle{empty}
%\setlength{\parindent}{0pt}
%\setlength{\parskip}{\baselineskip}
%Copyright \copyright\ \the\year\ \thanklessauthor
%
%\par\smallcaps{Published by \thanklesspublisher}
%
%\par\smallcaps{tufte-latex.googlecode.com}
%
%\par Licensed under the Apache License, Version 2.0 (the ``License''); you may not
%use this file except in compliance with the License. You may obtain a copy
%of the License at \url{http://www.apache.org/licenses/LICENSE-2.0}. Unless
%required by applicable law or agreed to in writing, software distributed
%under the License is distributed on an \smallcaps{``AS IS'' BASIS, WITHOUT
%WARRANTIES OR CONDITIONS OF ANY KIND}, either express or implied. See the
%License for the specific language governing permissions and limitations
%under the License.\index{license}
%
%\par\textit{First printing, \monthyear}
%\end{fullwidth}

% r.5 contents
%\tableofcontents
%\listoffigures
%\listoftables

%% r.7 dedication
%\cleardoublepage
%\vfill
%\begin{doublespace}
%\noindent\fontsize{18}{22}\selectfont\itshape
%\nohyphenation
%Dedicated to those who appreciate \LaTeX{} 
%and the work of \mbox{Edward R.~Tufte} 
%and \mbox{Donald E.~Knuth}.
%\end{doublespace}
%\vfill
%\vfill



\chapter[Infrarrojo]{Espectroscopía de Infrarrojo}
\section{Introducción}
\subsection{El espectro infrarrojo del aire}
En esta primera práctica vamos a estudiar el espectro de rotación-vibración
de las moléculas presentes en el aire que respiramos. Para ello usaremos
espectroscopía en el infrarrojo, que permite estudiar la interacción de la
materia con la \textbf{radiación infrarroja} ($\lambda=0.7$ $\mu$m$-1$ mm) 
y así obtener espectros vibracionales con estructura fina rotacional. 

Los enlaces de las moléculas no son rígidos, sino que se comportan,
en una primera aproximación, como muelles, que pueden extenderse y contraerse.
Matemáticamente, estas vibraciones se pueden expresar en términos de unos
\textbf{modos normales}, que aparecen asociados a \textbf{frecuencias} ($\nu$)
o \textbf{números de onda} ($\tilde{\nu}=\nu/c=1/\lambda$) característicos. 
En el 
caso de las moléculas lineales, el número de modos normales de vibración 
es $3N-5$, mientras que para moléculas no lineales hay $3N-6$ modos normales.
Para que un modo vibracional de una molécula sea activo en el infrarrojo, 
su vibración debe resultar en un cambio de su \textbf{momento dipolar}.
Algunas vibraciones son completamente inactivas; es el caso de aquellas
vibraciones que resultan en un movimiento que no cambia el momento dipolar
de la molécula. Algunos enlaces son más activos que otros en el infrarrojo. 
En general, cuanto mayor es la polaridad de un enlace, mayor es su 
absorción en esta región del espectro. 

Como hemos dicho, en esta práctica nos centraremos en una muestra de aire.
Aunque  las moléculas que se encuentran en el aire en mayor concentración 
son el nitrógeno (N$_2$) y el oxígeno (O$_2$), estas moléculas no son activas 
en el infrarrojo. Al tratarse de moléculas diatómicas lineales ($N=2$), 
tienen un solo modo de vibración, el estiramiento del enlace, y este 
estiramiento no cambia el momento dipolar. Otras moléculas que están 
presentes en el aire sí son activas en el infrarrojo. Es el caso de las 
moléculas de monóxido de carbono (CO) y de dióxido de carbono (CO$_2$).
En el caso del CO, al ser lineal, también hay un solo modo de vibración,
correspondiente al estiramiento del enlace, que aparece asociado a un 
número de onda $\tilde{\nu}=2143$ cm$^{-1}$
en el espectro de vibración. El CO$_2$, en cambio, tiene cuatro modos 
normales de vibración, uno inactivo y tres activos. El modo inactivo es el 
\textbf{estiramiento simétrico}, que no resulta en un cambio en el momento
dipolar. Los modos activos son el \textbf{estiramiento asimétrico}, que
aparece a 2349 cm$^{-1}$, y la \textbf{flexión}, que es un modo degenerado y
aparece a 667 cm$^{-1}$.

\subsection{Modelos teóricos para las transiciones vibracionales}
Para el estudio de las transiciones que podemos observar en el infrarrojo
usaremos dos modelos teóricos que hemos estudiado en la parte teórica de la
asignatura de Química Física II. Las transiciones entre niveles 
vibracionales las estudiaremos usando el modelo del \textbf{oscilador
armónico}. Así, la energía de los niveles vibracionales depende del 
número cuántico $v$,
\begin{equation}
    E_v=\bigg(v+\frac{1}{2}\bigg)h\nu
\end{equation}
donde $v=0,1,2...$. 
Dividiendo por $hc$, la energía vibracional se puede expresar en 
\textbf{números de onda}, que expresamos como $\tilde{\nu}$ o
$\tilde{\omega}_e$ y tienen unidades de cm$^{-1}$. La ecuación 
resultante es
\begin{equation}
%    G_v=\bigg(v+\frac{1}{2}\bigg)\tilde{\nu}
    G_v=\bigg(v+\frac{1}{2}\bigg)\tilde{\omega}_e
\end{equation}
La frecuencia (o el número de onda) está relacionado con parámetros
fundamentales de la molécula
\begin{equation}
    \tilde{\nu}=\frac{1}{2\pi c}\Big(\frac{k}{\mu}\Big)^{1/2},
    \label{eq:nu}
\end{equation}
donde $k$ es la constante de fuerza del enlace y $\mu$ es la masa 
reducida, que en el caso de una molécula diatómica es 
\begin{equation}
    \mu=\frac{m_1m_2}{m_1+m_2}.
\end{equation}
La regla de selección para las transiciones vibracionales es 
$\Delta v=\pm 1$, tratándose $+1$ del proceso de absorción 
y $-1$ del proceso de emisión. La energía correspondiente a
estas transiciones es por tanto siempre la misma, 
\begin{equation}
    \Delta G=G(v+1)-G(v) = \bigg(v+1+\frac{1}{2}\bigg)\tilde{\nu} -
     \bigg(v+\frac{1}{2}\bigg)\tilde{\nu} = \tilde{\nu}
\end{equation}
Dada la energía disponible a condiciones de temperatura 
ambiente, $k_BT/hc=200$ cm$^{-1}$ , la mayor parte de las 
moléculas están en sus estados fundamentales de acuerdo con
la distribución de Boltzmann. Por tanto, al irradiar una 
muestra típicamente observamos la \textbf{transición fundamental}, 
($v=0\rightarrow v=1$).

La validez del modelo del oscilador armónico es sólo aproximada.
En realidad, para números cuánticos elevados, la distorsión del
potencial es considerable y por ello hay que incluir correcciones
de anarmonicidad. Así, el nivel energético correspondiente al 
número cuántico $v$ es
\begin{equation}
    G(v) = \bigg(v+\frac{1}{2}\bigg)\tilde{\omega}_e - \bigg(v+\frac{1}{2}\bigg)^2x_e\tilde{\omega}_e.
\end{equation}
En esta expresión hemos definido la \textbf{constante de anarmonicidad}, 
$x_e=\tilde{\nu}/4D_e$, donde $D_e$ es la profundidad del potencial
tipo Morse, que representa mejor la dependencia de la energía con
la distancia. La anarmonicidad también determina la aparición
de \textbf{sobretonos}, correspondientes a las transiciones
$v=0\rightarrow 2$ o $v=0\rightarrow 3$, aunque estas transiciones 
están en principio prohibidas en la aproximación del oscilador armónico.

En el caso de las moléculas poliatómicas, cada modo normal de vibración
funciona como un oscilador armónico independiente. Para cada modo 
podremos definir una constante de muelle, lo que resulta en un
campo de fuerza.

\subsection{Modelos teóricos para las transiciones rotacionales}
Para las transiciones rotacionales usamos los resultados  
correspondientes al \textbf{rotor rígido}. En este modelo, 
la energía está cuantizada en función del valor del número cuántico
$J$,
\begin{equation}
    E_J=\frac{h^2}{8\pi^2I}J(J+1),
\end{equation}
donde $I$ es el momento de inercia y $J$ puede adoptar los valores 
$J=0,1,2...$. Como hemos hecho en el caso de los niveles energéticos
vibracionales, podemos expresar la energía rotacional en números de 
onda,
\begin{equation}
    F=BJ(J+1)
    \label{eq:F}
\end{equation}
donde estamos definiendo la constante rotacional $B$. Este
tipo de transiciones aparecen en el espectro de microondas y su
regla de selección es $\Delta J=\pm 1$. La energía correspondiente 
a estas transiciones es
\begin{align}
    &\tilde{\nu}_{J\rightarrow J+1} = 2B(J+1)\\
    &\tilde{\nu}_{J\rightarrow J-1} = -2BJ.
\end{align}

La aproximación del rotor rígido no es exacta, y por tanto es necesario
introducir correcciones debidas al movimiento centrífugo de las moléculas.
Esta \textbf{distorsión centrífuga} en los niveles energéticos depende 
de la frecuencia vibracional, pues son mayores cuando el enlace es más 
elástico. La corrección empírica para la energía resulta en la siguiente
expresión
\begin{equation}
    F(J)=BJ(J+1) - D_JJ^2(J+1)^2,
\end{equation}
donde $D_J$ es la \textbf{constante de distorsión centrífuga}, 
\begin{equation}
    D_J=4B^3/\tilde{\nu}^2\label{eq:centrif}    
\end{equation}
Si consideramos este efecto en la energía de la transición, obtenemos
\begin{align}
    &\tilde{\nu}_{J\rightarrow J+1} = 2B(J+1) - 4D_J(J)^3\\
    &\tilde{\nu}_{J+1\rightarrow J} = -2BJ + 4D_JJ^3
\end{align}

\subsection{El espectro roto-vibracional}
En un \textbf{espectro roto-vibracional} como los que estudiaremos 
en esta práctica usando espectroscopía infrarroja, observamos 
principalmente la transición vibracional fundamental, 
$v=0\rightarrow v=1$. En este espectro aparecen además 
transiciones rotacionales en forma de \textbf{estructura fina}. 
De tal manera que las bandas que observamos corresponden
a transiciones entre niveles energéticos 
\begin{equation}
    s(v, J) = G(v) + F(J) = \Big(v+\frac{1}{2}\Big)\tilde{\nu} + BJ(J+1)
\end{equation}
Dado que el momento de inercia de la molécula cambia con la extensión
del enlace, la constante rotacional depende en realidad del número
cuántico $v$, de la siguiente manera
\begin{equation}
B_v = B_{eq} - \alpha\bigg(v+\frac{1}{2}\bigg)
\end{equation}
Por tanto, los niveles energéticos implicados en las transiciones, si no 
consideramos la distorsión centrífuga ni la anarmonicidad, son
\begin{equation}
    s(v, J) = G(v) + F(J) = \Big(v+\frac{1}{2}\Big)\tilde{\nu} + B_vJ(J+1)
    \label{eq:svj}
\end{equation}
Si incorporamos las correcciones de anarmonicidad y centrífuga
los niveles energéticos en unidades de números de onda son
\begin{equation}
    s(v, J) = \Big(v+\frac{1}{2}\Big)\tilde{\nu} - \Big(v+\frac{1}{2}\Big)^2\tilde{\nu}x_e + B_vJ(J+1)-DJ^2(J+1)^2
\label{eq:svjexp}
\end{equation}
A partir de las expresiones \ref{eq:svj} y \ref{eq:svjexp} para 
los niveles energéticos podemos obtener las frecuencias correspondientes
a las transiciones en el infrarrojo. 

Como resultado del acoplamiento entre vibración y rotación, en torno a
la frecuencia de vibración fundamental, $\tilde{\nu}$, encontramos 
múltiples picos, que constituyen la \textbf{rama P} (para $\Delta J=-1$, 
con $\tilde{\nu}_P<\tilde{\nu}$) y la \textbf{rama R} (para 
$\Delta J=+1$, con $\tilde{\nu}_R>\tilde{\nu}$). Además, en 
algunos casos, para los modos normales que corresponden a un 
movimiento perpendicular al eje de la molécula, aparece una rama 
más, centrada en $\tilde{\nu}$ (\textbf{rama Q}). Para todas estas
transiciones, podemos obtener expresiones compactas para sus 
frecuencias
\begin{align}
    &\tilde{\nu}_P(J)=\tilde{\nu} - (B_1+B_0)J + (B_1-B_0)J^2 \label{eq:nuP}\\
    &\tilde{\nu}_Q(J)=\tilde{\nu} + (B_1-B_0)J(J+1) \label{eq:nuQ}\\
    &\tilde{\nu}_R
    (J)=\tilde{\nu} + (B_1+B_0)(J+1) + (B_1-B_0)(J+1)^2 \label{eq:nuR}
\end{align}

\subsection{El método de las diferencias entre combinaciones}
A partir de los valores de $\tilde{\nu}$ para las transiciones
correspondientes a la rama P y la rama Q, podemos obtener un 
gran número de parámetros moleculares. El método de análisis
consiste en combinar valores concretos de la rama P y la
rama Q para así poder obtener las constantes rotacionales.

La primera combinación es la de transiciones de ambas ramas
que proceden del mismo nivel vibracional. Si restamos ambas
frecuencias, obtenemos
\begin{align}
\begin{split}
    \tilde{\nu}_R(J) - \tilde{\nu}_P(J)&=
    (B_1+B_0)(J+1) + (B_1-B_0)(J+1)^2 - \\
    &- (B_1-B_0)J^2 + (B_1 + B_0)J =4B_1\big(J+\frac{1}{2}\big)
\end{split}
\end{align}
Por tanto, representando $\big(J+1/2\big)$ frente a
$\tilde{\nu}_R(J) - \tilde{\nu}_P(J)$ obtendremos una
recta cuya pendiente es cuatro veces el valor de $B_1$.
Asimismo, podemos calcular la diferencia entre transiciones
que llegan al mismo nivel,
\begin{align}
    \begin{split}
    \tilde{\nu}_R(J-1) - \tilde{\nu}_P(J+1)&=
    (B_1+B_0)J - (B_1-B_0)J^2 - \\
    - (B_1-B_0)(J+1)^2 &+ (B_1 + B_0)(J+1) =4B_0\big(J+\frac{1}{2}\big)
    \end{split}
\end{align}
De la misma manera que hemos hecho para $B_1$,  a partir
de la representación de $(J+1/2)$ frente a $\tilde{\nu}_R(J-1) -
\tilde{\nu}_P(J+1)$ podemos obtener el valor de $B_0$.

\section{Obtención de parámetros a partir de los espectros}
Una vez obtenidos los valores de las constantes
vibracionales $B_0$ y $B_1$ podemos estimar el origen de banda,
es decir, el valor de $\tilde{\nu}$, dado que a partir de la 
Ecuación \ref{eq:nuP}, podemos escribir que
\begin{equation}
    \tilde{\nu} = \tilde{\nu}_P(1) + 2B_0 
\end{equation}
y a partir de la Ecuación \ref{eq:nuR},
\begin{equation}
    \tilde{\nu} = \tilde{\nu}_R(0) - 2B_1 
\end{equation}
Usando el valor bibliográfico correspondiente al primer sobretono,
$\tilde{\nu}(0\rightarrow 2)$, podemos calcular la constante de 
distorsión centrífuga. Para ello planteamos el siguiente
sistema de ecuaciones
\begin{align}
\begin{split}
\tilde{\nu}(0\rightarrow 1) = \tilde{\nu} - 2\tilde{\nu}x_e\\
\tilde{\nu}(0\rightarrow 2) = 2\tilde{\nu} - 6\tilde{\nu}x_e
\end{split}
\end{align}
A partir del valor de la frecuencia fundamental, podemos
obtener parámetros moleculares como la constante de fuerza
del oscilador armónico usando la Ecuación \ref{eq:nu}.
También a partir de la constante rotacional podemos obtener
el valor del momento de inercia usando la relación 
$B=h/8\pi^2Ic$. Y como el momento de inercia para una molécula
lineal diatómica es $I=\mu R^2$, podemos calcular la 
distancia de enlace. Finalmente, podemos obtener la
constante de dispersión centrífuga a partir de su definición
en la Ecuación \ref{eq:centrif}.

En el caso del monóxido de carbono, para el nivel $v=0$ nos
encontramos que sólo están poblados los $J$
pares mientras que en el caso de $v=1$, sólo están poblados los
$J$ impares. Por este motivo va a haber una serie de bandas no
visibles, correspondientes a las bandas $R_1$ y $P_1$.

\chapter[UV-visible]{Espectroscopía Ultravioleta-Visible}
En la segunda práctica utilizaremos otro tipo de espectroscopía,
la que usa radiación ultravioleta-visible. Este tipo de radiación,
de menor longitud de onda y por tanto más energética que la 
infrarroja, afecta a los niveles electrónicos de las moléculas.
Lamentablemente, en este caso no podemos obtener expresiones
analíticas compactas para las energías 
correspondientes a las transiciones. 

\begin{table}[b!]
\centering
    \begin{tabular}{|c|c|c|c|}
    \hline
     Cromóforo & Transición &  $\lambda_{max}$ (nm) & Intensidad \\
     \hline
     \chemfig{C=C} & $\pi\rightarrow\pi^\star$ & 190 & media\\
    \hline  
     \chemfig{C=O} & $n\rightarrow\pi^\star$& 300 & débil\\
      & $n\rightarrow\sigma^\star$& 190 & intensa\\
    \hline  
     \chemfig{C=C-C=C} & $\pi\rightarrow\pi^\star$ & 210 & intensa \\
    \hline  
     \chemfig{C=C-C=O} & $n\rightarrow\pi^\star$ & 330 & débil \\
     \chemfig{C=C-C=O} & $\pi\rightarrow\pi^\star$ & 210 & intensa \\
      & $\pi\rightarrow\pi^\star$ & 210 & intensa \\
    \hline  
     & $\pi\rightarrow\pi^\star$ & 180 & muy intensa \\
& $\pi\rightarrow\pi^\star$ & 200 & intensa \\
      \chemfig{*6(=-=-=-)} & $\pi\rightarrow\pi^\star$ & 255 & débil\\
\hline
\end{tabular}
\caption{Algunos cromóforos de moléculas orgánicas, tipos de
electrones implicados, las longitudes de onda aproximadas e intensidad de las bandas correspondientes.}\label{tab:chrom}
\end{table}

En el caso de moléculas poliatómicas, podemos relacionar 
las transiciones electrónicas observadas a excitaciones de
electrones pertenecientes a tipos de átomos específicos. 
Las partes de las moléculas responsables de transiciones 
que podemos observar en esta región del espectro 
electromagnético se llaman cromóforos. En la Tabla 
\ref{tab:chrom} mostramos algunos cromóforos de moléculas
orgánicas a las que a menudo se aplica este tipo de
espectroscopía, los tipos de orbitales implicados y 
las longitudes de onda aproximadas a las que se observan. 
Típicamente, observamos transiciones entre el orbital molecular
ocupado de más alta energía y el orbital desocupado de más
baja energía. En función de los diferentes tipos de orbitales
moleculares implicados en las transiciones, las transiciones
serán más o menos energéticas. 

En el caso de los enlaces \chemfig{C=C} observamos la excitación
de electrones $\pi$ que llegan a orbitales $\pi^\star$ 
antienlazantes. A estas transiciones se las llama
$\pi\rightarrow\pi^\star$ y su energía corresponde al espectro UV. 
Cuando el doble enlace se encuentra en una cadena conjugada, 
las energías de los orbitales moleculares se acercan entre sí
y pueden llegar al rango del visible. En el caso del enlace
\chemfig{C=O} nos encontramos un electrón no enlazante $n$
que puede excitarse para llegar a un orbital vacío $\pi^\star$
antienlazante (transición $n\rightarrow\pi^\star$). Estas
transiciones en realidad están prohibidas por simetría y por
tanto suelen ser bastante débiles.

En esta práctica partimos de la longitud de onda de absorción
del acetaldehido y una serie de compuestos en ciclohexano. 
Para una serie de compuestos carbonílicos, 
diferentes tipos de sustituyentes modificarán por efecto 
inductivo o por efecto de resonancia la longitud de onda
de absorción del enlace. Asimismo, mediremos la absorción
de la acetona en diferentes disolventes, que pueden formar 
en distinto grado enlaces de hidrógeno que estabilizan los 
electrones 
$n$.
%\begin{thebibliography}{}
%\bibitem{atkins_depaula} Atkins, P and De Paula, J. 2006. ``Physical Chemistry, 8th Edition''. Oxford University Press.
%
%\bibitem{atkins} Atkins, P and Friedman, R. 2005. ``Molecular Quantum Mechanics, 4th Edition''. Oxford University Press.
%
%\bibitem{levine} Levine, I. N. 2013. ``Quantum
%Chemistry, 7th Edition''. Pearson.
%\end{thebibliography}

\end{document}
