% ****** Start of file apssamp.tex ******
%
%   This file is part of the APS files in the REVTeX 4.1 distribution.
%   Version 4.1r of REVTeX, August 2010
%
%   Copyright (c) 2009, 2010 The American Physical Society.
%
%   See the REVTeX 4 README file for restrictions and more information.
%
% TeX'ing this file requires that you have AMS-LaTeX 2.0 installed
% as well as the rest of the prerequisites for REVTeX 4.1
%
% See the REVTeX 4 README file
% It also requires running BibTeX. The commands are as follows:
%
%  1)  latex apssamp.tex
%  2)  bibtex apssamp
%  3)  latex apssamp.tex
%  4)  latex apssamp.tex
%
\documentclass[notitlepage, amsmath,amssymb,
 aps,12pt,tightenlines]{revtex4-1}
%\documentclass[%
% reprint,
%superscriptaddress,
%groupedaddress,
%unsortedaddress,
%runinaddress,
%frontmatterverbose, 
%preprint,
%showpacs,preprintnumbers,
%nofootinbib,
%nobibnotes,
%bibnotes,
 %amsmath,amssymb,
 %aps,
%pra,
%prb,
%rmp,
%prstab,
%prstper,
%floatfix,
%]{revtex4-1}
\usepackage[spanish, es-tabla]{babel}
\usepackage[utf8x]{inputenc}
\usepackage[T1]{fontenc}
\usepackage{txfonts}
\usepackage{palatino}
\usepackage{graphicx}% Include figure files
\usepackage{dcolumn}% Align table columns on decimal point
\usepackage{xcolor}
\usepackage{bm}% bold math
%\usepackage{hyperref}% add hypertext capabilities
%\usepackage[mathlines]{lineno}% Enable numbering of text and display math
%\linenumbers\relax % Commence numbering lines

%\usepackage[showframe,%Uncomment any one of the following lines to test 
%%scale=0.7, marginratio={1:1, 2:3}, ignoreall,% default settings
%%text={7in,10in},centering,
%%margin=1.5in,
%%total={6.5in,8.75in}, top=1.2in, left=0.9in, includefoot,
%%height=10in,a5paper,hmargin={3cm,0.8in},
%]{geometry}

\begin{document}

\title{Desarrollo matemático de las soluciones del ion H$_2^+$ 
en la teoría de orbitales moleculares}% Force line breaks with \\

\author{David De Sancho}
 \email{david.desancho@ehu.eus}
 \affiliation{Departamento de Polímeros y Materiales Avanzados:
 Física, Química y Tecnología\\
 Universidad del País Vasco\\
 Donostia - San Sebasti\'an}

%\date{\today}% It is always \today, today,
             %  but any date may be explicitly specified
\maketitle

\subsection*{Introducción}
El problema que tenemos entre manos es identificar las soluciones
de la ecuación de Schrödinger para el ion H$_2^+$. Las variables 
relevantes para este problema son las coordenadas $r_A$ y $r_B$,
correspondientes a las distancias entre el (único) electrón y
los núcleos A y B. Recordemos, asimismo, que la distancia entre los
núcleos $R$ permanece constante. Esto es consecuencia de 
la aproximación de
Born-Oppenheimer, por la que asumimos que los núcleos 
permanecen fijos en la escala de tiempo en que se produce
el movimiento de los electrones. $R$
es por tanto un parámetro y para diferentes valores de este parámetro
obtendremos diferentes valores propios para las funciones propias
del hamiltoniano.

\subsection*{Hamiltoniano}
Como en casos anteriores, escribimos el Hamiltoniano usando los 
términos de energía cinética y potencial correspondientes
\begin{equation}
    \hat{H}=-\frac{\hbar^2}{2m_e}\nabla^2 -
    \frac{e^2}{4\pi \varepsilon_0r_A} - 
    \frac{e^2}{4\pi \varepsilon_0r_B} +
    \frac{e^2}{4\pi \varepsilon_0R} 
\end{equation}
En el lado derecho de la igualdad en esta expresión, el primer término
corresponde a la energía cinética del electrón; el segundo y tercero,
a las interacciones del electrón con los núcleos A y B, respectivamente;
y el cuarto, a la repulsión entre los dos núcleos, que será una
constante para cada valor del parámetro $R$ (claramente, este último
término será mayor cuando la distancia entre los núcleos sea más
corta). Por conveniencia matemática, utilizaremos unidades atómicas
para simplificar el hamiltoniano, resultando en la siguiente expresión
\begin{equation}
    \hat{H}=-\frac{1}{2}\nabla^2 -
    \frac{1}{r_A} - 
    \frac{1}{r_B} +
    \frac{1}{R} 
\end{equation}
Como de costumbre, este hamiltoniano tendrá una serie de soluciones, sus
funciones propias, cuyos autovalores serán las energías 
\begin{equation}
     \hat{H}\Psi_j(r_A, r_B; R) = E_j\Psi_j(r_A, r_B; R) 
     \label{eq:schrodinger}
\end{equation}

\subsection*{Combinación lineal de orbitales atómicos}
Para resolver este problema de valores propios, usamos la teoría 
de orbitales moleculares. En esta teoría, las soluciones se 
construyen como combinaciones lineales de orbitales atómicos.
En el caso del ion molecular H$_2^+$ usamos las soluciones 
correspondientes a los orbitales atómicos $1\mathrm{s}$.
Así, en general, usamos como función de prueba las siguientes
soluciones
\begin{equation}
    \Psi_\pm=c_11\mathrm{s_A} \pm c_21\mathrm{s_B}
    \label{eq:lcao}
\end{equation}
donde $c_1$ y $c_2$ son los coeficientes de los orbitales 
atómicos, que en este casos son iguales. En lo que resta, 
nos centraremos exclusivamente en la solución $\Psi_+$.

\subsection*{Cálculo de la energía}
Si multiplicamos ambos lados de la Ecuación 
\ref{eq:schrodinger} por $\Psi_+^\star$, integramos,
y reorganizamos términos podemos obtener la expresión 
para la energía
\begin{equation}
    E_+=\frac{\int d\mathbf{r}\Psi_+^\star\hat{H}\Psi_+}{\int d\mathbf{r}\Psi_+^\star\Psi_+}
    \label{eq:exp}
\end{equation}

Podemos resolver en primer lugar la integral correspondiente
al denominador de la Ecuación \ref{eq:exp}. Así, sustituyendo
la combinación lineal positiva de la expresión \ref{eq:lcao} 
con los coeficientes
\begin{equation}
\begin{split}
    \int d\mathbf{r}\Psi_+^\star\Psi_+ = &
    \int d\mathbf{r}(1\mathrm{s^\star_A}+1\mathrm{s^\star_B})(1\mathrm{s_A} +\mathrm{s_B})\\
    =& \color{red}{\int d\mathbf{r}1\mathrm{s^\star_A}1\mathrm{s_A}+
    \int d\mathbf{r}1\mathrm{s^\star_B}1\mathrm{s_B}}+
    {\color{blue}\int d\mathbf{r}1\mathrm{s^\star_A}1\mathrm{s_B}+
    \int d\mathbf{r}1\mathrm{s^\star_B}1\mathrm{s_A}}
    \label{eq:denom}
\end{split}
\end{equation}
Los dos primeros términos en el lado derecho de esta
expresión, mostrados en color rojo, corresponden a la
integración de la densidad  de probabilidad para los
orbitales atómicos $1\mathrm{s}$ centrados en las
posiciones de los núcleos A y B. Su valor, 
al estar estas funciones normalizadas, es 
$\int d\mathbf{r}1\mathrm{s^\star_A}1\mathrm{s_A}=\int d\mathbf{r}1\mathrm{s^\star_B}1\mathrm{s_B}=1$. 
Los otros dos términos, mostrados en azul, son lo que
denominamos la integral de solapamiento,
\begin{equation}
S(R) = \int d\mathbf{r}1\mathrm{s^\star_A}1\mathrm{s_B}=
    \int d\mathbf{r}1\mathrm{s^\star_B}1\mathrm{s_A}
\label{eq:overlap}
\end{equation}
El valor de esta integral depende de la distancia $R$
entre los núcleos, dado que el solapamiento definido por 
el producto entre las funciones atómicas será 0 si los núcleos
están separados entre sí y será 1 en caso de que los núcleos
estén superpuestos. Por tanto, así obtenemos el valor de 
la Ecuación \ref{eq:denom}
\begin{equation}
    \int d\mathbf{r}\Psi_+^\star\Psi_+ = 2 + 2S = 2(1+S)
\end{equation}

Claramente, el numerador es mucho más complicado, al tener que
aplicar sobre la función del orbital molecular todos los términos
del hamiltoniano:
\begin{equation}
    \begin{split}
        \int d\mathbf{r}\Psi_+^\star\hat{H}\Psi_+=&
       \int d\mathbf{r}(1\mathrm{s^\star_A}+1\mathrm{s^\star_B})\bigg( -\frac{1}{2}\nabla^2 -
    \frac{1}{r_A} - 
    \frac{1}{r_B} +
    \frac{1}{R}\bigg)(1\mathrm{s_A}+1\mathrm{s_B}) = \\
    & \int d\mathbf{r}(1\mathrm{s^\star_A}+1\mathrm{s^\star_B})\bigg( -\frac{1}{2}\nabla^2 -
    \frac{1}{r_A} - 
    \frac{1}{r_B} +
    \frac{1}{R}\bigg)1\mathrm{s_A} + \\
    & \int d\mathbf{r}(1\mathrm{s^\star_A}+1\mathrm{s^\star_B})\bigg( -\frac{1}{2}\nabla^2 -
    \frac{1}{r_A} - 
    \frac{1}{r_B} +
    \frac{1}{R}\bigg)1\mathrm{s_B}
    \end{split}
    \label{eq:num}
\end{equation}
Si desarrollamos esta ecuación veremos que en ella encontramos
las soluciones para el hamiltoniano electrónico del hidrógeno
\begin{align}
    \bigg( -\frac{1}{2}\nabla^2 - \frac{1}{r_A}\bigg)1\mathrm{s_A}=E_\mathrm{1s}1\mathrm{s_A}\\
    \bigg( -\frac{1}{2}\nabla^2 - \frac{1}{r_B}\bigg)1\mathrm{s_B}=E_\mathrm{1s}1\mathrm{s_B}
\end{align}
Podemos sustituir estas expresiones en la Ecuación 
\ref{eq:num}, con lo cual obtenemos 
\begin{equation}
\begin{split}
    \int d\mathbf{r}\Psi_+^\star\hat{H}\Psi_+=
     & \int d\mathbf{r}
     (1\mathrm{s^\star_A} + 1\mathrm{s^\star_B})
     \bigg( E_\mathrm{1s} - \frac{1}{r_B} +\frac{1}{R}\bigg)
     1\mathrm{s_A} + \\
     & \int d\mathbf{r}(1\mathrm{s^\star_A}+1\mathrm{s^\star_B})
     \bigg( E_\mathrm{1s} - \frac{1}{r_A}  +\frac{1}{R}\bigg)
     1\mathrm{s_B} \\
     = & 2E_\mathrm{1s}(1 + S) + 
     \color{red}{\int d\mathbf{r} 1\mathrm{s^\star_A} \bigg(-\frac{1}{r_B}+\frac{1}{R}\bigg) 1\mathrm{s_A}}
     +
     \int d\mathbf{r} 1\mathrm{s^\star_B} \bigg(-\frac{1}{r_A}+\frac{1}{R}\bigg) 1\mathrm{s_B}
     \\
     + & {\color{blue}\int d\mathbf{r} 1\mathrm{s^\star_B} \bigg(-\frac{1}{r_B}+\frac{1}{R}\bigg) 1\mathrm{s_A} 
     +  \int d\mathbf{r} 1\mathrm{s^\star_A}   
     \bigg(-\frac{1}{r_A}+\frac{1}{R}\bigg) 1\mathrm{s_B}}
\end{split}
\label{eq:num2}
\end{equation}
En esta expresión, tenemos dos términos,
en rojo, que podemos resolver de manera 
idéntica por separado. Así, podemos escribir,
\begin{equation}
    {\color{red}\int d\mathbf{r} 1\mathrm{s^\star_A} \bigg(-\frac{1}{r_B}+\frac{1}{R}\bigg) 1\mathrm{s_A}} = 
    -\int d\mathbf{r} 1\mathrm{s^\star_A} \frac{1}{r_B}1\mathrm{s_A}
    +
    \int d\mathbf{r} 1\mathrm{s^\star_A} \frac{1}{R}1\mathrm{s_A}= -j+\frac{1}{R}
\end{equation}
Aquí estamos definiendo la integral de Coulomb,
$j$, que representa la interacción
electrostática entre la densidad en torno 
al orbital 1$\mathrm{s_A}$ y el núcleo del
átomo B. 
Por otro lado, en la Ecuación \ref{eq:num2}
encontramos otros dos términos análogos pero con índices
cruzados, que representamos en azul, y resolvemos de la
siguiente manera
\begin{equation}
    {\color{blue}\int d\mathbf{r} 1\mathrm{s^\star_B} \bigg(-\frac{1}{r_B}+\frac{1}{R}\bigg) 1\mathrm{s_A}} = 
    -\int d\mathbf{r} 1\mathrm{s^\star_B} \frac{1}{r_B}1\mathrm{s_A}
    +
    \int d\mathbf{r} 1\mathrm{s^\star_B} \frac{1}{R}1\mathrm{s_A}= -k+\frac{S}{R}
\end{equation}
donde estamos definiendo la integral de intercambio, $k$, 
y usando la integral de solapamiento (Ec. \ref{eq:overlap}).
Para interpretar la integral de intercambio, que implica
un efecto puramente mecanocuántico, podemos pensar que 
representa la interacción entre el exceso de densidad 
debido al solapamiento de los orbitales atómicos con 
la carga positiva de los núcleos.
Dado que en el numerador de la Ecuación \ref{eq:num2}
 tenemos dos integrales ``rojas'' y dos ``azules'', 
 el resultado del numerador  es
\begin{equation}
    \int d\mathbf{r}\Psi_+^\star\hat{H}\Psi_+=
2E_\mathrm{1s}(1 + S) + 2\bigg(-j+\frac{1}{R}\bigg) + 2\bigg( -k+\frac{S}{R}\bigg)
\end{equation}

Finalmente, combinando numerador y denominador obtenemos la
solución final de la Ecuación \ref{eq:exp},
\begin{equation}
    E = \frac{2E_\mathrm{1s}(1 + S) + 2\bigg(-j+\frac{1}{R}\bigg) + 2\bigg( -k+\frac{S}{R}\bigg)}{2 + 2S} = 
    E_\mathrm{1s} + \frac{1}{R} - \frac{j+k}{1+S}
\end{equation}

\end{document}
%
% ****** End of file apssamp.tex ******